\chapter*{前言}

在数学中, 对于每一个数学对象(例如极限), 我们会例行公事般地考虑它的一些常见的性质. 比如说, 这个对象最基本的例子是什么, 这种对象是否存在, 如果存在的话它是否具有唯一性, 它的子对象和商对象(如果有的话)都具有什么性质(比如说遗传了原来的对象的什么性质), 这个对象的可计算性以及在特定映射下的行为等等.




这份讲义cover的内容: 

- “抽象废话”范畴论, 当然只有一点点

- 作为工具和灵感来源的矩阵理论

- 几乎同构于矩阵理论的线性映射理论, 但我们会研究一些无限维情况

- 经典的群结构$\Z / n\Z$, 以及一般的群论

- 经典的环结构$\F [x]$, 以及初步的环论

- 域理论

- 初步的模论

换句话说, 本讲义大致覆盖高等代数与基础抽象代数的内容. 





抽象的代数需要直觉, 例如“自由度”“置换”等想法. 用具体详实的例子引出抽象理论. 


很多想法都是读者所熟悉的, 例如数论中的中国剩余定理, 多项式理论中的因式分解等. 因此这份讲义将不会侧重解释这些初等理论, 而是试图将其推广. 


