\chapter{初等数论的基本结果}

\section{整除}

\subsection{整数的可除性}

\begin{definition}{整除}
	设$a,b$为任意两个整数$(b \neq 0)$,若存在一个整数$q$使等式$$a=bq$$成立,称$b$\textbf{整除}$a$(或$a$被$b$整除),记作$b \mid a$,此时称$b$为$a$的\textbf{因数}. \\
	若满足上述等式的整数$q$不存在,称$b$\textbf{不能整除}$a$(或$a$不能被$b$整除),记作$b \nmid a$.
\end{definition}

由定义可知整除满足以下基本性质:

\begin{theorem}{整除的基本性质}
	对任意整数$x,y,z$,有 \\
	(1)自反性:$x \mid x$. \\
	(2)传递性:若$x \mid y$且$y \mid z$,有$x \mid z$. \\
	(3)反对称性:若$x \mid y$且$y \mid x$则$|x|=|y|$.
\end{theorem}
\begin{remark}
	实际上,满足自反性、传递性与反对称性的二元关系被称作“偏序关系”,例如集合的包含关系.
\end{remark}
\begin{remark}
	反对称性可以进行推广:若$x \mid y$且$y \neq 0$,则$|x| \leq |y|$.这一条性质可以利用整除关系得到大小关系,在解决整除方程时很有用.
\end{remark}

整除还有一些比较自然的性质:

\begin{theorem}{整除的性质}
	(1)(整除与线性组合的关系)若$x \mid y_i~(i=1,\cdots ,n)$,则$x \mid \sum_{i=1}^{n} b_i \cdot y_i$,其中$b_i \in \mathbb{Z}~(i=1,2,\cdots ,n)$. \\
	(2)对于任意的$z \neq 0$,$x \mid y \Leftrightarrow xz \mid yz$. \\
	(3)$n$个连续整数中有且仅有一个是$n$的倍数. \\
	(4)任意$n$个连续整数之积一定是$n!$的倍数.
\end{theorem}

整除的性质固然比较良好,现在我们尝试将它推广到一般情况.

\begin{theorem}{带余除法}
	设$a,b$为整数,$b>0$,则存在唯一的整数$q,r$,使得$a=bq+r$,其中$0 \leq r <b$.
\end{theorem}
\begin{proof}
	\buzhou{1}存在性:在数列$$\cdots , -2b ,-b , 0 ,b,2b,\cdots $$中,$a$必然在某两项之间,即存在$q$使$qb \leq a < (q+1)b$.此时取$r=a-bq$满足$0 \leq r <b$. \\
	\buzhou{2}唯一性:假设$a=bq_1+r_1=bq_2+r_2$且$q_1 \neq q_2,~r_1 \neq r_2$,则$b(q_1-q_2)=r_1-r_2$,于是$b \mid r_2-r_1$. \\
	由题目要求,$0 \leq r_1 < b,~0 \leq r_2 < b$,则$0 \leq |r_1-r_2| < b$.联系上式,必然有$r_1-r_2=0$,矛盾!故唯一性成立.
\end{proof}

从整除到带余除法,我们一直在寻找整数可除性的下限.有一类非常顽固的数,它们除以任何不为$1$及它本身的数均不能整除,称作\textbf{素数}(prime);相应地,不是素数的数就是合数.需要注意,$1$既不是素数也不是合数.

关于素数的“通项公式”,人们有过很多猜想.例如,形如$F_m=2^{2^m}+1~(m=1,2,\cdots )$的数被称为“费马数”.由于$F_0 \sim F_4$均为素数,费马曾认为所有的$F_m$都是素数,然而实际上已知的素数只有$F_0 \sim F_4$(人们猜测这之后的费马数全是合数,但是还没有证明).另一个猜想是“梅森素数”,即形如$M_n=2^n-1~(n=1,2,\cdots )$的素数.截至2018年12月,已知51个梅森素数,最大的是$2^{82589933}-1$.

\begin{example} % 初等数论p4
	(1)若$ax_0+by_0$是形如$ax+by$~(其中$x,y$为任意整数,$a,b$是两个不全为$0$的整数)的数中的最小正数,求证:$$(ax_0+by_0) \mid (ax+by)$$对所有$x,y$均成立. \\
	(2)已知$a,b$是任意整数,且$b \neq 0$.求证:存在两个整数$s,t$使得$$a=bs+t,~|t| \leq \frac{|b|}{2}$$
	成立.且当$b$为奇数时,这样的$s,t$唯一存在.并讨论$b$为偶数时存在个数的情况.
\end{example}
\begin{proof}
	(1)设$ax+by=q(ax_0+by_0)+r$,其中$0 \leq r < ax_0+by_0$.整理上式,可得$$r=a(x-qx_0)+b(y-qy_0)$$
	假设$r \neq 0$,则$r$也是形如$ax+by$的数,又因为$r < ax_0+by_0$,这与$ax_0+by_0$是最小的形如$ax+by$的正数矛盾.故$r=0$,即$(ax_0+by_0) \mid (ax+by)$. \\
	(2)\buzhou{1}存在性: \\
	$b > 0$时:由带余除法定理,存在$q,r$使$a=bq+r$与$0 \leq r <b$成立. \\
	若$r \leq \dfrac{b}{2}$,取$t=r,~s=q$即可;若$r < \dfrac{b}{2}$,则取$t=r-b,~s=q+1$,此时$-\dfrac{b}{2} \leq t < \dfrac{b}{2}$,即$|t| \leq \dfrac{|b|}{2}$. \\
	$b < 0$时,将上面的$b$替换为$-b$,同理可证. \\
	\buzhou{2}唯一性:\\
	$b$为奇数时,$s,t$唯一存在.假设$a=bs_1+t_1=bs_2+t_2$,则$b(s_1-s_2)=t_2-t_1$,可得$b \mid t_2-t_1$.由$|t_2-t_1| \leq |t_2|+|t_1| < |b|$(其中等号应在$t_1+t_2=b$处取到,然而$b$为奇数,故不存在满足该条件的整数$t_1,t_2$,故等号取不到),必有$t_2-t_1=0$,即$t_2=t_1$,矛盾!于是这样的$s,t$唯一存在. \\
	$b$为偶数时,$s,t$不唯一存在.例如,对于满足$a=3 \cdot \dfrac{b}{2}$的$a,b$,取$(s,t)=(1,\dfrac{b}{2}),(2,-\dfrac{b}{2})$均符合题意.
\end{proof}

\begin{example} % 初等数论p19
	(1)证明素数有无穷多个. \\
	(2)若$2^n+1$是素数(其中$n$为正整数),则$n$是$2$的方幂.
\end{example}
\begin{proof}
	(1)假设素数只有有限个,设为$p_1,p_2, \cdots ,p_m$.令$N=p_1p_2 \cdots p_n + 1$,取$N$的一个素因子$p$.\\
	若$p=p_i$,则由$p \mid p_1p_2\cdots p_n +1$有$p \mid 1$,与$p$是素数矛盾.故$p$为不同于任何一个$p_i$的素数,与假设矛盾.于是素数个数无穷. \\
	(2)首先,若$n$为奇数,则有$$2^n+1 = (3-1)^n+1 = 3^n + C_n^1 3^{n-1}(-1) + \cdots +C_n^{n-1} 3^1 (-1)^{n-1} + (-1)^n + 1$$
	于是$2^n+1$是$3$的倍数. \\
	其次,当$n$为偶数时,其必然可以表示为$k \cdot 2^m$的形式,其中$k$为奇数.于是有
	\begin{align*}
		2^n+1 = &\left( 2^{2^m} \right)^k + 1 = \ssb{ (2^{2^m}+1)-1 }^k + 1 \\
		&= (2^{2^m}+1)^k + C_k^1 (2^{2^m}+1)^{k-1}(-1) + \cdots + C_k^{k-1}(2^{2^m}+1)^1(-1)^{k-1} + (-1)^k + 1
	\end{align*}
	故$2^{2^m}+1$可以整除$2^n+1$. \\
	综上,$n$只能是$2$的方幂.
\end{proof}

\subsection{最大公约数与最小公倍数}

\begin{definition}{最大公约数}
	设$a_1,a_2, \cdots ,a_n$是$n~(n \geq 2)$个整数.满足下面两个条件的整数$d$称为它们的\textbf{最大公约数}(greatest common divisor),记作$\gcd (a_1,a_2, \cdots ,a_n)$或简记为$(a_1,a_2, \cdots ,a_n)$: \\
	(1)$d$是$a_1,a_2, \cdots ,a_n$的公约数,即$d \mid a_1,d \mid a_2, \cdots ,d \mid a_n$; \\
	(2)$d$是所有$a_1,a_2, \cdots ,a_n$的公约数中最大的. \\
	特别地,若$(a_1,a_2, \cdots ,a_n)=1$,则称$a_1,a_2, \cdots ,a_n$\textbf{互素};若$a_1,a_2, \cdots ,a_n$中任意两个是互素的,则称它们\textbf{两两互素}.互素与两两互素不是等价的.
\end{definition}
\begin{remark}
	注意到,任意一组整数必然有公约数;若它们不全为$0$,则公约数只有有限多个,此时的最大公约数存在且是唯一的.\\ 
	另外,若$d$是$a_1,a_2, \cdots ,a_n$的公约数,则$-d$也是$a_1,a_2, \cdots ,a_n$的公约数,那么最大公约数一定是正整数.
\end{remark}

\begin{theorem}{最大公约数的性质}{zvdagsytuu}
	(1)$(ab,ac)=a(b,c)$. \\ 
	(2)若$(a,b)=1$,则$(a,bc)=(a,c)$. \\
	(3)若$d \mid a,~d \mid b$则$d \mid (a,b)$. \\
	(4)若$a \mid c,~b \mid c$则$ab \mid c(a,b)$. \\
	(5)$a,b$的公约数与$(a,b)$的因数相同. \\
	(6)若$d$是$a,b$的任一公约数,则$$\ssb{\frac{a}{d},\frac{b}{d}} = \frac{(a,b)}{|d|}$$
	特别地,$$\ssb{\frac{a}{(a,b)},\frac{b}{(a,b)}}=1$$
	(7)$(a,b,c)=((a,b),c)$.
\end{theorem}
\begin{remark}
	这些性质用算术基本定理证明更方便,这里先不证明了.
\end{remark}

\begin{theorem}{辗转相除}{vjvrxliu}
	设$a,b,k$为整数,则$(a,b)=(a,b-ka)$.
\end{theorem}
\begin{proof}
	记$d$为$a,b$的任意公约数.由于$d \mid a,~d \mid b$,可得$d \mid b-ka$,于是$d$是$a,b-ka$的某个公约数. \\
	同理,任取$d$为$a,b-ka$的任意公约数,可得$d$是$a,b$的某个公约数.于是由所有$a,b$公约数构成的集合与所有$a,b-ka$的公约数构成的集合相等,其中的最大元素也相等.
\end{proof}

利用定理\ref{thm:vjvrxliu},我们可以很方便地求解两个整数的最大公约数.例如,$(114,514)=(114,514-4\times 114) = (114,58) = (114-2\times 58 ,58)=(-2,58)=2$\footnote{严格来讲,这里将$114$变为$-2$的操作不是辗转相除,但这样计算会简便一些.}.这样的求解算法称作\textbf{Euclid算法}(辗转相除法).其严格定义如下:

\begin{theorem}{Euclid算法}
	设$a,b$为整数,$b \neq 0$,按下述方式反复做带余除法,有限步之后停止(即余数为$0$),称做\textbf{Euclid算法}: \\
	用$b$除$a$:$a=bq_0+r_0,\quad 0<r<|b|$; \\
	用$r_0$除$b$:$b=r_0q_1+r_1,\quad 0<r_1<r_0$; \\
	用$r_1$除$r_0$:$r_0=r_1q_2+r_2,\quad 0<r_2<r_1$; \\
	$\cdots \cdots$ \\
	用$r_{n-1}$除$r_{n-2}$:$r_{n-2}=r_{n-1}q_n+r_n,\quad 0<r_n<r_{n-1}$; \\
	用$r_n$除$r_{n-1}$:$r_{n-1}=r_nq_{n+1}$. \\
	则$(a,b)=(r_0,b)=(r_1,r_2)= \cdots = (r_{n+1},r_n)=r_n$.
\end{theorem}
\begin{remark}
	实际上,由于余数$r_0,r_1,\cdots ,r_n,\cdots $为整数,且满足$r_0 > r_1 > \cdots > r_{n} > \cdots \geq 0$,必然在某一步时余数为$0$.(无穷递降法)
\end{remark}

\begin{theorem}{Bezout定理}
	若$a,b$为不全为$0$的整数,则存在整数$x,y$,使得$$ax+by=(a,b)$$成立.
\end{theorem}
\begin{remark}
	$x,y$并不唯一,因为可以通过$(a,b)=ax+by=a(x+kb)+b(y-ka)$的形式调整(其中$k$为任意整数).
\end{remark}
\begin{remark}
	特别地,若$(a,b)=1$,则存在整数$x,y$使$ax+by=k$.($k$为任意整数)
\end{remark}
\begin{proof} % 解法一:自己写的 解法二:https://oi-wiki.org/math/number-theory/bezouts/#证明
	\sw{一}由于$a$与$-a$在上式中等价,不妨设$a \geq b \geq 0$. 记$n=a+b$,对$n$进行归纳. \\
	\buzhou{1}当$n=1$时,即$a=1,b=0$,取$x=1$即符合题意. \\
	\buzhou{2}假设当$n=1$至$n=k-1$时命题均成立.当$n=k$时, \\
	若$b=0$,取$x=1$符合题意;若$b > 0$即$b \geq 1$,则$a-b \leq k-1$,由归纳假设知方程$$(a-2b)x+by=(a-2b,b)$$
	有解$(x,y)$.于是方程$ax+by=(a,b)$有解$(x,y-2x)$,命题成立. \\
	由第二数学归纳法知原命题成立. \\
	\sw{二}同解法一,若$a,b$中有一个为$0$,显然成立.在$a,b$均不为$0$时,不妨设$a \geq b >0$. \\
	记$a=a_1(a,b),~b=b_1(a,b)$,于是$(a_1,b_1)=1$.则原命题等价于存在$x,y$使得$a_1x+b_1y=1$. \\
	将求$(a_1,b_1)$的过程写成Euclid算法形式:
	\begin{align*}
		a_1 &= q_1b_1+r_1, \quad 0 \leq r_1 < b_1 \\
		b_1 &= q_2r_1+r_2, \quad 0 \leq r_2 < r_1 \\
		r_1 &= q_3r_2+r_3, \quad 0 \leq r_3 < r_2 \\
		& \cdots \cdots \\
		r_{n-3} &= q_{n-1}r_{n-2} + r_{n-1}, \quad 0 \leq r_{n-1} < r_{n-2} \\
		r_{n-2} &= q_nr_{n-1} + r_n, \quad 0 \leq r_n < r_{n-1} \\
		r_{n-1} &= q_{n+1}r_n
	\end{align*}
	由于$(a_1,b_1)=1$,可知其中$r_n=1$,即有$$r_{n-2} = q_nr_{n-1} + 1,\quad i.e. \quad 1=r_{n-2} - q_nr_{n-1}$$
	将倒数第三个式子$r_{n-1} = r_{n-3} - q_{n-1}r_{n-2}$代入上式,得$$1 = (1+q_nq_{n-1})r_{n-2} - q_nr_{n-3}$$
	同样地,将Euclid算法中的式子逐步代入以消去$r_{n-2},\cdots ,r_1$.最后等式右边一定会得到$a_1,b_1$的线性组合形式(因为右边没有常数项),即存在$x,y$使得$a_1x+b_1y=1$,原命题得证.
\end{proof}

类似地,可以定义最小公倍数:

\begin{definition}{最小公倍数}
	设$a_1,a_2, \cdots ,a_n$是$n~(n \geq 2)$个整数.满足下面两个条件的整数$D$称为它们的\textbf{最小公倍数}(least common multiple),记作$\lcm (a_1,a_2, \cdots ,a_n)$或简记为$[a_1,a_2, \cdots ,a_n]$: \\
	(1)$D$是$a_1,a_2, \cdots ,a_n$的公倍数,即$a_1 \mid D,a_2 \mid D, \cdots ,a_n \mid D$; \\
	(2)$D$是所有$a_1,a_2, \cdots ,a_n$的公倍数中最小的.
\end{definition}

\begin{theorem}{最小公倍数的性质}{zvxngsbwuu}
	(1)$[ab,ac]=a[b,c]$. \\
	(2)若$a \mid D,~b \mid D$则$[a,b] \mid D$. \\
	(3)$a,b$的公倍数与$[a,b]$的倍数相同. \\
	(4)$[a,b,c]=[[a,b],c]$. \\
	(5)$(a,b)[a,b]=|ab|$.
\end{theorem}
\begin{remark}
	第五条性质最常用,因为最大公约数的性质远比最小公倍数的好.
\end{remark}

\begin{example} % 初等数论p9,p14
	(1)证明两整数$a,b$互素的充要条件是:存在两个整数$s,t$满足$$as+bt=1$$ \\
	(2)应用上节例题(2)证明:$(a,b)=ax_0+by_0$,其中$ax_0+by_0$是形如$ax+by$~(其中$x,y$为任意整数)的整数里的最小正数.
\end{example}
\begin{solution}
	(1)必要性显然.下证充分性:由$(a,b) \mid a,~(a,b) \mid b$,可得$(a,b) \mid as+bt$,于是$(a,b)=1$,即$a,b$互素. \\
	(2)首先,由于$(a,b) \mid a,~(a,b) \mid b$,可得$(a,b) \mid ax_0+by_0$. \\
	其次,由Bezout定理可得,存在整数$x,y$使得$ax+by = (a,b)$,又因为$ax_0 + by_0 \mid ax+by$,故$ax_0 + by_0 \mid (a,b)$. \\
	综上可得,$(a,b)=ax_0+by_0$.
\end{solution}
\begin{remark}
	这两道例题都旨在说明整除与线性组合的强关联性.
\end{remark}

\subsection{Euclid引理与算术基本定理}

\begin{theorem}{Euclid引理}
	若$p$是一个素数,且$p \mid ab$,则$p \mid a$或$p \mid b$.
\end{theorem}
\begin{proof}
	假设$p \nmid a$且$p \nmid b$.由于$p$是素数,又$(a,p) \mid p$,可得$(a,p)=1$. \\
	由Bezout定理,存在整数$m,n$使得$am+pn=1$成立,于是$abm+bpn=b$.由于$p \mid ab$,由上式有$p \mid b$,矛盾!则原命题成立.
\end{proof}

\begin{theorem}{算术基本定理}
	每个不等于$1$的正整数均可分解为有限个素数的乘积.如果不计素因数在乘积中的次序,则该分解方式是唯一的.
\end{theorem}
\begin{proof}
	对于正整数$n>1$, \\
	\buzhou{1}存在性:第$1$步:取$n$的一个素因子$p_1$,于是有$n=n_1 \cdot p_1$; \\
	第$i$步:取$n_{i-1}$的一个素因子$p_i$,于是有$n_{i-1}=n_i \cdot p_i$. \\
	对于上述算法,由于每一步$n_i$都严格减少,因此一定在某一步停止,即此时$n_{i-1}=1$.最后可得分解为$n=p_1p_2 \cdots p_{i-1}$. \\
	\buzhou{2}唯一性:假设$n = p_1^{\alpha _1} p_2^{\alpha _2} \cdots p_k^{\alpha _k} = q_1^{\beta _1} q_2^{\beta _2} \cdots q_l^{\beta _l}$~(这两种表示方法不全相同). \\
	对于$p_1$,由于$p_1 \mid n$,可得$p_1 \mid q_1^{\beta _1} q_2^{\beta _2} \cdots q_l^{\beta _l}$.由Euclid引理,$q_1, \cdots ,q_l$中必有一个素数与$p_1$相同.同理可得$\{ p_1,\cdots ,p_k \} \subseteq \{ q_1,\cdots ,q_l \}$,由对称性可得$\{ q_1,\cdots ,q_l \} \subseteq \{ p_1,\cdots ,p_k \}$,于是$\{ p_1,\cdots ,p_k \} = \{ q_1,\cdots ,q_l \}$. \\
	不妨设$q_i=p_i~(i=1, \cdots ,k)$,即$p_1^{\alpha _1} p_2^{\alpha _2} \cdots p_k^{\alpha _k} = p_1^{\beta _1} q_2^{\beta _2} \cdots p_k^{\beta _k}$.对于任意给定的$i$,若$\alpha _i \neq \beta _i$,不妨设$\alpha _i < \beta _i$,则由于$$\frac{n}{p_i^{\alpha _i}} = p_1^{\alpha _1} \cdots p_{i-1}^{\alpha _{i-1}} p_{i+1}^{\alpha _{i+1}} \cdots p_{k}^{\alpha _{k}} = p_1^{\beta _1} \cdots p_{i-1}^{\beta _{i-1}} p_{i}^{\beta _{i} - \alpha _{i}} p_{i+1}^{\beta _{i+1}} \cdots p_{k}^{\beta _{k}}$$
	可知$p_i \mid p_1^{\alpha _1} \cdots p_{i-1}^{\alpha _{i-1}} p_{i+1}^{\alpha _{i+1}} \cdots p_{k}^{\alpha _{k}}$,这是不可能的. \\
	于是,对于任意$i~(i=1, \cdots ,k)$,$\alpha _i = \beta _i$,即这两种表示方法相同,与假设矛盾.故这样的表示方法是唯一的.
\end{proof}

由算术基本定理,不等于$\pm 1$的非零整数$n$可以唯一地表示为$$n = \varepsilon p_1^{\alpha _1} p_2^{\alpha _2} \cdots p_k^{\alpha _k}$$
其中$p_1,p_2, \cdots ,p_k$为互不相同的素数,$\alpha _1 ,\alpha _2,\cdots ,\alpha _k$为正整数,$\varepsilon =\pm 1$.这称为$n$的素因数标准分解.(当$\alpha _1 ,\alpha _2,\cdots ,\alpha _k$为非负整数时,称作$n$的素因数分解.为了方便计算,有些时候会选择取$\alpha _i$为非负整数以补齐位数)

从素因数分解的角度,我们得到了一种新的判断整除的方法,并由此得到了计算最大公约数、最小公倍数的方法:

\begin{proposition}{从素因数分解看整除}
	设$a=p_1^{\alpha _1} p_2^{\alpha _2} \cdots p_k^{\alpha _k},~b=p_1^{\beta _1} p_2^{\beta _2} \cdots p_k^{\beta _k}$.则
	$$b \mid a \quad \Longleftrightarrow \quad \forall i,~0 \leq \beta _i \leq \alpha _i$$
	因而有
	$$(a,b)=\sum_{i=1}^{k} p_i^{\min {\alpha _i,\beta _i}}, \quad \quad [a,b]=\sum_{i=1}^{k} p_i^{\max {\alpha _i,\beta _i}}$$
\end{proposition}

作为练习,读者不妨返回上一节,尝试利用素因数分解证明定理\ref{thm:zvdagsytuu}和定理\ref{thm:zvxngsbwuu}.

\subsection{不定方程}

先从最简单的二元一次不定方程开始研究:

\begin{theorem}{二元一次不定方程解的关系}{eryryicigrxi}
	设$a,b,c$为整数,满足$ax+by=c$~(其中$a,b$不全为$0$).若该不定方程有一组整数解$(x_0,y_0)$,则它的一切解都可以写成$$x=x_0-\frac{b}{(a,b)}t,\quad y=y_0+\frac{a}{(a,b)}t$$
	的形式,其中$t$为任意整数.
\end{theorem}
\begin{proof}
	首先,显然上述形式可以满足$ax+by=c$.对于$ax+by=c$的一组解$(x',y')$,有$$ax'+by' = c = ax_0+by_0,\quad \textit{即} \quad a(x'-x_0) + b(y'-y_0) = 0$$
	记$(a,b)=d~,a=a_1d,b=b_1d$,则$(a_1,b_1)=1$.代入上式,即得$$a_1(x'-x_0)=-b_1(y'-y_0)$$
	于是$a_1 \mid y'-y_0$,即存在整数$t$使得$y'-y_0=ta_1$,即$y'=y_0+a_1t$.将这样的$y'$代回原式可得$x'=x_0-b_1t$.于是原方程所有的解均可表示为上述形式.
\end{proof}

\begin{theorem}{二元一次不定方程解的存在性}
	设$a,b,c$为整数,满足$ax+by=c$~(其中$a,b$不全为$0$).该二元一次不定方程有整数解当且仅当$(a,b) \mid c$.
\end{theorem}
\begin{proof}
	\buzhou{1}必要性:设方程有一组解$(x_0,y_0)$,即$ax_0+by_0=c$,由$(a,b) \mid ax_0+by_0$,自然有$(a,b) \mid c$. \\
	\buzhou{2}充分性:由Bezout定理可知,存在整数$m,n$使$am+bn=(a,b)$.记$c=c_1(a,b)$,则$amc_1+bnc_1=c$,即$(mc_1,nc_1)$是原方程的一组整数解.
\end{proof}

具体来讲,如何找到不定方程的一组解,除了通过观察,更快速的是借助下一章的同余工具,这里简要示范(读者可以在看过下一章之后再回来做).

多元一次不定方程是二元情况的推广,在求解的时候一般作换元以消去尽可能多的变量,最后只需要解一些二元不定方程即可.

\begin{example}
	(1)解不定方程:$1145x+14y=1919810$. \\
	(2)将$\dfrac{17}{60}$表示为分母两两互素的三个既约分数之和.
\end{example}
\begin{solution}
	(1)两边同时模$14$,可得$$11x \equiv 4 \mod 14$$
	即$-3x \equiv 18 \mod 14$,即$x \equiv -6 \mod 14$.于是,$(8,136475)$是原方程的一组解.它的所有解可以表示为$$x=8-14t,\quad y=136475+1145t$$
	的形式,其中$t$为任意整数. \\
	(2)由于$60=3 \times 4 \times 5$,设$\dfrac{17}{60} = \dfrac{x}{3} + \dfrac{y}{4} + \dfrac{z}{5}$,即解不定方程$$20x+15y+12z=17$$
	记$t=4x+3y$,于是$5t+12z=17$.这两个不定方程的解分别为$$\begin{cases}
		x=t-3u \\ y=-t+4u
	\end{cases}, \quad \begin{cases}
		t=1-12v \\ z=1+5v
	\end{cases}$$
	其中$u,v$是任意整数.将第二组中关于$t$的解带入第一组,即得$$\begin{cases}
		x=1-12v-3u \\ y=-1+12v+4u \\ z=1+5v
	\end{cases}$$
\end{solution}

还有一些高次不定方程,需要通过合理的分析与处理进行解决.最有名的二次不定方程莫过于“勾股数”方程:$x^2+y^2=z^2$.

\begin{example}
	证明: \\
	(1)设$n$为正整数,有$(a^n,b^n)=(a,b)^n$. \\
	(2)设$a,b$为互素的正整数,$ab=c^n$~($c$为整数),则$a,b$都是正整数的$n$次方幂. \\
	(3)设正整数$x,y,z$满足$x^2+y^2=z^2$.则存在正整数$m,n$,使得$$x=m^2-n^2,~y=2mn,~z=m^2+n^2 ~~\cor ~~ x=2mn,~y=m^2-n^2,~z=m^2+n^2$$
\end{example}
\begin{proof}
	(1)记$(a,b)=d,~a=a_1d,~b=b_1d$,则$(a_1,b_1)=1$.于是$$(a^n,b^n)=(d^na_1^n,d_nb_1^n)=d^n(a_1^n,b_1^n)=d^n$$
	(2)由(1)可得$$(a,c)^n = (a^n,c^n) = (a^n,ab) = a(a^{n-1},b) = a$$
	同理可得$b=(b,c)^n$.故$a,b$都是正整数的$n$次方幂. \\
	(3)由于$x^2+y^2=z^2$及“对于正整数$n$,$n^2$除$4$余$0$或$1$”及$(x,y,z)=1$,可知$(x,y,z)$的奇偶性分别可能为$(\textit{奇},\textit{偶},\textit{奇})\cor (\textit{奇},\textit{偶},\textit{奇})$.不妨设$y$为偶数. \\
	容易发现满足$m^2=\dfrac{z+x}{2},~n^2=\dfrac{z-x}{2}$的有理数$m,n$符合题意.下证这样的$m,n$为整数. \\
	注意到$\dfrac{z+x}{2} \cdot \dfrac{z-x}{2} = \ssb{\dfrac{y}{2}}^2$,由(2)中结论,只需证明$$\ssb{\frac{z+x}{2}, \frac{z-x}{2}}=1$$
	化简之,即证明$(z+x,z-x)=2$,即$(x,z)=1$.这是显然的,不妨设$(x,y,z)=1$(因为$x,y,z$同时扩大对结论无影响),假设$(x,z)>1$,即存在素数$p$使得$p \mid x$且$p \mid z$,由于$y^2=z^2-x^2$,可得$p \mid y^2$,即有$p \mid y$.这与$(x,y,z)=1$矛盾!
\end{proof}

\section{同余}

\subsection{同余运算}

\begin{definition}{同余运算}
	设$a,b,m~(m \neq 0)$是整数,若$m \mid a-b$,则称$a,b$\textbf{模$m$同余},记作$$a \equiv b \mod m$$
	若$m \nmid a-b$,则称$a,b$模$m$不同余,记作$$a \not\equiv b \mod m$$
\end{definition}
\begin{remark}
	在具体题目情景中,若上下文的模都是一样的,可以省略“$\operatorname{mod} m$”符号.
\end{remark}

\begin{theorem}{同余运算的基本性质}
	对任意整数$x,y,z,m~(m \neq 0)$,在模$m$意义下,有 \\
	(1)自反性:$x \equiv x$. \\
	(2)传递性:若$x \equiv y$且$y \equiv z$,有$x \equiv z$. \\
	(3)反对称性:若$x \equiv y$,则$y \equiv x$.
\end{theorem}
\begin{remark}
	实际上,满足自反性、传递性与对称性的二元关系被称作“等价关系”,例如两个对象的相等关系.从这一点出发,我们也可以看出同余运算的优点:将整除变为“等式”,“等式”左右可以像真正的等式一样运算(基础运算即定理\ref{thm:tsyuyysr}中的(1);除法会在后面讲到).
\end{remark}

\begin{theorem}{同余运算的性质}{tsyuyysr}
	(1)关于加法与乘法的性质:若$a \equiv b \mod m,~c \equiv d \mod m$,则$$a+c \equiv b+d \mod m, \qquad ac \equiv bd \mod m$$
	一般地,若$A_{\alpha _1 \cdots \alpha _k} \equiv B_{\alpha _1 \cdots \alpha _k} \mod m$且$x_i \equiv y_i \mod m~(i=1,2,\cdots ,k)$,则有$$\sum_{\alpha _1, \cdots ,\alpha _k} A_{\alpha _1 \cdots \alpha _k} x_1^{\alpha _1} \cdots x_k^{\alpha _k} \equiv \sum_{\alpha _1, \cdots ,\alpha _k} B_{\alpha _1 \cdots \alpha _k} y_1^{\alpha _1} \cdots y_k^{\alpha _k} \mod m$$
	(2)若$a \equiv b \mod m$且$a \equiv b \mod n$,则$$a \equiv b \mod [m,n]$$
	(3)若$ac \equiv bc \mod m$,其中$c \neq 0$,则$$a \equiv b \mod \frac{m}{(c,m)}$$
\end{theorem}


\subsection{剩余系}

同余的本质,实际上是将正整数集$\mathbb{Z}$分为若干集合,每个集合模$m$的余数循环.类似于这种思路,可以对所有模$m$同余的数建立一个等价类:

\begin{theorem}{剩余类}{ugyulw}
	对于任意给定的正整数$m$,全部整数可以划分成$m$个集合,记作$K_0,K_1, \cdots ,K_{m-1}$,其中$K_r~(r=0,1,\cdots ,m-1)$是由一切形如$qm+r~(q=0,\pm 1,\pm 2,\cdots )$组成的.这些集合具有下列性质: \\
	(1)每一整数必包含在且仅在上述的一个集合里面. \\
	(2)两个整数同在一个集合当且仅当这两个整数模$m$同余.
\end{theorem}
\begin{proof}
	(1)由带余除法定理,任取整数$a$,记$a=a_1m + r~(0 \leq r < m)$,于是$a$就在$K_r$中,且这里的$r$是由$a$唯一确定的. \\
	(2)\buzhou{1} 充分性:若$a \equiv b \mod m$,则由定义知$m \mid a-b$.记$a-b = km$,$b=qm + r$,则$$a = (k+q)m +r$$
	故$a,b$同在$K_r$中. \\
	\buzhou{2} 必要性:若$a,b$均在$K_r$中,记$$a=q_1m+r,~b=q_2m+r$$
	则$a-b = (q_1-q_2)m$,故$a \equiv b \mod m$.
\end{proof}

\begin{definition}{剩余类与完全剩余系}
	定理\ref{thm:ugyulw}中的$K_0,K_1, \cdots ,K_{m-1}$称作模$m$的\textbf{剩余类};若$a_0,a_1,\cdots ,a_{m-1}$是$m$个整数,并且其中任意两数都不在同一个剩余类里,则$a_0,a_1,\cdots ,a_{m-1}$称作模$m$的一个\textbf{完全剩余系}.
\end{definition}
\begin{remark}
	用同余的语言,可以得到一个等价定义:$m$个整数构成模$m$的一个完全剩余系当且仅当它们两两模$m$不同余.
\end{remark}

例如,对于正整数$m$,最小的非负完全剩余系为$\{ 0,1,2,\cdots ,m-1 \}$.

\begin{theorem}{完全剩余系的性质}{wjqrugyuxi}
	设正整数$m$,整数$a$满足$(a,m)=1$,$b$是任意整数.若$S$是一个模$m$的完全剩余系,则$T=aS+b$也是模$m$的完全剩余系.
\end{theorem}
\begin{proof}
	对于$T$,任取其中元素$x,y$,记$x=ap+b,~y=aq+b$. \\
	假设$x \equiv y \mod m$,则$ap \equiv ap \mod m$.应用定理\ref{thm:tsyuyysr}的第三条可知$p \equiv q \mod m$,这与$S$是完全剩余系矛盾.故$T$中元素两两模$m$不同余,则$T$是模$m$的完全剩余系.
\end{proof}

\begin{theorem}
	设正整数$m$,整数$a$满足$(a,m)=1$,$b$是任意整数.存在整数$x$使得$ax \equiv b \mod m$且所有满足该条件的$x$在模$m$的同一个剩余类中.
\end{theorem}
\begin{proof}
	由于$\{ 1,2,\cdots ,m \}$是模$m$的一个完全剩余系,由定理\ref{thm:wjqrugyuxi}可知$\{ a,2a,\cdots ,ma \}$也是模$m$的一个完全剩余系.于是对任意整数$b$,总存在$x \in \{ 1,2,\cdots ,m \}$使得$ax \equiv b \mod m$. \\
	若有$x,y$均满足条件,即$ax \equiv b \mod m,~ay \equiv b \mod m$,则$a(x-y) \equiv 0 \mod m$,故$x \equiv y \mod m$,即所有满足条件的$x$在模$m$的同一个剩余类中.
\end{proof}

在上述定理中,如果取$b=1$,则意味着在$(a,m)=1$时,总存在$x$使得$ax \equiv 1 \mod m$.称这样的$x$为$a$在\textbf{模$m$意义下的乘法逆元},记$x \equiv a^{-1} \mod m$.因为所有这样的$a^{-1}$组成了一个模$m$的剩余类,故可以将乘法逆元(不严谨地)当做除法来考虑.

\subsection{欧拉定理,费马小定理与阶}

\begin{definition}{欧拉函数}
	定义\textbf{欧拉函数}$\varphi :\mathbb{Z} \to \mathbb{Z}$满足,$\varphi (a)$的值等于在$0,1,\cdots ,a-1$中与$a$互素的数的个数.
\end{definition}

\begin{definition}{缩剩余系}
	若$\varphi (m)$个整数$a_1,\cdots ,a_{\varphi (m)}$包含于模$m$的一个完全剩余系,且其中任一整数均与$m$互素,则称它们构成模$m$的一个\textbf{缩剩余系}(简化剩余系).
\end{definition}

类似于完全剩余系,有

\begin{theorem}{缩剩余系的性质}
	设正整数$m$,整数$a$满足$(a,m)=1$.若$S$是一个模$m$的缩剩余系,则$T=aS$也是模$m$的缩剩余系.
\end{theorem}
\begin{proof}
	由于$|T|=|S|=\varphi (m)$,只需证明$T$中的$\varphi (m)$个元素模$m$互不同余.实际上,若$x_1,x_2 \in S$满足$ax_1 \equiv ax_2 \mod m$,由于$(a,m)=1$,则有$x_1 \equiv x_2 \mod m$,矛盾.故原定理得证.
\end{proof}

欧拉函数是一个积性函数,即对于互素的正整数$m_1,m_2$,有$\varphi (m_1m_2) = \varphi (m_1) \cdot \varphi (m_2)$.为了证明这个定理,若设$S_1,S_2$分别为模$m_1,m_2$的缩系,注意到$\varphi (m_1) \cdot \varphi (m_2)$即为$m_2S_1+m_1S_2$的元素个数.于是只要证明:

\begin{theorem}
	若整数$m_1,m_2$满足$(m_1,m_2)=1$,设$S_1,S_2$分别为模$m_1,m_2$的缩系,则$m_2S_1+m_1S_2$是模$m_1m_2$的缩系.
\end{theorem}
\begin{proof}
	\buzhou{1} 先证明一个更一般的情况:若整数$m_1,m_2$满足$(m_1,m_2)=1$,设$S_1,S_2$分别为模$m_1,m_2$的完系,则$m_2S_1+m_1S_2$是模$m_1m_2$的完系. \\
	首先注意到$|m_2S_1+m_1S_2|=|S_1| \cdot |S_2|=m_1m_2$,于是只需证明$m_2S_1+m_1S_2$中的$m_1m_2$个整数模$m_1m_2$互不同余. \\
	假设存在$x_1,y_1 \in S_1$与$x_2,y_2 \in S_2$满足$$m_2x_1 + m_1x_2 \equiv m_2y_1 + m_1y_2 \mod m_1m_2$$
	于是$$m_1(x_2-y_2) \equiv m_2(x_1-y_1) \mod m_1m_2$$
	则有$$x_1 \equiv y_1 \mod m_1,\quad x_2 \equiv y_2 \mod m_2$$
	矛盾.故$m_2S_1+m_1S_2$中的$m_1m_2$个整数模$m_1m_2$互不同余. \\
	\buzhou{2} 在上述一般情况中,若任取$x_1 \in S_1$与$x_2 \in S_2$均有$(x_1,m_1)=(x_2,m_2)=1$,即为原命题情景. \\
	一方面,由于$(m_2x_1+m_1x_2,m_1)=1$与$(m_2x_1+m_1x_2,m_2)=1$,显然有$(m_2x_1+m_1x_2,m_1m_2)=1$; \\
	另一方面,若$(m_2x_1+m_1x_2,m_1m_2)=1$,则分别有$(m_2x_1+m_1x_2,m_1)=1$与$(m_2x_1+m_1x_2,m_2)=1$,于是$(x_1,m_1)=(x_2,m_2)=1$. \\
	综上,$m_2S_1+m_1S_2$是模$m_1m_2$的完系.
\end{proof}

由此我们可以计算$\varphi (n)$.

\begin{theorem}
	设$n=p_1^{\alpha _1} \cdots p_k^{\alpha _k}$,则$$\varphi (n) = n \ssb{1-\frac{1}{p_1}} \cdots \ssb{1-\frac{1}{p_k}}$$
\end{theorem}
\begin{proof}
	由欧拉函数是积性函数,有$$\varphi (n) = \varphi (p_1^{\alpha _1}) \cdots \varphi (p_k^{\alpha _k})$$
	再来看$\varphi (p^k)$,其中$p$是素数而$k$是任意正整数.由定义,$\varphi (p^k)$表示在$0,1,p^k-1$中与$p^k$互素的数的个数,即从$0,1,p^k-1$中去除能被$p$整除的数的个数,即$$\varphi (p^k) = p^k - \lfloor \frac{p^k}{p} \rfloor = p^k - p^{k-1}$$
	代入上式,即得$$\varphi (n) = p_1^{\alpha _1}\ssb{1-\frac{1}{p_1}} \cdots p_k^{\alpha _k}\ssb{1-\frac{1}{p_k}} = n \ssb{1-\frac{1}{p_1}} \cdots \ssb{1-\frac{1}{p_k}}$$
\end{proof}

以上是计算$\varphi (n)$的一条路.或者,在不知道它是积性函数的情况下,通过数集合的元素个数可以得到同样的结果.

设$n=p_1^{\alpha _1} \cdots p_k^{\alpha _k}$,那么一个数$p$与$n$互素,当且仅当$p_i \nmid p~(i=1,\cdots ,k)$.记$P_i=\{ 1\leq p \leq n: p_i \mid p\}$,那么$$\varphi (n) = n - | \bigcup_{i=1}^{k} P_i | = n- \ssb{\sum_{i=1}^{k} |P_i| - \sum_{1\leq i_1 < i_2 \leq k} |P_{i_1} \cap P_{i_2}| + \cdots + (-1)^{k-1} |P_1 \cap \cdots \cap P_k|  }$$
其中,由于$p_i$两两互素,可得$$|P_{i_1} \cap \cdots \cap P_{i_m}| = \frac{n}{p_{i_1} \cdots p_{i_m}}$$
于是$$\varphi (n) = n - \sum_{i=1}^{k} \frac{n}{p_i}  + \sum_{1\leq i_1 < i_2 \leq k} \frac{n}{p_{i_1}p_{i_2}} - \cdots + (-1)^{k} \frac{n}{p_1 \cdots p_k}$$
稍作整理,即得$$\varphi (n) = n \ssb{1-\frac{1}{p_1}} \cdots \ssb{1-\frac{1}{p_k}}$$

\begin{theorem}{Euler定理}
	设$m$是大于$1$的整数,$(a,m)=1$,则$$a^{\varphi (m)} \equiv 1 \mod m$$
\end{theorem}
\begin{proof}
	
\end{proof}

\subsection{同余式与中国剩余定理}

类似于上文所述,如果将同余运算考虑为相等符号(即构造若干“等价类”——剩余类进而将剩余类作为一个数的方法),我们可以拓广多项式的根相关的内容.

\begin{definition}{同余式}
	设$f(x)$表示多项式$a_nx^n + \cdots + a_1x + a_0$,其中$a_i \in \mathbb{Z}~(i=0,1,\cdots ,n)$.对于正整数$m$,则$$f(x) \equiv 0 \mod m$$
	称作模$m$的\textbf{同余式}.若$a_n \not\equiv 0 \mod m$,则$n$称作该同余式的\textbf{次数}.
\end{definition}

若$f(a) \equiv 0 \mod m$,则剩余类$K_a$中任意整数$x$都满足$f(x) \equiv 0$,于是可以定义:

\begin{definition}{同余式的解}
	若$a$是使得$f(a) \equiv 0 \mod m$成立的一个整数,则$x \equiv a \mod m$称作该同余式的一个\textbf{解}.
\end{definition}
