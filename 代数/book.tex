\documentclass{plainbook}

\usepackage{amsfonts}
\usepackage{amsmath}
\usepackage{amssymb}
\usepackage{hyperref}
\usepackage{svg}
\usepackage{booktabs}
\usepackage{framed}
\usepackage{tikz-cd}
\usepackage{epigraph}


% font; do not use in overleaf

\usepackage[UTF8,scheme=plain,fontset=none]{ctex}
    \setCJKmainfont[BoldFont={Source Han Serif SC-SemiBold},ItalicFont={FZKai-Z03}]{FZShuSong-Z01}
    \setCJKsansfont[BoldFont={Source Han Serif SC-SemiBold}]{FZKai-Z03}
    \setCJKmonofont[BoldFont={Source Han Serif SC-SemiBold}]{FZFangSong-Z02}
    \setCJKfamilyfont{zhsong}{FZShuSong-Z01}
    \setCJKfamilyfont{zhhei}{Source Han Serif SC-SemiBold}
    \setCJKfamilyfont{zhkai}[BoldFont={Source Han Serif SC-SemiBold}]{FZKai-Z03}
    \setCJKfamilyfont{zhfs}[BoldFont={Source Han Serif SC-SemiBold}]{FZFangSong-Z02}


\title{临时笔记}

% Set the authors of the book (multiple authors separated by \and).
\author{bilibili:晨沐公Kasumi \quad github:MATHhahetaDEATH}

% Set the date to the current date.
\date{\today}

% customised commands
\definecolor{winered}{rgb}{0.5,0,0}
\newcommand{\exref}[1]{\ref}
\newcommand{\xl}[1]{\overrightarrow{#1}}
\newcommand{\ssb}[1]{\left( #1 \right)}
\newcommand{\flr}[1]{\lfloor #1 \rfloor}
\newcommand{\ang}[1]{\langle #1 \rangle}
\newcommand{\R}{\mathbb{R}}
\newcommand{\C}{\mathbb{C}}
\newcommand{\Z}{\mathbb{Z}}
\newcommand{\F}{\mathbb{F}}
\newcommand{\id}{\mathrm{id}}
\newcommand{\lmap}{\mathcal{L}}
\newcommand{\mmatrix}{\mathcal{M}}
\newcommand{\sw}[1]{\boxed{\text{解法 #1}} \ }
\newcommand{\buzhou}[1]{$#1^{\circ} \ $}
\usepackage{ulem}
	\newcommand{\tk}{\uline{\hspace{4em}}}
\newcommand{\pspace}{\vspace{0.5em}}
\usepackage{amsmath,amsfonts}
	\DeclareMathOperator{\spn}{span}
	\DeclareMathOperator{\card}{card}
	\DeclareMathOperator{\ic}{i}
	\DeclareMathOperator{\arccot}{arccot}
	\DeclareMathOperator{\setjianfa}{\textbackslash}
	\DeclareMathOperator{\nul}{null}
	\DeclareMathOperator{\rank}{rank}
	\DeclareMathOperator{\rge}{range}
	\DeclareMathOperator{\sgn}{sgn}
	\DeclareMathOperator{\T}{T}
	\DeclareMathOperator{\im}{im}
	\DeclareMathOperator{\ord}{ord}
	\DeclareMathOperator{\sym}{Sym}

\makeatletter
\newcommand*\bigcdot{\mathpalette\bigcdot@{.5}}
\newcommand*\bigcdot@[2]{\mathbin{\vcenter{\hbox{\scalebox{#2}{$\m@th#1\bullet$}}}}}
\makeatother

% Begin the document.
\begin{document}

% Front matter section.
\frontmatter

% Include the title page, which is located in the FrontMatter subfolder.
\include{./FrontMatter/titlepage}

% Create the book's title page.
\maketitle\pagebreak

% Include the dedication page from the FrontMatter subfolder.
% \include{./FrontMatter/dedication}

% Include the epigraph page from the FrontMatter subfolder.
% This code snippet creates a quote block attributed to an author.

% Vertically space the content evenly, pushing the quote to the center of the page.
\vspace*{\fill}

% Set the font size to \Large (large) and the text style to italics.
\Large\textit{Young man, in mathematics you don’t understand things. You just get used to them. }

% Add some vertical space after the quote.
\bigskip

% The author's name is right-aligned and set in sans-serif small caps.
\begin{flushright}
    \sffamily\scshape John von Neumann
\end{flushright}

% Set the font back to the default (normal font size and style).
\normalfont\normalsize

% Vertically space the content evenly again, pushing any remaining space to the bottom of the page.
\vspace*{\fill}


% Include the foreword page from the FrontMatter subfolder.
\chapter*{前言}

在数学中, 对于每一个数学对象(例如极限), 我们会例行公事般地考虑它的一些常见的性质. 比如说, 这个对象最基本的例子是什么, 这种对象是否存在, 如果存在的话它是否具有唯一性, 它的子对象和商对象(如果有的话)都具有什么性质(比如说遗传了原来的对象的什么性质), 这个对象的可计算性以及在特定映射下的行为等等.




这份讲义cover的内容: 

- “抽象废话”范畴论, 当然只有一点点

- 作为工具和灵感来源的矩阵理论

- 几乎同构于矩阵理论的线性映射理论, 但我们会研究一些无限维情况

- 经典的群结构$\Z / n\Z$, 以及一般的群论

- 经典的环结构$\F [x]$, 以及初步的环论

- 域理论

- 初步的模论

换句话说, 本讲义大致覆盖高等代数与基础抽象代数的内容. 





抽象的代数需要直觉, 例如“自由度”“置换”等想法. 用具体详实的例子引出抽象理论. 


很多想法都是读者所熟悉的, 例如数论中的中国剩余定理, 多项式理论中的因式分解等. 因此这份讲义将不会侧重解释这些初等理论, 而是试图将其推广. 




% Include the preface page from the FrontMatter subfolder.
% \include{./FrontMatter/preface}

% Include the acknowledgement page from the FrontMatter subfolder.
% \chapter*{致谢}



\undersign

% Table of contents page.
\tableofcontents

% Main matter section.
\mainmatter

% 先写线性代数部分吧

\part{预备知识}

% 集合论回顾

\chapter{集合论入门, 映射与二元关系}

\section{公理化的集合论}

在高中我们已经学过朴素的集合论. 但是, 什么样的数学对象才是一个集合? 描述同一群对象的集合是唯一的吗? 为什么集合是无序的, 不重复的? 这些问题都需要通过引入公理体系来解决. 

本小节不会细致深入地讲解集合论的公理化体系, 因为这样会严重脱离《数学分析》的主旨. 

\subsection{集合的基本性质}

先来解决不同集合的等价问题. 

\begin{axiom}{外延公理}
	两个集合$A$和$B$相等当且仅当它们的元素相同.
\end{axiom}

容易验证, 集合的相等是一个等价关系(后面会提到), 也即它满足: 
\begin{enumerate}
	\item 自反性: 对于任一集合$A$都有$A=A$.
	\item 对称性: 若$A=B$, 则$B=A$.
	\item 传递性: 若$A=B$且$B=C$, 则$A=C$.
\end{enumerate}

外延公理告诉我们, 描述同一群对象的任意集合都是相等的. 因此, 从等价类的角度来看, 它的确是唯一的. 

接着解决集合的无序性、不重复性问题. 

\begin{axiom}{配对公理}
	对于任意集合$X, Y$, 存在一个集合$Z$使得$X$和$Y$是它仅有的元素. 特别地, 若$X=Y$, 则将$Z$视作只有唯一元素. 
\end{axiom}

由配对公理, 存在集合$\{ X, Y \}$和$\{ Y, X \}$, 而由外延公理这两个集合是相等的, 于是集合是无序的. 另一方面, 容易说明集合$\{ X, X \}$就等于$\{ X \}$, 于是集合是不重复的. 

\subsection{集合的运算}

到目前为止, 我们说明了集合的一些基本性质. 为了从一堆双元素集中得到更大的集合, 需要引入并运算. 

\begin{axiom}{并集公理}
	对于一个集合族$M$(即元素都是集合的集合), 存在另一个集合$\bigcup M$, 其元素恰包含所有属于$M$的集合的元素. 这样的集合称作$M$的\textit{并}(union). 
\end{axiom}

特别地, 若$M=\{ A, B \}$, 则$\bigcup M$可以记作$A \cup B$. 

\begin{axiom}{分离公理}
	任意集合$A$和性质$P$都对应另一个集合$B$, 其元素恰包含那些在集合$A$中而具有性质$P$的. 
\end{axiom}

这实际上是在说, $B=\{ x \in A :  P(x) \}$也是一个集合. 

结合并集公理, 马上可以定义集合族$M$的\textit{交}(intersection)为: $$\bigcap M : = \{ x \in \bigcup M :  \forall X, X \in M \Rightarrow x \in X \}.$$
特别地, 若$M= \{ A, B \}$, 则$\bigcap M$记作$A \cap B$. 

顺便还能定义集合的\textit{差}(difference)和\textit{补}(complement): $$A - B : = \{ x \in A :  x \notin B \}.$$
如果$A$是$M$的一个子集, 则定义: $$A^c : = M - A.$$

另外, 分离公理也表明, 对任意集合$X$都存在一个不包含任何元素的子集$\varnothing _X$. 由外延公理可知对任意集合$X, Y$都有$\varnothing _X = \varnothing _Y$. 我们称该集合为\textit{空集}(empty set), 记为$\varnothing$. 

由公理体系定义的集合运算, 自然具有我们在朴素集合论中学过的那些性质. 

\begin{proposition}{集合运算的运算律}
	设集合$A, B, C$, 集合族$\{ B_{\alpha} :  \alpha \in I \}$(这里$I$是指标集). 
	\begin{itemize}
		\item 交、并满足交换律, 即$$A \cap B = B \cap A,  \qquad A \cup B = B \cup A.$$
		\item 交、并满足结合律, 即
	$$A \cap B \cap C = (A \cap B) \cap C = A \cap (B \cap C), $$
	$$A \cup B \cup C = (A \cup B) \cup C = A \cup (B \cup C).$$
		\item 交对并、并对交满足分配律, 即
	$$A \cap \ssb{\bigcup_{\alpha \in I} B_\alpha} = \bigcup_{\alpha \in I} \ssb{A \cap B_{\alpha}}, $$
	$$A \cup \ssb{\bigcap_{\alpha \in I} B_\alpha} = \bigcap_{\alpha \in I} \ssb{A \cup B_{\alpha}}.$$
	\end{itemize}
\end{proposition}

就像中学数学所阐释的那样, 补和交、并之间有一种特殊的运算律: 

\begin{theorem}{de Morgan定律}
	设集合族$\{ E_{\alpha} :  \alpha \in I \}$, 其中$I$是指标集.则$$\ssb{\bigcup_{\alpha \in I}E_{\alpha} }^c = \bigcap_{\alpha \in I} E_{\alpha}^c, \qquad \ssb{\bigcap_{\alpha \in I}E_{\alpha} }^c = \bigcup_{\alpha \in I} E_{\alpha}^c.$$
\end{theorem}
\begin{proof}
	任取$x \in \ssb{\bigcup_{\alpha \in I}E_{\alpha} }^c$, 由定义得$x \notin \bigcup_{\alpha \in I}E_{\alpha}$, 所以对任意$\alpha \in I$都有$x \notin E_{\alpha}$, 即对任意$\alpha \in I$都有$x \in E_{\alpha}^c$, 从而可得$\ssb{\bigcup_{\alpha \in I}E_{\alpha} }^c \subseteq \bigcap_{\alpha \in I} E_{\alpha}^c$.
	
	同理可证$\ssb{\bigcup_{\alpha \in I}E_{\alpha} }^c \supseteq \bigcap_{\alpha \in I} E_{\alpha}^c$, 所以$$\ssb{\bigcup_{\alpha \in I}E_{\alpha} }^c = \bigcap_{\alpha \in I} E_{\alpha}^c.$$
	
	在上式左右同取补集, 立得第二个等式.
\end{proof}

最后一种构造更大集合的方式, 就是枚举一个集合的所有子集.

\begin{axiom}{幂集公理}
	对任意集合$X$, 总存在它的\textit{幂集}(power set)$\mathcal{P}(X)$, 其元素恰为$X$的所有子集.
\end{axiom}

作为应用, 幂集公理允许我们构造两个集合的Cartesian积(后面会讲到). 

前五个公理限制了构造新集合的方式, 公理化体系下的集合论已经初步成型.接下来要介绍的三条公理, 主要都是修修补补. 

\subsection{无限集}

我们知道, 自然数集$\mathbb{N}$理应当是无限的, 然而利用前五条公理还无法说明这样的无限集存在. 我们可以考虑利用递推的形式定义无限大的集合. 更确切地说, 由于现在只知道空集的存在, 应该选用空集的迭代来构造无限集合. 

为了让下面的公理叙述更简单, 首先引入集合的后继这一概念. 定义集合$X$的\textit{后继}(successor)为: $$X^{+} : = X \cup \{ X \}, $$
也就是说, 将$X$本身放入到$X$中. (实际上, 这里的$X$不应当属于其本身, 因此上述并是不交并)

\begin{axiom}{无穷公理}
	存在包含空集和自身任何一个元素的后继的集合. 这样的集合称作是\textit{归纳的}(inductive). 
\end{axiom}

联系公理一至四, von Neumann提出了一种构造自然数集的方法, 通过定义自然数集为所有归纳集的交集, 即最小的归纳集. 

要验证该交集为最小的归纳集并不难. 首先注意到, 任何归纳集都应包含以下元素: $$\varnothing , \quad \varnothing ^{+}=\varnothing \cup \{ \varnothing \}=\{ \varnothing \} , \quad (\varnothing ^{+})^{+} = \{ \varnothing \} \cup \{ \{ \varnothing \} \} = \{ \varnothing ,  \{ \varnothing \}\} , \quad \cdots .$$
把这些\footnote{这里的写法不太严谨, 因为在用该定义证明归纳原理之前并不十分清楚这些元素具体是什么样子. 严格地来说, “这些”指代$\varnothing$导出的一切后继. }元素组成的集合记作$N_0$. 由交的定义可知$$\mathbb{N} \subseteq N_0.$$
另一方面, 由于$\varnothing \in \mathbb{N}$, 所以$N_0 \subseteq \mathbb{N}$. 从而$\mathbb{N} = N_0$. 这也同时说明$\mathbb{N}$是最小的归纳集.

\begin{remark}
	下面做一点关于无穷集合构造的补充: 将一些特定集合(所谓序数)间的序关系理解为$\in$, 那么按照上述方式定义$X^+$, 我们就有$X \in X^+$, 并且这样的构造是使得$X^+$比$X$大的最小序数. 另外, $N_0$就是可数无穷序数$\omega$. 遗憾的是, 囿于篇幅, 我们无法介绍序数相关的知识. 
\end{remark}

将$\mathbb{N}$中$\varnothing$的$n$次后继这个特征提取出来, 可知$\mathbb{N}$就是一般意义上认为的自然数集. 实际上, 我们会在习题中验证, 这里定义的$\mathbb{N}$满足Peano公理. 

\begin{axiom}{替换公理}
	令$\mathcal{F}(x, y)$是如下命题: 对于$X$中的任意元素$x_0$, 存在唯一的$y_0$使得$\mathcal{F}(x_0, y_0)$成立. 那么满足以下条件的$y$构成一个集合: 存在$x \in X$使得$\mathcal{F}(x, y)$成立.
\end{axiom}

或者, 用映射的语言来描述, 替换公理就是在说: $f$是定义在集合$X$上的一个映射, 那么$f$的值域也是一个集合.

\subsection{Russell悖论}

在构造无限集的过程中, 可能会遇到如下问题: 设集合$A$满足$$A = \{ x: x \notin x \}$$
	那么$A \in A$是否成立? 如果成立, 那么由$A$的定义可知$A \notin A$; 如果不成立, 那么$A$就满足$x \notin x$, 从而$A \in A$. 这就是著名的Russell悖论. 现在我们尝试用构造新公理的方法修补这个问题.

\begin{axiom}{正则公理}
	任何非空集合$X$都存在一个元素$x$, 使得$x \cap X = \varnothing$.
\end{axiom}

结合配对公理, 可以证明$X \in X$这种情况是不存在的. 否则, 当$X$不是空集时, 考虑集合$\{ X\}$, 其中存在一个元素$x$, 此时只能是$X$, 使得$X \cap \{ X\}=\varnothing$, 然而$X \in X$告诉我们$X \cap \{ X\} \ni X$, 出现矛盾. 当$X$是空集时, $X$内存在一个元素本就与其定义矛盾.

然而, 使用正则公理只是人为禁用掉了Russell悖论出现的条件, 代价是减少集合论的可用范围(实际上禁掉这个条件没有特别大的影响). Russell悖论不可能被最终解决.

\subsection{选择公理}

最后一条公理是选择公理, 该公理可以得到许多重要的定理, 然而它的否定形式与前八条公理也可相容. 这种情况就类似于Euclid平面几何公理体系中的第五条, 当存在的时候就是常见的Euclid几何体系, 当不存在或存在其相反形式的时候就是另一套数学体系. 因此, 选择公理被独立于前八条之外. 

\begin{axiom}{选择公理}
	对于任何由互不相交且非空的集合形成的集合族, 存在另一个集合$C$, 使得对该集合族中的任意元素$X$, $X \cap C$恰有一个元素.
\end{axiom}

至此, 我们可以用一套公理体系来定义集合. 

\newpage
\section{映射与二元关系}

\subsection{映射}

本节内容在高中数学里已经出现过, 这里简要地复习概念并做一些推广.

\begin{definition}{映射}
	\vspace{-2em}
	\begin{itemize}
		\item 设$A$和$B$为两个集合, 若对$A$中每个元素$x$, 都存在$B$中唯一的元素$y$与之对应, 则称此对应关系为一个\textit{映射}(map), 记作$$f: A \to B, ~~x \mapsto y.$$
		\item $x$在$B$中的对应元素$y$称为$x$在$f$下的\textit{象}(image), $x$称为$y$在$f$下的\textit{原象}(preimage), 记作$$f(x) = y, ~ x \in A.$$
		\item 集合$A$称作映射$f$的\textit{定义域}(domain); 集合$B$称为映射$f$的\textit{陪域}(codomain); $A$中所有元素在$f$下的象组成的集合称为$f$的\textit{值域}(range), 记作$f(A)$.
		\item 两个映射相等, 当且仅当它们的定义域、对应关系、陪域相同.
	\end{itemize}
\end{definition}

从集合论的视角看, 一个映射其实就是确定的三元组$(A, B, f)$, 其中$A$是定义域, $B$是陪域, $f$是对应关系.

\begin{definition}{部分映射}
	设映射$f: X \to Y$与集合$A \subseteq X$, 定义$f$在$A$上的\textit{部分映射}(partial mapping)为: $$f|_A : = A \to X, ~~x \mapsto f(x).$$
\end{definition}
\begin{remark}
	部分映射$f|_A$的值域就是$f(A)$.
\end{remark}

利用部分映射, 我们可以得到一个新的记号$f(A)$, 表示在$f$映射下, 包含在定义域中的集合$A$在陪域中所对应的那个集合, 即$$f(A) : = \{ y \in Y: \exists x,  (x \in A) \wedge (y=f(x)) \}.$$
在$A$就是定义域本身的时候, 容易发现$f(A)$是$f$的值域.

同样地, 还能定义另一个记号$f^{-1}(B)$, 表示包含在值域中的集合$B$在定义域中对应的那个集合, 即$$f^{-1}(B) : = \{ x \in X: f(x) \in B \}.$$

用一张图就能很好地表示上述定义: 

\begin{figure}[h!]
	\centering
	\includegraphics[width=8cm]{attachment/Acr1745354698752707434.pdf}
	\caption{与映射相关的一些集合, 图源Zorich Fig. 1.6}
\end{figure}

\begin{definition}{双射}
	设映射$f: A \to B$.
	\begin{itemize}
		\item 称$f$是\textit{单射}(injection), 若$\forall x,y \in A, f(x)=f(y) \Rightarrow x=y$. 
		\item 称$f$是\textit{满射}(surjection), 若$\forall y \in B, \exists x \in A(f(x)=y)$. 
		\item 称$f$是\textit{双射}(bijection), 若$f$既是单射又是满射. 
	\end{itemize}
\end{definition}

\begin{definition}{映射的复合}
    设映射$f: A \to B$, $g: B \to C$, 则它们的\textit{复合映射}(composite mapping)~$gf: A \to C$定义为$$(gf)(x)=g(f(x)) \ (x \in A).$$
    注意复合运算有先后顺序. 另外, 为了强调复合运算, $gf$也可记作$g \circ f$.
\end{definition}

容易验证, 这样的“乘法”运算满足结合律与分配律、不满足交换律.

\begin{definition}{恒等映射}
	设映射$f: A \to A$.称$f$是$A$上的一个\textit{恒等映射}(identity mapping), 如果$$\forall x\in A, ~f(x)=x.$$
	并把$f$记作$\mathcal{I}_A$.
\end{definition}
\begin{remark}
	设映射$f :  A \to B$, 容易验证有
	\begin{center}
		$f\mathcal{I}_A=f, \quad \mathcal{I}_Bf=f.$
	\end{center}
\end{remark}
\begin{remark}
	需要证明恒等映射是良定义的, 即集合$A$上的所有恒等映射是相等的: 假设存在两个不同的恒等映射$\mathcal{I}_1, \mathcal{I}_2$, 那么$\mathcal{I}_1 = \mathcal{I}_1 \mathcal{I}_2 = \mathcal{I}_2$, 这与假设矛盾.
\end{remark}

\begin{definition}{逆映射}
	设映射$f: A \to B$.称$f$是\textit{可逆的}(inverible), 如果存在映射$g: B \to A$满足$$fg=\mathcal{I}_B, \quad gf=\mathcal{I}_A.$$
	特别地, 称$g$为$f$的\textit{逆映射}(inverse mapping).
\end{definition}
\begin{remark}
	必须要求$g$和$f$的两种复合均等于恒等映射. 
\end{remark}

逆映射是唯一的. 实际上, 设映射$g_1, g_2$为$f: A \to B$的不同的逆映射, 那么$g_1 = g_1\mathcal{I}_B = g_1fg_2 = \mathcal{I}_Ag_2 = g_2$, 这与假设矛盾. 既然一个映射的逆映射是唯一的, 我们可以用符号$f^{-1}$来表示它. 需要区分逆映射与原象集.

有些函数在定义域上并非是可逆的, 然而利用部分映射可以得到其一部分的逆映射, 例如三角函数.

下面的命题刻画了何时映射是可逆的.

\begin{proposition}{可逆性等价于双射性}
	设映射$f: A \to B$, 则$f$可逆当且仅当它是双射.
\end{proposition}
\begin{proof}
	(1) 必要性: 设$f$可逆, 即存在映射$g: B \to A$满足$fg=\mathcal{I}_B, gf=\mathcal{I}_A$. 下面证明$f$是双射. 
	
	设$x, y \in A$使得$f(x)=f(y)$, 那么由$x=gf(x)=gf(y)=y$可知$f$是单射. 另一方面, 设$z \in B$, 由于$z=fg(z)$, 这表明$B \subseteq f(A)$, 故$B = f(A)$, 于是$f$是满射. 
	
	(2) 充分性: 设$f$是单射和满射, 下面证明存在映射$g: B \to A$满足$fg=\mathcal{I}_B, gf=\mathcal{I}_A$. 
	
	人为地取$g$, 使得$g(x)$是$A$中唯一使得$f(g(x))=x$的那个元素(唯一存在性由$f$是双射可以得到保证).按照$g$的定义, 自然有$fg=\mathcal{I}_B$. 另一方面, 任取$x \in A$, 由于$f(gf(x)) = (fg)(f(x)) = f(x)$并且$f$是单射, 可得$gf(x)=x$, 所以$gf=\mathcal{I}_A$.
\end{proof}

最后我们来看映射与集合运算的关系: 

\begin{proposition}{映射与集合运算的关系}
	设映射$f:A \to B$, $E_{\alpha}$是$A$的子集, $E'_{\alpha}$是$B$的子集(对任意$\alpha \in I$). 则: 
	$$f \left( \bigcup_{\alpha \in I} E_{\alpha} \right) = \bigcup_{\alpha \in I} f(E_{\alpha});\qquad f \left( \bigcap_{\alpha \in I} E_{\alpha} \right) \subseteq \bigcap_{\alpha \in I} f(E_{\alpha}); $$
	$$f^{-1} \left( \bigcup_{\alpha \in I} E'_{\alpha} \right) = \bigcup_{\alpha \in I} f^{-1}(E'_{\alpha});\qquad f^{-1} \left( \bigcap_{\alpha \in I} E'_{\alpha} \right) = \bigcap_{\alpha \in I} f^{-1}(E'_{\alpha}). $$
\end{proposition}
\begin{remark}
	证明从略. 注意第二个式子的等号不一定能取得. 实际上还可以证明, 等号取得当且仅当$f$是单射. 
\end{remark}

\subsection{二元关系}

幂集公理允许我们构造两个集合的Cartesian积.

\begin{definition}{Cartesian积}
	设集合$A$和$B$, 定义它们的\textit{Cartesian积}(Cartesian product,  direct product)如下: $$A \times B : = \{ (a, b): a \in A, b \in B \}.$$
\end{definition}
\begin{remark}
	不难发现Cartesian积是一个可逆的过程, 也即任何一个在$A \times B$中的元素都可以回溯到其在$A$和$B$中的对应元素.因而Cartesian积不满足交换律和结合律.
\end{remark}
\begin{remark}
	特别地, 记$A^2: =A \times A$, 以及$A^n : = A^{n-1} \times A~(n \geq 2)$.
\end{remark}

\begin{definition}{二元关系}
	设非空集合$S$, 则称$S^2$的一个子集$\mathcal{R}$为$S$上的一个\textit{二元关系}(binary relation).若$(a, b) \in \mathcal{R}$, 则称$a, b$有$\mathcal{R}$关系, 记作$a\mathcal{R}b$.
\end{definition}

例如, 对于集合族$M$, 定义在$M$上的关系$$\Delta : = \{ (X, Y) \in M^2 :  \forall x, (x \in X) \Leftrightarrow (x \in Y) \}, $$
那么集合$A, B$相等就可以表述为$A \Delta B$.

一类在数学中很重要的关系就是等价关系, 它为我们阐明了数学对象的相似性和一致的本质.

\begin{definition}{等价关系}
	设集合$S$及定义在$S$上的关系$\mathcal{R}$, 如果对任意$a, b, c \in S$都有: 
	\begin{enumerate}
		\item 自反性: $a\mathcal{R} a$; 
		\item 对称性: $a\mathcal{R} b \Rightarrow b\mathcal{R} a$; 
		\item 传递性: $a\mathcal{R} b \wedge b\mathcal{R} c \Rightarrow a\mathcal{R} c$.
	\end{enumerate}
	则称$\mathcal{R}$是$S$上的一个\textit{等价关系}(equivalence relation), 通常记作$\sim$.
\end{definition}

把所有等价的元素放在一起, 就形成了\textit{等价类}(equivalence class). 具体地, 定义$a$在$\mathcal{R}$下的等价类$$[a]_{\mathcal{R}} : = \{ x \in S: x\mathcal{R}a \}, $$其中$\mathcal{R}$是$S$上的一个等价关系.

例如, 数论中模$n$的同余关系就是一类等价关系, 而模$n$的同余类就是等价类.

等价类内元素都具有同等地位, 都能代表整个等价类, 否则它们也不会被称作是等价的.

\begin{proposition}{等价类相等等价于代表元素等价}
	设$\mathcal{R}$是$S$上的等价关系, 对于$a, b \in S$有$[a]_{\mathcal{R}} = [b]_{\mathcal{R}} \Leftrightarrow a\mathcal{R} b$. 
\end{proposition}
\begin{proof}
	必要性显然. 充分性: 任取$c \in [a]_{\mathcal{R}}$, 由传递性知$c \mathcal{R} b$, 所以$c \in [b]_{\mathcal{R}}$, 从而$[a]_{\mathcal{R}} \subseteq [b]_{\mathcal{R}}$.同理有$[b]_{\mathcal{R}} \subseteq [a]_{\mathcal{R}}$, 所以$[a]_{\mathcal{R}} = [b]_{\mathcal{R}}$.
\end{proof}

还是以模$n$的同余类为例. 我们发现, 任何一个整数都会出现且仅会出现在一个同余类里, 换句话说, 所有的同余类构成类对整数集合的划分. 

一般地, 所有的等价类都可以构成对特定集合的划分. 

\begin{definition}{集合的划分}
	对于给定集合$S$, 集合族$X=\{ S_{\alpha} :  \alpha \in I \}$. 称$X$是$A$的一个\textit{划分}(partition), 如果
	\begin{center}
		1) $S = \bigcup_{\alpha \in I} S_{\alpha}$; \qquad 2) $\forall \alpha \neq \beta , ~S_{\alpha} \cap S_{\beta}$.
	\end{center}
\end{definition}

\begin{proposition}{}
	设$\mathcal{R}$是$S$上的一个等价关系, 则集合族$\{ [a]_{\mathcal{R}}: a \in S \}$构成了$S$的一个划分. 特别地, 称该集合为\textit{$S$模$\mathcal{R}$的商集}, 记作$S / \mathcal{R}$. 
\end{proposition}
\begin{proof}
	首先我们证明, 所有$[a]_{\mathcal{R}}$的并集恰等于$S$. 注意到对任意$a \in S$, $a \in [a]_{\mathcal{R}} \wedge [a]_{\mathcal{R}} \subseteq S$. 所以$S \subseteq \bigcup_{a \in S} [a]_{\mathcal{R}} \subseteq S$, 从而$S = \bigcup_{a \in S} [a]_{\mathcal{R}}$. 
	
	接着证明这些集合都是不交并. 对于$[a]_{\mathcal{R}} \neq [b]_{\mathcal{R}}$, 假设存在$c \in [a]_{\mathcal{R}} \cap [b]_{\mathcal{R}}$, 那么$c \in [a]_{\mathcal{R}} \wedge c \in [b]_{\mathcal{R}}$, 由等价关系的传递性, $a\mathcal{R}b$, 与假设矛盾. 于是该集合族中任意两个元素交集为空.
\end{proof}

借助商集, 我们可以证明一个有趣的命题: 

\begin{proposition}{}
	设集合$A,B$. 则任一映射$f:A \to B$均可被写成满射$\varphi :A \to C$与单射$\psi :C \to B$的复合. 
\end{proposition}
\begin{proof}
	考虑等价关系$\sim := \{ (x_1,x_2) \in A^2:f(x_1)=f(x_2) \}$, 取$C = A / \sim$. 构造$\varphi :A \to C,x \mapsto [x]_{\sim}, \psi : C \to B,[x]_{\sim} \mapsto f(x)$. 显然$\varphi$是满射, $\psi$是单射, 且$f=\psi \circ \varphi$. 
\end{proof}

类比等价关系, 可以定义序关系. 然而就像实数集中的$<$和$\leq$关系一样, 序关系可能有两种形式: 严格的和不严格的. 一般地, 我们更希望使用后者, 例如后面会介绍数列极限运算是维持不严格序关系的. 

容易看出, 上面两种情况的区别在于自反性, 所以只需要把下方定义中的自反性去掉, 就能得到严格偏序关系的定义. 

\begin{definition}{偏序关系}
	设集合$S$及定义在$S$上的关系$\mathcal{R}$, 如果对任意$a, b, c \in S$都有: 
	\begin{enumerate}
		\item 自反性: $a\mathcal{R} a$; 
		\item 反对称性: $a\mathcal{R} b \wedge b\mathcal{R} a \Rightarrow a=b$; 
		\item 传递性: $a\mathcal{R} b \wedge b\mathcal{R} c \Rightarrow a\mathcal{R} c$.
	\end{enumerate}
	则称$\mathcal{R}$是$S$上的一个\textit{偏序关系}(partially ordered relation), 记作$\preceq$.
\end{definition}

为什么偏(partially, 部分地)序关系不直接称作序关系呢? 这是因为, 有些序关系并不能覆盖所有元素. 例如对于给定集合的幂集, 其中某些元素并不存在包含关系. 再例如, 实数间的大小关系就可以覆盖所有元素. 从而引出另一个概念, 全序关系: 

\begin{definition}{全序关系}
	设集合$S$及定义在$S$上的关系$\mathcal{R}$, 如果对任意$a, b, c \in S$都有: 
	\begin{enumerate}
		\item 反对称性: $a\mathcal{R} b \wedge b\mathcal{R} a \Rightarrow a=b$; 
		\item 传递性: $a\mathcal{R} b \wedge b\mathcal{R} c \Rightarrow a\mathcal{R} c$; 
		\item 完全性: $a\mathcal{R} b \vee b\mathcal{R} a$.
	\end{enumerate}
	则称$\mathcal{R}$是$S$上的一个\textit{全序关系}(totally ordered relation), 同时称$S$是一个\textit{全序集}(totally ordered set).
\end{definition}
\begin{remark}
	完全性蕴含了自反性.
\end{remark}

\newpage
\section{集合的基数}

高中数学中, 我们学过有限集合的元素个数. 从直观上看, 似乎无限集合不会存在元素个数这一说法, 但我们又熟知实数远比整数多, 那么这种相对的元素个数比较是怎样建立的? 

来考虑这样一个问题: 给定两个有限集合$A, B$, 如何比较它们的元素个数. 最一般的想法应该是在它们之间构造一个映射$f: A \to B$, 如果$f$是双射则$A, B$元素个数相等, 如果是单射则$A$的元素个数不多于$B$的元素个数, 如果是满射则$B$的元素个数不多于$A$的元素个数(这些用反证法容易说明). 

相对应地, 既然我们只需要考虑无限集合之间的相对“元素个数”多少, 而不需要得到一个绝对数值, 就可以仿照上方的方法定义一个无限集合的“相对元素个数”. 非常直观地, 我们也将其称为“势”, 这是否让你想起电势? 在接下来的内容中, 你将看到集合的“势”的参考位置一般取用自然数集合. 

\begin{definition}{等势集合}
	对于集合$A, B$, 若存在单射$f: A \to B$, 则称$A$的势小于等于$B$. 特别地, 若单射$f$同时也是一个满射, 即$f$是双射, 则称$A, B$\textit{等势}(equipollent). 
\end{definition}

很自然地, 我们可以证明集合的等势关系是一个等价关系. 为了证明势的小于等于是一个全序关系, 需要下方的定理: 

\begin{theorem}{Schröder–Bernstein} \label{thm:sb}
	给定集合$A, B$.若在$A, B$间存在两个单射$f: A \to B$与$g: B \to A$, 则在它们之间也存在一个双射$h: A \to B$.
\end{theorem}
\begin{proof}
	\underline{\textbf{证法一}}(不依赖选择公理的构造性证明)~~不妨考虑$A,B$非空. 由于$g$是$B \to g(B)$的双射, 我们可以用$g(B)$替换$B$, 即不妨设$f(A) \subseteq B \subseteq A$. 令$A_0=A, B_0=B$, 递归地定义$A_{n+1} = f(A_n), B_{n+1}=f(B_n)$. 于是得到$A_0 \supseteq B_0 \supseteq \cdots \supseteq A_n \supseteq B_n \supseteq \cdots$. 直接给出$h$的构造: $$h(x) = \begin{cases}
		f(x) & \exists n \in \Z _{+}, x \in A_n-B_n \\ x & \textit{否则}
	\end{cases}. $$
	
	下面验证$h$是双射: 若$h(x)=h(y)$而$x \neq y$, 只能$x \in A_N-B_N$, $y \notin A_n-B_n, \forall n$. 但是$y=h(y)=h(x)=f(x) \in A_{N+1}, \notin B_{N_1}$, 即$y \in A_{N+1}-B_{N+1}$, 矛盾. 这说明$h$是单射. 另一方面, 任取$y \in B$, 若$y \notin f(A)$, 由$B_0 \supseteq A_1$可知不存在$n \geq 1$使得$f(y) \in A_n$, 从而$h(y)=y$. 这说明$h$是满射. 
\end{proof}

由上方的定理, 容易得到势的小于等于关系满足反对称性. 该关系的完全性是选择公理的推论(这里略去). 再加上传递性(例如, $A, B$之间存在单射$f$, $B, C$之间存在单射$g$, 则$g|_{f(A)} \circ f$是$A, C$间的单射), 马上得到该关系是一个全序关系. 

从而, 我们可以利用等价类的思想刻画一个无限集合的相对元素个数.

\begin{definition}{集合的基数}
	\vspace{-2em}
	\begin{itemize}
		\item 设集合的等势关系$\mathcal{R}$.对于集合$X$, 称$[X]_{\mathcal{R}}$为其\textit{基数}(cardinal)或势, 记作$|X|$.
		\item 定义$|X| = |Y|$, 如果$X$与$Y$等势.
		\item 定义$|X| \leq |Y|$, 如果$X$与$Y$的某个子集等势.
	\end{itemize}
\end{definition}

容易证明集合基数的小于等于关系也是一个全序关系. 

现在对用等价类定义的基数做一些说明: 这种定义方式其实不够好, 因为如果我们要考虑所有基数构成的“集合”, 实际上是在考虑一个集合族, 而集合族不一定是集合. 更好的方法是取等价类中的某个代表元素(一般取的是最小序数). 利用取代表元素的定义方法, 我们可以引入定理\ref{thm:sb}的第二种证明: 

\begin{proof}
	\underline{\textbf{证法二}}(承认选择公理的证明, 了解即可)~~先不加证明地给出一个引理(良序定理): 任何集合$S$上均存在一个序结构$\prec$, 使得$(S,\prec)$是一个良序结构, 即对任意$S$的子集都存在关于$\prec$的极小元. 等价地, 存在唯一的极小序数$|S|$使得$S,|S|$之间存在双射. 在下方的证明中, 我们实际上将基数的反对称性处理成了序数的反对称性(不加证明地承认). 
	
	回到原题, 不妨考虑$A,B$非空. 则存在唯一的极小序数$|A|$和双射$\varphi :A \to |A|$. 类似地定义$|B|$和$\psi :B \to |B|$. 由于$\psi \circ f:X \to |Y|$是单射, 将其视作陪域等于值域的映射是就是双射. 由替换公理, 其值域$\rge (\psi \circ f)$亦是集合, 故可良序化, 即存在唯一极小序数$|\rge (\psi \circ f)|$和双射$\alpha :\rge (\psi \circ f) \to |\rge (\psi \circ f)|$. 于是$$\alpha \circ \psi \circ f:X \to |\rge \psi \circ f| \leq |Y|$$
	是双射, 而由$|X|$的极小性可知$|X| \leq |\rge \psi \circ f| \leq |Y|$. 同理可得$|Y| \leq |X|$. 于是$|X|=|Y|$. 
\end{proof}



关于无限集合, Cantor曾证明: (这里$|X|<|Y|$承自然的严格偏序定义)

\begin{theorem}{}
	设集合$X$, 则$|X|< |\mathcal{P}(X)|$.
\end{theorem}
\begin{proof}
	若$X$是空集, 则显然成立. 从而, 只考虑$X$非空的情况. 
	
	由于$\mathcal{P}(X)$涵盖所有$X$的一元子集, 故显然有$|X| \leq |\mathcal{P}(X)|$. 假设有$|X| = | \mathcal{P}(X)|$, 那么存在双射$f: X \to X$. 
	
	根据$f$, 取$B=\{ x \in X: x \notin f(x) \}$, 显然$B \in \mathcal{P}(A)$, 从而存在$x$使得$f(x)=B$. 此时, 若$x \in B$, 则由$B$的定义知$x \notin B$, 矛盾; 同理, 若$x \notin B$, 则可得$x \in B$, 也矛盾. 
\end{proof}




% 范畴论初步

\chapter{范畴论初步}

\begin{definition}{范畴}
	
\end{definition}


\begin{example} % Aluffi 23 https://en.wikipedia.org/wiki/Comma_category
	Slice Categories
\end{example}





\part{线性映射与代数结构}

% 向量空间

\chapter{向量空间}

\section{从$\F ^{n}$说起}

\subsection{复数与复数域}

首先来温习一下复数域$\C$的定义与它满足的性质:

\begin{definition}{复数}
	记$z=a+b\ic $($a,b \in \R$)为一个\textit{复数},其中$\ic ^2=-1$.由所有复数构成的集合记为$\C$. \\
	$\C$上的加法与乘法定义如下:
	$$(a+b\ic ) + (c+d\ic ) = (a+c) + (b+d)\ic $$
	$$(a+b\ic )(c+d\ic ) = (ac-bd) + (ad+bc)\ic $$
\end{definition}

\begin{proposition}{复数运算的性质}{Fxkvi}
	(1) 交换性质$$\forall \alpha , \beta \in \C , \alpha + \beta = \beta + \alpha , \alpha \beta = \beta \alpha$$
	(2) 结合性质$$\forall \alpha , \beta , \lambda \in \C , (\alpha + \beta) + \lambda = \alpha + (\beta + \lambda) , (\alpha \beta) \lambda = \alpha (\beta \lambda)$$
	(3) 单位元$$\forall \lambda \in \C , \lambda + 0 = \lambda , 1 \lambda = \lambda$$
	(4) 加法逆元$$\forall \alpha \in \C , \exists ! \beta \in \C , \alpha + \beta = 0$$
	(5) 乘法逆元$$\forall \alpha \in \C (\alpha \neq 0) , \exists ! \beta \in \C , \alpha \beta = 1$$
	(6) 分配性质$$\forall \lambda , \alpha , \beta \in \C , \lambda (\alpha + \beta) = \lambda \alpha + \lambda \beta$$
\end{proposition}
\begin{proof}
	这里只选择部分性质证明: \\
	(1) 加法交换性质:设$\alpha = a+b\ic , \beta = c+d\ic ~(a,b,c,d \in \R )$,则
	\begin{align*}
		\alpha + \beta &= (a+b\ic ) + (c+d\ic ) \\
		&= (a+c) + (b+d)\ic \\
		&= (c+a) + (d+b)\ic \\
		\beta + \alpha &= (c+d\ic ) + (a+b\ic ) \\
		&= (c+a) + (d+b)\ic
	\end{align*}
	因此有$\alpha + \beta = \beta + \alpha$ \\
	(2) 乘法单位元:设$\lambda = a+b\ic ~ (a,b \in \R )$,那么$$1 \lambda = (1+0\ic )(a+b\ic ) = a + b\ic = \lambda$$
	(3) 加法逆元:先证明存在.设$\alpha = a+b\ic $,取$\beta = (-a) + (-b)\ic $,则$\alpha + \beta = 0+0\ic = 0$;\\
	再证明唯一.假设$\beta _1, \beta _2 \in \C $均为$\alpha$的加法逆元,那么$$\beta _1 = \beta _1 + 0 = \beta _1 + \alpha + \beta _2 = 0 + \beta _2 = \beta _2$$
	这与假设矛盾,则$\alpha$的加法逆元是唯一的.
\end{proof}

由此可以引出\textit{域}的正式定义:

\begin{definition}{域}
	\textit{域}是一个集合$\F$,它带有加法与乘法两种运算(分别在加法与乘法上封闭),且这些运算满足命题\ref{pro:Fxkvi}所示所有性质.
\end{definition}
\begin{remark}
	最小的域是一个集合$\{ 0,1 \}$,带有通常的加法与乘法运算,但规定$1+1=0$.
\end{remark}

容易验证,$\R$与$\C$都是域.本书中用$\F$来表示$\R$或$\C$.

总是用$\beta$表示$\alpha$的逆元非常不自然,因此定义出加/乘法逆元的表示与减/除法.

\begin{definition}{加法逆元,减法,乘法逆元,除法}
	设$\alpha , \beta \in \C $.
	\begin{itemize}
		\item 令$- \alpha$表示$\alpha$的加法逆元,即$-\alpha$是使得$$\alpha + (-\alpha) = 0$$成立的唯一复数.
		\item 对于$\alpha \neq 0$,令$\alpha ^{-1}$表示$\alpha$的乘法逆元,即$\alpha ^{-1}$是使得$$\alpha (\alpha ^{-1}) = 1$$成立的唯一复数.
		\item 定义$\C $上的\textit{减法}:$$\beta - \alpha = \beta + (-\alpha)$$
		\item 定义$\C $上的\textit{除法}:$$\beta / \alpha = \beta (1 / \alpha)$$
	\end{itemize}
\end{definition}

\subsection{$\F ^{n}$}

在中学的向量板块,我们认识到一个向量可以表示为有序数组$(a,b)$的形式,并且在立体几何板块利用三维下的向量进行了许多计算.那么向量的定义能否推广到更高维度呢?

\begin{definition}{$\F ^{n}$}
	$\F ^{n}$是$\F$中元素组成的长度为$n$的组的集合,即$$\F ^{n} = \{ (x_1,\cdots ,x_n) : x_j \in \F , j=1, \cdots ,n \}$$
	特别地,对于由无限长度序列构成的集合,称作$\F ^{\infty}$,即
	$$\F ^{\infty} = \{ (x_1,\cdots ,x_n, \cdots) : x_j \in \F , j=1, \cdots ,n, \cdots \}$$
	对于$\F ^{n}$中的某个元素$(x_1,\cdots ,x_n)$,称$x_j ~(i=1,\cdots ,n)$为$(x_1,\cdots ,x_n)$的第$j$个\textit{坐标}. \\
	$\F ^{n}$上的\textit{加法}定义为对应坐标相加,即
	$$(x_1, \cdots , x_n) + (y_1 , \cdots , y_n) = (x_1+y_1, \cdots , x_n+y_n)$$
	对于$\F ^{\infty}$
	$$(x_1, \cdots , x_n, \cdots) + (y_1 , \cdots , y_n ,\cdots) = (x_1+y_1, \cdots , x_n+y_n ,\cdots)$$
	$\F ^{n}$上的\textit{标量乘法}:一个数$\lambda ~(\lambda \in \F )$与$\F ^{n}$中元素的乘积这样计算:用$\lambda$乘以该元素的每个坐标,即
	$$\lambda (x_1,\cdots ,x_n) = (\lambda x_1, \cdots ,\lambda x_n)$$
	对于$\F ^{\infty}$
	$$\lambda (x_1,\cdots ,x_n, \cdots) = (\lambda x_1, \cdots ,\lambda x_n,\cdots)$$
	我们暂时不讨论$\F ^{n}$上元素之间的乘法.
\end{definition}

当$\F$代表$\R$且$n=2,3$时,$\F ^{n}$中的元素就相当于我们熟悉的平面向量、空间向量.实际上,所有在$\F$中的元素都被称为\textit{标量},所有在$\F ^{n}$中的元素如果被看做是一个从原点指向某定点的有向线段时,它就是\textit{向量}.我们一般用小写字母表示标量,用加粗的小写字母表示$\F ^{n}$中的元素,例如$\F ^{4}$中的元素$$\boldsymbol{x} = (x_1,x_2,x_3,x_4)$$
特别地,用$\boldsymbol{0}$表示所有坐标全是$0$的元素,即$$\boldsymbol{0} = (0, \cdots , 0)$$

$\F ^{n}$同样也具有类似于$\F$的一些性质:

\begin{proposition}{$\F ^{n}$的性质}{xlxkvi}
	(1) 交换性质$$\forall \boldsymbol{u},\boldsymbol{v} \in \F ^{n} , \boldsymbol{u} + \boldsymbol{v} = \boldsymbol{v} + \boldsymbol{u}$$
	(2) 结合性质$$\forall \boldsymbol{u},\boldsymbol{v},\boldsymbol{w} \in \F ^{n}, a,b \in \F, (\boldsymbol{u} + \boldsymbol{v}) + \boldsymbol{w} = \boldsymbol{u} + (\boldsymbol{v} + \boldsymbol{w}) , (ab) \boldsymbol{v} = a (b\boldsymbol{v})$$
	(3) 加法单位元$$\exists ! \boldsymbol{0} \in \F ^{n}, \forall \boldsymbol{v} \in \F ^{n} , \boldsymbol{v} + \boldsymbol{0} = \boldsymbol{v}$$
	(4) 加法逆元$$\forall \boldsymbol{v} \in \F ^{n} , \exists ! \boldsymbol{w} \in \F ^{n} , \boldsymbol{v} + \boldsymbol{w} = \boldsymbol{0}$$
	(5) 乘法单位元$$\forall \boldsymbol{v} \in \F ^{n} , 1\boldsymbol{v} = \boldsymbol{v}$$
	(6) 分配性质$$\forall a,b \in \F , \boldsymbol{u},\boldsymbol{v} \in \F ^{n} , a (\boldsymbol{u} + \boldsymbol{v}) = a\boldsymbol{u} + a\boldsymbol{v} , (a+b)\boldsymbol{v} = a\boldsymbol{v}+b\boldsymbol{v}$$
\end{proposition}
\begin{proof}
	这里只选择部分证明:\\
	(1) 交换性质:设$\boldsymbol{u} = (u_1, \cdots ,u_n),\boldsymbol{v} = (v_1, \cdots ,v_n)$,则
	\begin{align*}
		\boldsymbol{u} + \boldsymbol{v} &= (u_1, \cdots ,u_n) + (v_1, \cdots ,v_n) \\
		&= (u_1+v_1, \cdots ,u_n+v_n) \\
		&= (v_1+u_1, \cdots ,v_n+u_n) \\
		&= (v_1, \cdots ,v_n) + (u_1, \cdots ,u_n) \\
		&= \boldsymbol{v} + \boldsymbol{u}
	\end{align*}
	(2) 加法单位元:先证明存在.若$\boldsymbol{v} = (v_1, \cdots ,v_n)$,取$\boldsymbol{-v} = (-v_1, \cdots ,-v_n)$,容易发现$\boldsymbol{v} + \boldsymbol{-v} = \boldsymbol{0}$; \\
	再证明唯一.假设存在两个加法单位元$\boldsymbol{0}$与$\boldsymbol{0'}$,则$$\boldsymbol{0} = \boldsymbol{0} + \boldsymbol{0'} = \boldsymbol{0'} + \boldsymbol{0} = \boldsymbol{0'}$$
	这与假设矛盾.因此最多只有一个加法单位元.
\end{proof}

\newpage
\section{向量空间}

类似于$\F ^{n}$,我们把向量空间定义为带有加法和标量乘法的集合$V$,其满足命题\ref{pro:xlxkvi}中的性质.请注意,由于不一定满足乘法交换性质,向量空间不一定是一个域.

\begin{definition}{加法,标量乘法}
	\begin{itemize}
		\item 集合$V$上的\textit{加法}是一个函数,它把每一对$u,v \in V$都对应到$V$中的一个元素$u+v$.
		\item 集合$V$上的\textit{标量乘法}是一个函数,它把任意$\lambda \in \F $和$v \in V$都对应到$V$中的一个元素$\lambda v$.
	\end{itemize}
\end{definition}
\begin{remark}
	换句话说,$V$对加法和标量乘法封闭.
\end{remark}

接下来可以正式定义向量空间:

\begin{definition}{向量空间}
	\textit{向量空间}就是带有加法和标量乘法的集合$V$,满足如下性质: \\
	(1) 交换性质$$\forall \boldsymbol{u},\boldsymbol{v} \in V , \boldsymbol{u} + \boldsymbol{v} = \boldsymbol{v} + \boldsymbol{u}$$
	(2) 结合性质$$\forall \boldsymbol{u},\boldsymbol{v},\boldsymbol{w} \in V, a,b \in \F, (\boldsymbol{u} + \boldsymbol{v}) + \boldsymbol{w} = \boldsymbol{u} + (\boldsymbol{v} + \boldsymbol{w}) , (ab) \boldsymbol{v} = a (b\boldsymbol{v})$$
	(3) 加法单位元$$\exists \boldsymbol{0} \in V, \forall \boldsymbol{v} \in V , \boldsymbol{v} + \boldsymbol{0} = \boldsymbol{v}$$
	(4) 加法逆元$$\forall \boldsymbol{v} \in V , \exists \boldsymbol{w} \in V , \boldsymbol{v} + \boldsymbol{w} = \boldsymbol{0}$$
	(5) 乘法单位元$$\forall \boldsymbol{v} \in V , 1\boldsymbol{v} = \boldsymbol{v}$$
	(6) 分配性质$$\forall a,b \in \F , \boldsymbol{u},\boldsymbol{v} \in V , a (\boldsymbol{u} + \boldsymbol{v}) = a\boldsymbol{u} + a\boldsymbol{v} , (a+b)\boldsymbol{v} = a\boldsymbol{v}+b\boldsymbol{v}$$
	向量空间中的元素被称为\textit{向量}或\textit{点}.
\end{definition}
\begin{remark}
	因为向量空间的标量乘法依赖于$\F$,所以一般会说$V$是$\F$ \textit{上的向量空间}.例如,平面点集$\R ^{2}$是$\R$上的向量空间.如果没有特别说明,默认$V$就表示在$\F$上的向量空间.
\end{remark}
\begin{remark}
	最小的向量空间是$\{ 0 \}$,它带有通常的加法和乘法运算.
\end{remark}
\begin{note}
	在向量空间的定义中并没有说明唯一性,这是因为唯一性可以通过已有的性质证明出.
\end{note}

现在介绍一个具体的例子:

\begin{definition}{$\F ^{S}$}
	设$S$是一个集合,我们用$\F ^{S}$表示$S$到$\F$的所有函数的集合. \\
	对于$f,g \in \F ^{S}$,对所有$x \in S$,规定$\F ^{S}$上的加和$f+g$满足$$(f+g)(x) = f(x) + g(x)$$
	对于$\lambda \in \F$和$f \in \F ^{S}$,对所有$x \in S$,规定$\F ^{S}$上的标量乘法得到的乘积$\lambda f \in \F ^{S}$满足$$(\lambda f)(x) = \lambda f(x)$$
\end{definition}

\begin{example}
	请证明$\F ^{S}$是$\F$上的向量空间,并指出它的加法单位元与加法逆元.
\end{example}

向量空间的定义中缺少了一些显而易见的性质,我们现在进行补充:

\begin{proposition}{向量空间的性质}{xlkjxkvi}
	\begin{itemize}
		\item 向量空间有唯一的加法单位元.
		\item 向量空间中的每个元素都有唯一的加法逆元.
		\item 对任意$\boldsymbol{v} \in V$都有$0\boldsymbol{v}=\boldsymbol{0}$.
		\item 对任意$a \in \F$都有$a\boldsymbol{0}=\boldsymbol{0}$.
		\item 对任意$\boldsymbol{v} \in V$都有$(-1)\boldsymbol{v}=\boldsymbol{-v}$.(等式右边的$\boldsymbol{-v}$表示$\boldsymbol{v}$的加法逆元)
	\end{itemize}
\end{proposition}
\begin{proof}
	设向量空间$V$, \\
	(1) 假设$V$中有两个不同的加法单位元$\boldsymbol{0},\boldsymbol{0'}$,那么$$\boldsymbol{0} = \boldsymbol{0} + \boldsymbol{0'} = \boldsymbol{0'} + \boldsymbol{0} = \boldsymbol{0'}$$
	这与假设矛盾,于是向量空间中只有唯一的加法单位元. \\
	(2) 对于$\boldsymbol{v} \in V$,假设$\boldsymbol{w},\boldsymbol{w'}$都是它的加法逆元,那么$$\boldsymbol{w} = \boldsymbol{w}+0 = \boldsymbol{w} + \boldsymbol{v} + \boldsymbol{w'} = 0 + \boldsymbol{w'} = \boldsymbol{w'}$$
	这与假设矛盾,于是向量空间中每个元素都有唯一的加法逆元. \\
	(3) 对于$\boldsymbol{v} \in V$,由于$$0\boldsymbol{v} = (0+0)\boldsymbol{v} = 0\boldsymbol{v} + 0\boldsymbol{v}$$
	在等式两边同时加上$0\boldsymbol{v}$的加法逆元,可得$0\boldsymbol{v} = 0$. \\
	(4) 与(3)同理,请读者自行证明. \\
	(5) 对于$\boldsymbol{v} \in V$,由于$$0 = (1+(-1))\boldsymbol{v} = \boldsymbol{v} + (-1)\boldsymbol{v}$$
	在等式两边同时加上$\boldsymbol{v}$的加法逆元,可得$(-1)\boldsymbol{v} = \boldsymbol{-v}$.
\end{proof}
\begin{remark}
	在(3)的证明过程中,由于在向量空间中只有分配性质能将标量乘法与向量的加法联系在一起,故必然会利用分配性质.
\end{remark}

\newpage
\section{子空间}

就像构造集合时要研究一个集合的子集一样,在向量空间中,我们也要研究它的子集.特别地,向量空间的子集如果也是向量空间,我们把它称作\textit{子空间}.

\subsection{子空间}

\begin{definition}{子空间}
    设向量空间$V$和它的一个子集$U$(采用与$V$相同的加法法则与标量乘法法则),如果$U$也是一个向量空间,则称$U$是$V$的\textit{子空间}.
\end{definition}

然而在实际应用中,每遇到一个子集$U$都证明一遍它是向量空间是很麻烦的.其实只需要证明以下三个关键性质:

\begin{proposition}{子空间的判定条件}
    设向量空间$V$的子集$U$,$U$是$V$的子空间当且仅当$U$满足下列条件: \\
    (1) 加法单位元$$0 \in U$$
    (2) 加法封闭性$$\forall u,v \in U, u+v \in U$$
    (3) 标量乘法封闭性$$\forall \lambda \in \F,v \in U,\lambda v \in U$$
\end{proposition}
\begin{proof}
    \buzhou{1} 必要性:当$U$是$V$的子空间时,由定义可知$U$是一个向量空间,则它自然满足上述条件. \\
    \buzhou{2} 充分性:当$U$满足上述条件时,由于$U$是$V$的子集并拥有相同的运算规则,显然$U$可以满足向量空间的所有性质.
\end{proof}
\begin{remark}
    该判定条件中有关加法单位元的性质等价于“$U$非空”.(取$v \in U,0 \in \F$,由标量乘法封闭性与命题\ref{pro:xlkjxkvi}的第三条可知$0v=0 \in U$)
\end{remark}
\begin{remark}
    实际上子空间的判定条件就是向量空间的必要条件:拥有加法单位元,且对加法和标量乘法封闭.
\end{remark}

\begin{example}
    请指出下列向量空间的所有子空间:(不要求证明唯一性,我们会在下一章给出证明) \\
    (1)定义在$\R$上的向量空间$\R ^{2}$; \\
    (2)定义在$\R$上的向量空间$\R ^{3}$.
\end{example}
\begin{solution}
    (1)$\{ 0 \}$、$\R ^2$和$\R ^2$中过原点的所有直线. \\
    (2)$\{ 0 \}$、$\R ^3$和$\R ^3$中过原点的所有平面. 
\end{solution}

\begin{example}
    证明下列结论:\\
    (1)若$b \in \F$,则$U = \{ (x_1,x_2,x_3,x_4) \in \F ^{4} : x_3 = 5x_4+b \}$是$\F ^{4}$的子空间当且仅当$b=0$; \\
    (2)区间$[0,1]$上的全体实值连续函数的集合是$\R ^{[0,1]}$的子空间; \\
    (3)区间$(0,3)$上满足条件$f'(2)=b$的实值可微函数的集合是$\R ^{(0,3)}$的子空间当且仅当$b=0$; \\
    (4)极限为$0$的复数序列组成的集合是$\C ^{\infty}$的子空间.
\end{example}
\begin{proof}
	(1)\buzhou{1} 充分性:当$b=0$时,显然$0=(0,0,0,0) \in U$.取$U$中两个元素$v=(v_1,v_2,5v_4,v_4)$与$u=(u_1,u_2,5u_4,u_4)$,取$\F$中标量$\lambda$.因为
	$$v+u = (v_1+u_1,v_2+u_2,5v_4+5u_4,v_4+u_4) = (v_1+u_1,v_2+u_2,5(v_4+u_4),v_4+u_4) \in U$$
	$$\lambda v = (\lambda v_1,\lambda v_2,\lambda 5v_4,\lambda v_4) = (\lambda v_1,\lambda v_2,5(\lambda v_4),\lambda v_4) \in U$$
	这告诉我们$U$对加法和标量乘法封闭,于是$U$是$\F ^{4}$的子空间. \\
	\buzhou{2} 必要性:任取$U$中两个元素$v=(v_1,v_2,5v_4+b,v_4)$与$u=(u_1,u_2,5u_4+b,u_4)$,取$\F$中标量$\lambda$.因为
	$$(0,0,0,0) \in U$$
	$$v+u = (v_1,v_2,5v_4+b,v_4) + (u_1,u_2,5u_4+b,u_4) = (v_1+u_1, v_2+u_2, 5(v_4+u_4)+2b, v_4+u_4) \in U$$
	$$\lambda v = (\lambda v_1, \lambda v_2 , 5\lambda v_4 + \lambda b ,\lambda v_4) \in U$$
	则$0=0+b,~ 5(v_4+u_4)+2b = 5(v_4+u_4)+b,~ 5\lambda v_4 + \lambda b = 5\lambda v_4 + b$,这要求$b=0$. \\
	(3)\buzhou{1} 充分性:设函数$0:x \mapsto 0$,容易验证$0$是该集合的加法单位元;取函数$f,g \in \R ^{(0,3)}$,由于$(f+g)'(2)=f'(2)+g'(2)=0$,可知$f+g \in \R ^{(0,3)}$,即该集合对加法封闭;取函数$f \in \R ^{(0,3)}$,标量$\lambda \in \F$,由于$(\lambda f)'(2) = \lambda f'(2) = 0$,可知$\lambda f \in \R ^{(0,3)}$,即该集合对标量乘法封闭. \\
	\buzhou{2} 必要性:由例题1.2.1的结论,该集合中必有加法单位元$0:x \mapsto 0$,则$0'(2)=0=b$;取函数$f,g \in \R ^{(0,3)}$,由于该集合对加法封闭,可知$(f+g)'(2)=f'(2)+g'(2)=2b=b$,则$b=0$;取函数$f \in \R ^{(0,3)}$,标量$\lambda \in \F$,由于该集合对标量乘法封闭,有$(\lambda f)'(2) = \lambda f'(2) = \lambda b = b$,则$b=0$.
\end{proof}

\subsection{子空间的和}

继续与集合比较.我们发现集合间有交、并、补等运算,向量空间中也有对应的运算,不过我们感兴趣的通常是它们的\textit{和}.(详细原因参考本节习题)

\begin{definition}{子集的和}
    设$U_1,\cdots ,U_m$都是$V$的子集,定义$U_1, \cdots ,U_m$的\textit{和}为$U_1, \cdots ,U_m$中元素所有可能的和构成的集合,记作$U_1+ \cdots +U_m$,即$$U_1+ \cdots +U_m = \{ u_1+ \cdots +u_m : u_j \in U_j,j=1, \cdots ,m \}$$
\end{definition}

\begin{example}
    证明下列结论: \\
    (1)设$$U = \{ (x,0,0) \in \F ^{3} : x \in \F \} , \quad W = \{ (0,y,0) \in \F ^{3} : y \in \F \}$$
    则$$U+W = \{ (x,y,0) : x,y \in \F \}$$
    (2)设$$U = \{ (x,x,y,y) \in \F ^{4} : x,y \in \F \} , \quad W = \{ (x,x,x,y) \in \F ^{4} : x,y \in \F \}$$
    则$$U+W = \{ (x,x,y,z) : x,y,z \in \F \}$$
\end{example}

两个集合的并集是包含它们的最小集合.相应地,两个子空间的和是包含它们的最小子空间.

\begin{proposition}{子空间的和是包含这些子空间的最小子空间}{ziksjmdehe}
    设$U_1,\cdots ,U_m$都是$V$的子空间,则$U_1+\cdots +U_m$是$V$的包含$U_1,\cdots ,U_m$的最小子空间.
\end{proposition}
\begin{proof}
    记$U=U_1+\cdots +U_m$. \\
    \buzhou{1} 证明$U$是$V$的子空间:显然$0=0 + \cdots + 0 \in U$;取$x_1+ \cdots +x_m,y_1+ \cdots +y_m \in U$,其中$x_i,y_i \in U_i$($i=1,\cdots ,m$),由于对任意$i$都有$x_i+y_1 \in U_i$,所以$(x_1+y_1) + \cdots + (x_m+y_m)$也在$U$中,因此$U$对加法封闭;取$x_1+ \cdots +x_m \in U$,由于对任意$i$都有$\lambda x_i \in U_i$,所以$\lambda x_1 + \cdots + \lambda x_m$也在$U$中,因此$U$对标量乘法封闭.综上,$U$是$V$的子空间.\\
    \buzhou{2} 证明$U$包含$U_1,\cdots ,U_m$:取$U_j$中元素$u_j$,再取其他子空间中的元素$0$,可知$u_j \in U$.因此任意一个子空间都包含于$U$. \\
    \buzhou{3} 证明$U$是最小的满足条件的子空间:假设存在一个更小的$U'$,由于$U'$包含$U_1, \cdots ,U_m$中的所有元素,又因为$U'$对加法封闭,故$U'$中必有$U_1+ \cdots +U_m$中所有元素,这与假设矛盾.因此$U$是最小的满足条件的子空间.
\end{proof}

\subsection{直和}

注意到子空间的和中的元素$u$可以用不同的$u_1+ \cdots + u_m$来表示.为了尽量避免这种不确定性,规定一种能够唯一地表示为上述形式的情形.

\begin{definition}{直和}
    设$U_1,\cdots ,U_m$都是$V$的子空间.和$U_1 + \cdots + U_m$称为\textit{直和},如果$U_1+ \cdots +U_m$中的每个元素都能唯一地表示成$u_1+ \cdots + u_m$的形式,其中每个$u_j$都属于$U_j$.特别地,用$U_1 \oplus \cdots \oplus U_m$表示一个直和.
\end{definition}

\begin{example}
    证明下列结论: \\
    (1)设$$U = \{ (x,y,0) \in \F ^{3} : x,y \in \F \}, \quad W = \{ (0,0,z) \in \F ^{3} : z \in \F \}$$
    则$\F ^{3} = U \oplus W$. \\
    (2)设$U_j$是$\F ^{n}$中除第$j$个坐标以外其余坐标全是$0$的向量所组成的子空间(例如,$U_2= \{ (0,x,0,\cdots ,0) \in \F ^{n} : x \in \F \}$),则$\F ^{n} = U_1 \oplus \cdots \oplus U_n$. \\
    (3)设$$U_1 = \{ (x,y,0) \in \F ^{3} : x,y \in \F \}, \quad U_2 = \{ (0,0,z) \in \F ^{3} : z \in \F \}, \quad U_3 = \{ (0,y,y) \in \F ^{3} : y \in \F \}$$
    则$U_1+U_2+U_3$不是直和.
\end{example}

每次都要构造一个反例来说明某个和不是直和过于麻烦,实际上有一种更简易的判别方法:

\begin{proposition}{直和的判定条件}
    设$U_1,\cdots ,U_m$都是$V$的子空间.“$U_1 + \cdots + U_m$是直和”当且仅当“$0$表示成$u_1+\cdots +u_m$(其中每个$u_j$都属于$U_j$)的唯一方式是每个$u_j$都等于$0$”.
\end{proposition}
\begin{proof}
    \buzhou{1} 必要性:由定义可知,若$U_1 + \cdots + U_m$是直和,则$\boldsymbol{0}$只有一种表示.又由$\boldsymbol{0} + \cdots + \boldsymbol{0} = \boldsymbol{0}$(其中第$j$个$\boldsymbol{0}$属于$U_j$)可知,这是唯一的表示方法. \\
    \buzhou{2} 充分性:设$U_1 + \cdots + U_m$中元素$v$,若$v$可以表示为$u_1 + \cdots + u_m$或$v_1 + \cdots + v_m$(其中$u_j,v_j \in U_j$),那么$0 = (u_1 - v_1) + \cdots + (u_m - v_m)$,即$u_j=v_j ~(j=1,\cdots ,m)$,于是$U_1 + \cdots + U_m$是直和.
\end{proof}

\begin{proposition}{两个子空间的直和}
    设$U$和$W$都是$V$的子空间,则$U+W$是直和当且仅当$U \cap W = \{ 0 \}$.
\end{proposition}
\begin{proof}
    \buzhou{1} 必要性:设$v \in (U \cap W)$,由于$0 = v + -v$,由命题\ref{pro:vihe}可知,$v = 0$. \\
    \buzhou{2} 充分性:假设有不为$0$的两个向量$u \in U,v \in W$,使得$0 = u + v$,那么$u = -v$.又因为$-v \in W$,可知$u \in v \in (U \cap W)$,于是$u=0$,这与假设矛盾.
\end{proof}



\section{有限维向量空间}

\subsection{张成空间}

首先介绍线性组合:

\begin{definition}{线性组合}
	对于$V$中的一组向量$v_1, \cdots ,v_m$,取$a_1, \cdots ,a_m \in \F$分别与每个元素相乘,就得到这组向量的\textit{线性组合},即$$a_1v_1 + \cdots + a_mv_m$$
	容易发现,一组向量的线性组合也是向量.
\end{definition}
\begin{remark}
	在描述一组向量时,为了避免出现歧义,通常不用括号括起来.这就类似于集合的表示中“$|$”与“$:$”的关系一样.
\end{remark}
\begin{remark}
	线性组合,实际上就是用来描述加法封闭性与标量乘法封闭性的.可以说,一个对加法、标量乘法封闭的集合中的任意元素都能被由所有元素构成的组的线性组合表示出来.
\end{remark}

\begin{example}
	请判断下列向量是否是$(2,1,-3),(1,-2,4)$的线性组合:
	$$(17,-4,2) \qquad (17,-4,5)$$
\end{example}

当这组向量的长度为$2$时,联系“平面向量基本定理”,可知若取平面上的两个基本的不共线向量$\xl{e_1},\xl{e_2}$,则平面上任意一个向量都能用这两个向量的线性组合表示.就像我们会用“所有满足$(x-a)^2+(y-b)^2=r^2$的点构成的集合”表示一个圆一样,所有能用这两个向量的线性组合表示的元素构成的集合是什么呢?

\begin{definition}{张成空间}
	$V$中的一组向量$v_1, \cdots v_m$的所有线性组合所构成的集合称为$v_1 , \cdots ,v_m$的\textit{张成空间},记为$\spn (v_1, \cdots ,v_m)$,即$$\spn (v_1, \cdots ,v_m) = \{ a_1v_1 + \cdots + a_mv_m : a_1 ,\cdots ,a_m \in \F \}$$
	特别地,定义空组$()$的张成空间为$\{ 0 \}$.
\end{definition}

\begin{example}
	前面的例子表明在$\F ^3$中,
	$$(17,-4,2) \in \spn ((2,1,-3),(1,-2,4))$$
	$$(17,-4,5) \notin \spn ((2,1,-3),(1,-2,4))$$
\end{example}

有了张成空间的定义,可知上文所述集合就是$\R ^2$,表示为$\R ^2 = \spn (\xl{e_1},\xl{e_2})$.

\begin{proposition}{张成空间是包含这组向量的最小子空间}
	$V$中一组向量的张成空间是包含这组向量的最小子空间.
\end{proposition}
\begin{proof}
	设$V$中向量组$v_1, \cdots ,v_m$的张成空间$\spn (v_1, \cdots ,v_m)$,记为$U$. \\
	\buzhou{1} 证明$U$是$V$的子空间:显然$0=0v_1 + \cdots + 0v_m \in U$;任取$U$中两个元素$u=a_1v_1 + \cdots a_mv_m$与$w=b_1v_1 + \cdots + b_mv_m$,作$u+w = (a_1+b_1)v_1 + \cdots + (a_m+b_m)v_m$,由$V$对加法封闭,可知$U$也对加法封闭;取$U$中一个元素$u=a_1v_1 + \cdots a_mv_m$与标量$\lambda \in \F$,由于$\lambda u = (\lambda a_1) v_1 + \cdots + (\lambda a_m)v_m$,由$V$对标量乘法封闭,可知$U$也对标量乘法封闭.综上,$U$是$V$的子空间.\\
    \buzhou{2} 证明$U$包含$v_1,\cdots ,v_m$:取$U$中元素$u_j$,令$u_j=0v_1 + \cdots + 1v_j + \cdots + 0v_m = v_j$,于是任意一个$v_j \in U$. \\
    \buzhou{3} 证明$U$是最小的满足条件的子空间:假设存在一个更小的$U'$,由于$U'$包含$v_1, \cdots ,v_m$,又因为$U'$对加法与标量乘法封闭,故$U'$中必有$v_1, \cdots ,v_m$的所有线性组合,即$|U'| \geq |\spn (v_1, \cdots ,v_m)| = |U|$,这与假设矛盾.因此$U$是最小的满足条件的子空间.
\end{proof}

\begin{definition}{张成}
	若$\spn (v_1, \cdots ,v_m) = V$,则称$v_1, \cdots ,v_m$\textit{张成}$V$.
\end{definition}

继续上文的例子.由于$\R ^2 = \spn (\xl{e_1},\xl{e_2})$,可知$\xl{e_1},\xl{e_2}$张成$\R ^2$.现在,取$\xl{e_1} = (1,0),\xl{e_2} = (0,1)$,则$\R ^2$中的任意一个向量均能表示为$a\xl{e_1} + b\xl{e_2} = (a,b)$的形式,这是一个标准的Cartesian坐标系.那如果$\xl{e_1},\xl{e_2}$只是两个普通的向量呢?可以构造出一种“平面非直角非单位长度坐标系”.总的来说,不论$\xl{e_1},\xl{e_2}$如何选取,它们总能作为两个“基底”张成$\R ^2$.更进一步,$\R ^{2}$中所有元素的自由度都是$2$(实际上这一点会在后面讲到,我们称能张成$V$的最小组的长度为$V$的维度).

\begin{example}
	请证明: \\
	(1)$\F ^{2}$上的向量组$(1,2),(3,5)$张成$\F ^{2}$. \\
	(2)$\F ^{2}$上的向量组$(1,2),(3,5),(6,7)$张成$\F ^{2}$. \\
	(3)$\F ^{n}$上的向量组$(1,0,\cdots ,0),(0,1,\cdots ,0),\cdots , (0,0, \cdots ,1)$张成$\F ^{n}$.(其中第$j$个向量的第$j$个坐标为$1$,其余都为$0$) \\
	(4)设$v_1,v_2,v_3,v_4$张成$V$,则$v_1-v_2,v_2-v_3,v_3-v_4,v_4$也张成$V$.
\end{example}
\begin{proof}
	只选择部分证明: \\
	(1)任取$\F ^2$上的向量$(x,y)$,由于$(3y-5x)(1,2)+(2x-y)(3,5)=(x,y)$,可知$\spn ((1,2),(3,5)) = \F ^{2}$,即$(1,2),(3,5)$张成$\F ^{2}$. \\
	(4)由于$v_1,v_2,v_3,v_4$张成$V$,任取$V$中元素$v$,设$$v=a_1v_1 + a_2v_2 + a_3v_3 + a_4v_4~(a_1,a_2,a_3,a_4 \in \F )$$
	因为$$v = a_1(v_1-v_2) + (a_1+a_2)(v_2-v_3) + (a_1+a_2+a_3)(v_3-v_4) + (a_1+a_2+a_3+a_4)v_4$$
	且由$\F$对加法封闭,$a_1,a_1+a_2,a_1+a_2+a_3,a_1+a_2+a_3+a_4 \in \F$,可知$v_1-v_2,v_2-v_3,v_3-v_4,v_4$也张成$V$.
\end{proof}
\begin{remark}
	从第二个例子可以看出,张成向量空间的组的长度不一定与$\R ^2$的维度相等.
\end{remark}

\subsection{有限维向量空间}

现在我们给出线性代数中的一个关键定义:

\begin{definition}{有限维向量空间,无限维向量空间}
	\begin{itemize}
		\item 如果一个向量空间可以由该空间中的某个向量组张成,则称这个向量空间是\textit{有限维的}.
		\item 相对应地,如果一个向量空间不是有限维的,则称这个向量空间是\textit{无限维的}.也就是说,如果一个向量空间不能由该空间中的任何向量组张成,它就是无限维的.
	\end{itemize}
\end{definition}

联系上一个例子中的第三条,由于$\F ^{n}$总能被这样一个向量组张成,它是有限维的.“维度”这个概念会在后面详细介绍,现在只是定性分析.

现在介绍一个具体的例子:

\begin{definition}{多项式,多项式的次数}
	\begin{itemize}
		\item 对于函数$p:\F \to \F$,若对任意$z \in \F$均存在$a_0, \cdots ,a_m \in \F$使得$$p(z) = a_0 + a_1z + a_2z^2 + \cdots a_mz^m$$
		则称$p$是系数属于$\F$的\textit{多项式}.
		\item 特别地,对于上式,当要求$a_m \neq 0$时,称$p$的\textit{次数}为$m$,记为$\deg p = m$.规定恒等于$0$的多项式的次数为$-\infty$.
		\item 定义$\mathcal{P} (\F)$是系数属于$\F$的所有多项式构成的集合.
		\item 对于非负整数$m$,定义$\mathcal{P}_{m} (\F)$表示系数在$\F$中且次数不超过$m$的所有多项式构成的集合.(约定$-\infty < m$).
	\end{itemize}
\end{definition}

\begin{example}
	请证明: \\
	(1)对每个非负整数$m$,$\mathcal{P} _{m} (\F)$是有限维向量空间. \\
	(2)$\mathcal{P} (\F)$是无限维向量空间.
\end{example}
\begin{proof}
	(1)由于$\mathcal{P} _{m} (\F) = \spn (1,z, \cdots ,z^m)$,可知$\mathcal{P} _{m} (\F)$是有限维向量空间. \\
	(2)假设$\mathcal{P} (\F)$中的一组多项式可以张成$\mathcal{P} (\F)$,记这组多项式中次数最高的多项式的次数为$m$,那么总能找到$z^{m+1}$不属于该张成空间,这与假设矛盾.故不存在任何一组多项式可以张成$\mathcal{P} (\F)$,即$\mathcal{P} (\F)$是无限维向量空间.
\end{proof}

\newpage
\section{线性无关}

\subsection{线性无关}

与子空间的和一样,我们倾向于研究那些有唯一表示形式的元素.

\begin{definition}{线性无关,线性相关}
	\begin{itemize}
		\item $V$中的一组向量$v_1, \cdots , v_m$称为\textit{线性无关},如果$\spn (v_1, \cdots ,v_m)$中每个向量可以唯一地表示成$v_1, \cdots ,v_m$的线性组合.规定空组$()$是线性无关的.
		\item 相对应地,如果一组向量不是线性无关的,则称这组向量\textit{线性相关}.也就是说,对于这组向量的张成空间,如果其中存在向量有不唯一的表示,它就是线性相关的.
	\end{itemize}
\end{definition}

\begin{proposition}{线性相关性的判定}
	\begin{itemize}
		\item $V$中一组向量$v_1, \cdots ,v_m$线性无关当且仅当使得$a_1v_1 + \cdots + a_mv_m = 0$成立的$a_1 , \cdots ,a_m \in \F$只有$a_1= \cdots =a_m =0$.
		\item 由线性相关的定义可知,$V$中一组向量$v_1, \cdots ,v_m$线性相关当且仅当存在不全为$0$的$a_1 , \cdots ,a_m \in \F$使得$a_1v_1 + \cdots + a_mv_m = 0$成立.
	\end{itemize}
\end{proposition}
\begin{proof}
	\buzhou{1} 充分性:假设$v \in \spn (v_1, \cdots , v_m)$有两种不同的线性组合表示,即
	$$v = a_1v_1 + \cdots + a_mv_m \qquad v = b_1v_1 + \cdots + b_mv_m$$
	两式相减,得到$0=(a_1-b_1)v_1 + \cdots + (a_m-b_m)v_m$.由所给条件,知$a_j=b_j ~(j=1,\cdots ,m)$,这与假设矛盾,于是$v$只有一种表示方法,即$v_1, \cdots ,v_m$线性无关. \\
	\buzhou{2} 必要性:首先,若令$a_1, \cdots , a_m$全为$0$,则有$0=0v_1 + \cdots + 0v_m$,这是$0$的一种表示形式;其次,由于$v_1, \cdots ,v_m$线性无关,$0$只有一种表示形式.综上,$0$的唯一表示形式就是$a_1= \cdots = a_m =0$.
\end{proof}

\begin{example}{\examplefont{线性无关的判断}}
	请证明: \\
	(1)$V$中一个向量$v$所构成的向量组是线性无关的当且仅当$v \neq 0$. \\
	(2)$V$中两个向量构成的向量组线性无关当且仅当每个向量都不能写成另一个向量的标量倍. \\
	(3)对每个非负整数$m$,$\mathcal{P} (\F)$中的组$1,z, \cdots ,z^m$线性无关. \\
	(4)设$v_1,v_2,v_3,v_4$在$V$中是线性无关的,则$v_1-v_2,v_2-v_3,v_3-v_4,v_4$也是线性无关的. \\
	(5)设$v_1, \cdots ,v_m$在$V$中线性无关,并设$w \in V$.证明:若$v_1+w , \cdots ,v_m+w$线性相关,则$w \in \spn (v_1 , \cdots ,v_m)$.
\end{example}
\begin{proof}
	(1)分别证明充分性和必要性的逆否命题成立,即证明“$v$构成的向量组线性相关当且仅当$v=0$”.充分性:当$v=0$,设$av=0~(a\in F )$,则存在不为$0$的$a$;必要性:设存在不为$0$的$a\in F$使$av=0$,则$v=0$. \\
	(2)设向量$u,v \in V$.分别证明充分性和必要性的逆否命题成立,即证明“$u,v$线性相关当且仅当每个向量可以写成另一个向量的标量倍”.充分性:记$u=\lambda v~(\lambda \neq 0)$,则存在不全为零的$a_1,a_2 \in \F $满足$a_1+a_2 \lambda =0$使$a_1v+a_2 u =0$成立;必要性:设存在不全为零的$a_1,a_2 \in \F $使$a_1 v + a_2 u=0$成立,不妨设$a_2 \neq 0$,则$u=-\dfrac{a_1}{a_2}v$. \\
	(3)首先不加证明地阐释一个引理:若一个多项式是零函数,则其所有系数均为$0$(会在第四章进行证明).于是,对于$p(z)=a_0+a_1z+ \cdots +a_mz^m=0$,必然有$a_0=a_1= \cdots =a_m$,即$1,z,\cdots ,z^m$线性无关. \\
	(4)由$v_1,v_2,v_3,v_4$线性无关,设\begin{equation}
		a_1v_1+a_2v_2+a_3v_3+a_4v_4=0 \label{202303251}
	\end{equation}
	其中$a_1,a_2,a_3,a_4 \in \F $.则必有$a_1=a_2=a_3=a_4=0$.对式\ref{202303251}进行变形,得到$$a_1(v_1-v_2)+(a_1+a_2)(v_2-v_3)+(a_1+a_2+a_3)(v_3-v_4)+(a_1+a_2+a_3+a_4)v_4=0$$
	由上可得此时$a_1=a_1+a_2=a_1+a_2+a_3=a_1+a_2+a_3+a_4=0$,即$v_1-v_2,v_2-v_3,v_3-v_4,v_4$线性无关. \\
	(5)设不全为$0$的$c_1,\cdots ,c_m \in \F $满足$$c_1(v_1+w)+ \cdots + c_m(v_m+w)=0$$
	即$$c_1v_1+ \cdots + c_mv_m = -(c_1+ \cdots + c_m)w$$
	由$v_1,\cdots ,v_m$线性无关,左式一定不为$0$.当$w=0$时,必然可以表示为$v_1,\cdots ,v_m$的线性组合形式,命题成立;当$w \neq 0$时,$c_1+ \cdots + c_m \neq 0$,故$$w=\frac{-c_1}{c_1+ \cdots + c_m}v_1 + \cdots + \frac{-c_m}{c_1+ \cdots + c_m}v_m$$
	则命题成立.
\end{proof}

\begin{example}{\examplefont{线性相关的判断}}
	请证明: \\
	(1)$\F ^{3}$中的向量组$(2,3,1),(1,-1,2),(7,3,8)$线性相关. \\
	(2)$\F ^{3}$中的向量组$(2,3,1),(1,-1,2),(7,3,c)$线性相关当且仅当$c=8$. \\
	(3)包含$0$向量的向量组线性相关.
\end{example}

\begin{proof}
	(1)设$a_1,a_2,a_3 \in \F$满足$$a_1(2,3,1)+a_2(1,-1,2)+a_3(7,3,8)=(0,0,0)$$
	即$$(2a_1+a_2+7a_3,3a_1-a_2+3a_3,a_1+2a_2+8a_3)=(0,0,0)$$
	可得$$\begin{cases}
		2a_1+a_2+7a_3=0 \\
		3a_1-a_2+3a_3=0 \\
		a_1+2a_2+8a_3=0
	\end{cases}$$
	化简之,得到任意满足$a_1=-2a_3,a_2=-3a_3$的$(a_1,a_2,a_3)$均符合要求,则存在一组不全为$0$的$(a_1,a_2,a_3)$满足上式,即$(2,3,1),(1,-1,2),(7,3,8)$线性相关. \\
	(实际上,将方程组中的第一个式子乘以$\dfrac{7}{5}$再与第二个式子乘以$-\dfrac{3}{5}$相加,就得到了第三个式子.也就是说,这三个式子中有一个式子是多余的,自然可以解出不全为$0$的$a_1,a_2,a_3$.) \\
	(2)充分性同上,下证必要性:设$a_1,a_2,a_3 \in \F$满足$$a_1(2,3,1)+a_2(1,-1,2)+a_3(7,3,c)=(0,0,0)$$
	即$$\begin{cases}
		2a_1+a_2+7a_3=0 \\
		3a_1-a_2+3a_3=0 \\
		a_1+2a_2+ca_3=0
	\end{cases}$$
	容易发现,$(a_1,a_2,a_3)=(0,0,0)$是方程组的一组解.要得到另一组不同的解,要求有效方程的个数严格小于变量个数,即其中一个方程可以表示为另两个的线性组合形式,记$$a_1+2a_2+ca_3=x(2a_1+a_2+7a_3)+y(3a_1-a_2+3a_3)~(x,y \in \F )$$
	化简之,即$$(2x+3y-1)a_1+(x-y-2)a_2+(7x+3y-c)a_3=0$$
	对任意$a_1,a_2,a_3$均成立,即$2x+3y-1=x-y-2=7x+3y-c=0$,解得$x=\dfrac{7}{5},y=-\dfrac{3}{5}$,于是$c=8$.
\end{proof}

线性相关与下列定义等价:

\begin{proposition}{线性相关的第二定义}
	$V$中一组向量$v_1, \cdots ,v_m$线性相关当且仅当其中存在一个向量能表示为其余向量的线性组合形式.
\end{proposition}
\begin{proof}
	\buzhou{1} 充分性:设该向量$v$能表示为$v_1, \cdots ,v_m$的线性组合形式,即$$v= a_1v_1 + \cdots + a_mv_m$$
	那么$0=a_1v_1 + \cdots + a_mv_m + (-1)v$.其中$-1$显然不为$0$,因此$v_1, \cdots ,v_m,v$线性相关. \\
	\buzhou{2} 必要性:设$0=a_1v_1 + \cdots + a_mv_m$.不妨令$a_j \neq 0$,那么有$$v_j = \frac{a_1}{-a_j} v_1 + \cdots + \frac{a_m}{-a_j} v_m$$
	这说明$v_j$可以表示为其余元素的线性组合.
\end{proof}
\begin{remark}
	在该证明过程中,不难发现定义里“其余”的重要性.
\end{remark}

实际上,利用这个定义更好理解线性相关的本质.上一小节的例题告诉我们,张成组(即张成某向量空间的向量组)的长度可以不同.容易证明,第一个例子中的向量组是线性无关的,而第二个例子中的向量组是线性相关的.实际上,像这样既是张成组又是线性无关的组,就称为基(详细内容在下一小节会讲到).


\subsection{线性相关性与张成}

下面的命题为我们阐释了线性相关性与张成的一个基本关系.

\begin{proposition}{线性相关性引理}
	设$v_1, \cdots ,v_m$是$V$中的一个线性相关的向量组,则存在$j \in \{ 1,2, \cdots ,m \}$使得: \\
	(a)$v_j \in \spn (v_1, \cdots , v_{j-1})$; \\
	(b)若从$v_1, \cdots ,v_m$中去掉第$j$项,则剩余组的张成空间等于$\spn (v_1, \cdots ,v_m)$.
\end{proposition}
\begin{proof}
	(a)由于$v_1, \cdots ,v_m$线性相关,存在不全为$0$的数$a_1, \cdots ,a_m \in \F$使得$a_1v_1 + \cdots + a_mv_m = 0$.(人为地)设该向量组的顺序满足$a_1, \cdots ,a_j$均不为$0$,从而有
	\begin{equation}
		v_j = \frac{a_1}{-a_j} v_1 + \cdots + \frac{a_{j-1}}{-a_j} v_{j-1} \label{xmxkxlgr}
	\end{equation}
	这意味着$v_j \in \spn (v_1, \cdots , v_{j-1})$; \\
	(b)取$\spn (v_1, \cdots ,v_m)$中某一元素$u$,设$u=b_1v_1 + \cdots + b_mv_m$,将式\ref{xmxkxlgr}代入可得
	$$u = \ssb{\frac{a_1b_j}{-a_j}+b_1}v_1 + \cdots + \ssb{\frac{a_{j-1}b_j}{-a_j}+b_{j-1}}v_{j-1} + b_{j+1} v_{j+1} + \cdots b_mv_m$$
	这表明对于$\spn (v_1, \cdots ,v_m)$中任一元素,它都在$\spn (v_1, \cdots ,v_{j-1} , v_{j+1}, \cdots ,v_m)$中,即原命题所述.
\end{proof}

由线性无关与张成的几何意义,我们能够想象:对于任意一个有限维向量空间,总是存在一组“基底”,这组基底可以线性表示任何向量空间中的元素,并且它们之间互不多余、缺一不可.这就类似于欧氏几何中的五条公理一样.通过这种直观的理解,不难得出以下命题,难的在于如何规整地证明.

\begin{proposition}{线性无关组与张成组长度的关系}{xxwgvi}
	在有限维向量空间$V$中,线性无关组的长度总是小于等于向量空间的每一个张成组的长度.
\end{proposition}
\begin{proof}
	设$V$中一个线性无关组$u_1, \cdots ,u_m$与张成组$w_1, \cdots w_n$. \\
	\buzhou{1}第$1$步:将线性无关组中的第$1$个元素$u_1$添加在张成组的开头,便形成组$$u_1,w_1, \cdots ,w_n$$
	由线性相关性引理,我们可以去掉某个$w$使得新的组仍张成$V$. \\
	\buzhou{2}第$j$步:将线性无关组中的第$j$个元素$u_j$添加在$u_{j-1}$后,由线性相关性引理,又因为$u_1, \cdots ,u_j$是线性无关的,我们可以去掉某个$w$使得新的组仍张成$V$. \\
	每经过一步,都会将组中的一个$w$换成一个$u$.因为在第$m$步后把所有的$u$都换完,可知$n \geq m$,即原命题所述.
\end{proof}

利用这一“直观”的结论,我们可以“直观”地证伪某些命题.

\begin{example}
	证明下列结论: \\
	(1)组$(1,2,3),(4,5,8),(9,6,7),(-3,2,8)$在$\R ^{3}$中一定不是线性无关的. \\
	(2)组$(1,2,3,-5),(4,5,8,3),(9,6,7,-1)$一定不能张成$\R ^{4}$.
\end{example}

利用命题\ref{pro:xxwgvi}的证明思路,还可以说明更多直观的结论:

\begin{proposition}{向量空间中的一些结论}{yixpjply}
	\begin{itemize}
		\item 在向量空间$V$中,每个线性相关的张成组都能通过去除某些元素得到一个线性无关的张成组.
		\item $V$是无限维向量空间当且仅当$V$中存在一个向量序列$v_1, v_2, \cdots$使得当$m$是任意正整数时$v_1, \cdots ,v_m$都是线性无关的.
		\item 有限维向量空间的子空间都是有限维的.
	\end{itemize}
\end{proposition}
\begin{proof}
	(1)\buzhou{1} 第$1$步:设$\mathcal{W}_1 = v_1, \cdots ,v_m$张成$V$.若$v_1 \notin \spn (v_2, \cdots ,v_m)$,则保持该组不变,并停止操作;若$v_1 \in \spn (v_2, \cdots ,v_m)$,则去掉$v_1$,并记新组$v_2, \cdots , v_m$为$\mathcal{W}_2$. \\
	\buzhou{2} 第$j$步:若$v_j \notin \spn (v_{j+1}, \cdots ,v_m)$,则保持$\mathcal{W}_j$不变,并停止操作;若$v_j \in \spn (v_{j+1}, \cdots ,v_m)$,则去掉$v_j$,并记新组$v_{j+1}, \cdots , v_m$为$\mathcal{W}_{j+1}$.在经过有限次操作后,一定会在某一步停止并返回一个线性无关的组,且能张成$V$. \\
	(2)充分性显然.下证必要性: \\
	\buzhou{1} 第$1$步:取$V$中的一个线性无关向量组$\mathcal{W}_1$,作它的张成空间$U_1$,取一元素$u \in (V \setjianfa U_1)$放入该组,得到一个新的组$\mathcal{W}_2$.显然该组仍是线性无关的(因为线性相关性的第二定义). \\
	\buzhou{2} 第$j$步:作$\mathcal{W}_j$的张成空间$U_j$,取一元素$u \in (V \setjianfa U_j)$放入$\mathcal{W}_j$,得到一个新的组$\mathcal{W}_{j+1}$. \\
	由于可以不断重复该过程,因此这样一个组$\mathcal{W}_j$会不断扩张并保持线性无关,即符合原命题要求. \\
	(3)设有限维向量空间$V$及其子空间$U$.由命题\ref{pro:xxwgvi},$V$中任意一个线性无关组的长度小于等于每一个$V$的张成组的长度,故该线性无关组的长度是有限的. \\
	\buzhou{1} 第$1$步:若$U=\{ 0 \}$,即$U = \spn ()$,则$U$符合要求;否则存在非零向量$v_1 \in U$. \\
	\buzhou{2} 第$j$步:若$U= \spn (v_1,\cdots ,v_{j-1})$,则$U$符合要求;否则存在$v_j \in U$满足$v_j \notin \spn (v_1,\cdots ,v_{j-1})$. \\
	由于第$j+1$步能够说明存在$v_1, \cdots , v_j \in U$使$v_1, \cdots , v_j$线性无关,而$U$中任意一个线性无关组的长度是有限的,故一定会在某一步停止,此时$U$即符合要求.
\end{proof}

\newpage
\section{基与维数}

\subsection{基}

上一节中多次出现“基底”这一关键词,现在我们来集中研究它:

\begin{definition}{基}
	若$V$中的一个向量组既线性无关又张成$V$,则称为$V$的\textit{基}.
\end{definition}

\begin{example}{\examplefont{基的例子}}
	请验证: \\
	(1)组$(1,0,\cdots ,0),(0,1,0,\cdots ,0), \cdots ,(0,\cdots ,0,1)$是$\F ^{n}$的基.(实际上,这称为$\F ^{n}$的\textit{标准基}) \\
	(2)组$(1,1,0),(0,0,1)$是$\{ (x,x,y) \in \F ^{3}:x,y \in \F \}$的基. \\
	(3)组$(1,-1,0),(1,0,-1)$是$\{ (x,y,z) \in \F ^{3}:x+y+z=0 \}$的基. \\
	(4)组$1,z, \cdots ,z^{m}$是$\mathcal{P}_m (\F)$的基.
\end{example}

我们发现张成和线性无关的定义十分类似:都出现了“线性组合”这一形式.将它们综合起来,就是基的判定命题:

\begin{proposition}{基的判定}
	$V$中的向量组$v_1, \cdots ,v_m$是$V$的基当且仅当每个$v \in V$都能唯一地写成以下形式$$v = a_1v_1 + \cdots + a_mv_m$$
	其中$a_1, \cdots ,a_m \in \F$.
\end{proposition}
\begin{proof}
	必要性显然.直接来看充分性:在$V$中任取一元素$v$,设它可以唯一地表示为$v = a_1v_1 + \cdots + a_mv_m$的形式. \\
	\buzhou{1} 张成:由张成的定义可知,$v_1, \cdots ,v_m$张成$V$. \\
	\buzhou{2} 线性无关:令$a_1, \cdots , a_m$全为$0$,则有$0=0v_1 + \cdots + 0v_m$,这是$0$的唯一表示形式,因此$v_1, \cdots ,v_m$线性无关.
\end{proof}

回顾上一节中命题\ref{pro:yixpjply}的第一条,实际上现在我们就能将其写成基的形式.

\begin{proposition}{基、线性无关组、张成组I}{kois}
	在有限维向量空间$V$中,
	\begin{itemize}
		\item 每个张成组都可以化简成$V$的一个基.
		\item 每个线性无关的向量组都可以扩充成$V$的一个基.
	\end{itemize}
\end{proposition}
\begin{proof}
	第一条已经在命题\ref{pro:yixpjply}中证明过,这里只证明第二条. \\
	设$V$中的线性无关组$v_1, \cdots ,v_m$与一个基$u_1, \cdots ,u_n$.作组$\mathcal{W} = v_1, \cdots ,v_m,u_1, \cdots ,u_n$,它显然张成$V$.由命题\ref{pro:xxwgvi}可知$m \leq n$.利用第一条的证明过程将$\mathcal{W}$化简为$v_1, \cdots , v_m , u_1, \cdots ,u_j$,可知它是$V$的一个基.
\end{proof}

以上的命题具有很强的可操作性.例如,取$\F ^{3}$的基的一部分(也就是一个线性无关组)$(1,1,4),(5,1,4)$,再取一个标准基$(1,0,0),(0,1,0),(0,0,1)$.在$(1,1,4),(5,1,4),(1,0,0),(0,1,0),(0,0,1)$中去掉$$(1,0,0)=-\frac{1}{4}(1,1,4)+\frac{1}{4}(5,1,4)$$
在$(1,1,4),(5,1,4),(0,1,0),(0,0,1)$中,由于$(0,1,0),(0,0,1)$都不能被$(1,1,4),(5,1,4)$线性表示,所以最后可以保留其中任意一个,即$(1,1,4),(5,1,4),(0,1,0)$和$(1,1,4),(5,1,4),(0,0,1)$都是$\F ^{3}$的基.

有了基这个工具之后,我们可以证明更多之前不能证明的结论:

\begin{proposition}{子空间与直和的关系}
	设$V$是有限维的,$U$是$V$的子空间,则存在$V$的子空间$W$使得$V=U \oplus W$.
\end{proposition}
\begin{proof}
	设$U$的一个基$u_1, \cdots , u_m$,按照命题\ref{pro:kois}的方法将这个$V$中的线性无关组扩充为$V$的基,记为$u_1, \cdots ,u_m,w_1, \cdots ,w_n$.取$W = \spn (w_1, \cdots ,w_n)$,下证这样的$W$就是满足题目要求的子空间: \\
	显然$U+W=V$.任取$v \in (U \cap W)$,设$$v=a_1u_1 + \cdots + a_mu
	_m \qquad v = b_1w_1 + \cdots + b_nw_n$$
	两式作差,得$0=a_1u_1 + \cdots a_mu_m + (-b_1)w_1 + \cdots + (-b_n)w_n$,由$u_1, \cdots ,u_m,w_1, \cdots ,w_n$是$V$的基可得$a_1 = \cdots = b_n = 0$,则$v=0$.由命题\ref{pro:ziksjmvihe}知$V=U \oplus W$.
\end{proof}

\subsection{维数}

继续研究向量空间的几何意义,我们发现基已经被定义了,但是最小的基还不太清楚.实际上,容易说明所有基的长度都是相等的,而任意一个基的长度就称作\textit{维数}.

\begin{proposition}{基的长度不依赖于基的选取}
	有限维向量空间的任意两个基的长度都相同.
\end{proposition}
\begin{proof}
	由命题\ref{pro:xxwgvi}可知,因为任意两个基$\mathcal{U},\mathcal{V}$都同时是线性无关向量组与张成组,所以$\mathcal{U}$的长度小于等于$\mathcal{V}$的长度、$\mathcal{U}$的长度大于等于$\mathcal{V}$的长度,于是它们的长度相等.
\end{proof}

\begin{definition}{维数}
	有限维向量空间$V$的任意基的长度称为这个向量空间的\textit{维数},记作$\dim V$.
\end{definition}

很明显,一个有限维向量空间的子空间也是有限维的,它的维数应当满足下列命题要求:

\begin{proposition}{子空间的维数}
	若$U$是有限维向量空间$V$的子空间,则$\dim U \leq \dim V$.
\end{proposition}
\begin{proof}
	取$U$的一个基$\mathcal{U}$,取$V$的一个基$\mathcal{V}$.因为$\mathcal{U}$是$V$的一个线性无关的子空间,由命题\ref{pro:xxwgvi}可知,$\mathcal{U}$的长度小于等于$\mathcal{V}$的长度,即$\dim U \leq \dim V$.
\end{proof}

借助维数,可以更快捷地证明一个向量空间的基.

\begin{proposition}{基、线性无关组、张成组II}
	设$V$是有限维向量空间.
	\begin{itemize}
		\item $V$中每个长度为$\dim V$的线性无关向量组都是$V$的基.
		\item $V$中每个长度为$\dim V$的张成组都是$V$的基.
	\end{itemize}
\end{proposition}
\begin{proof}
	(1)设$V$中的一个线性无关向量组$v_1, \cdots ,v_m$,取$V$中一个基$w_1 , \cdots , w_n$,由命题\ref{pro:kois}可知,$v_1, \cdots ,v_m$可以扩充为基,然而在此过程中由于$m=n$,实际上没有发生任何扩充,故$v_1, \cdots ,v_m$本来就是一个基. \\
	(2)证明过程同(1),留作习题.
\end{proof}

\begin{example}
	证明以下结论: \\
	(1)组$(5,7),(4,3)$是$\F ^{2}$的基. \\
	(2)证明$1,(x-5)^2,(x-5)^3$是$\mathcal{P}_{3} (\R)$的子空间$U$的一个基,其中$U$定义为$$U = \{ p \in \mathcal{P}_{3} (\R) : p'(5)=0 \}$$
\end{example}

类比并集的元素个数计算公式(即容斥原理),子空间和的的维数计算公式如下:

\begin{proposition}{子空间和的维数}
	如果$U_1$和$U_2$是有限维向量空间的两个子空间,则$$\dim (U_1+U_2) = \dim U_1 + \dim U_2 - \dim (U_1 \cap U_2)$$
\end{proposition}
\begin{proof}
	设$U_1 \cap U_2$的基$v_1, \cdots ,v_m$,则$v_1, \cdots ,v_m \in U_1,U_2$;设$U_1, U_2$的基分别为$v_1, \cdots ,v_m,u_1, \cdots ,u_j$与$v_1, \cdots ,v_m,u'_1, \cdots ,u'_k$. \\
	取$U_1+U_2$中的$0$元素,它能被表示为$w_1+w_2$的形式(其中$w_1 \in U_1,w_2 \in U_2$).因此设
	\begin{align*}
		0 &= w_1 + w_2 = (a_1v_1 + \cdots a_mv_m + b_1u_1 + \cdots + b_ju_j) + (a'_1v_1 + \cdots a'_mv_m + b'_1u'_1 + \cdots + b'_ku'_k) \\
		&= (a_1+a'_1)v_1 + \cdots + (a_m + a'_m)v_m + b_1u_1 + \cdots + b_ju_j + b'_1u'_1 + \cdots + b'_ku'_k
	\end{align*}
	于是
	$$ (-a_1-a'_1)v_1 + \cdots + (-a_m - a'_m)v_m + (-b_1)u_1 + \cdots + (-b_j)u_j = b'_1u'_1 + \cdots + b'_ku'_k $$
	等式左边属于$U_1$,等式右边属于$U_2$,因此$b'_1u'_1 + \cdots + b'_ku'_k \in U_1 \cap U_2$.设
	$$b'_1u'_1 + \cdots + b'_ku'_k = c_1v_1 + \cdots + c_mv_m$$
	于是$$b'_1u'_1 + \cdots + b'_ku'_k + (-c_1)v_1 + \cdots + (-c_m)v_m = 0$$
	这个式子告诉我们$b'_1 = \cdots = b'_k = c_1 = \cdots c_m =0$.同理可得$b_1 = \cdots = b_j =0$.带入上式,可得
	$$(a_1+a'_1)v_1 + \cdots + (a_m + a'_m)v_m = 0$$
	所以$a_1+a'_1 = \cdots = a_m + a'_m =0$. \\
	综上,对于$U_1+U_2$中的$0$,它只有唯一一种线性表示形式,即满足$$a_1+a'_1 = \cdots = a_m + a'_m = b'_1 = \cdots = b'_k = b_1 = \cdots = b_j$$
	故$v_1, \cdots ,v_m,u_1, \cdots ,u_j,u'_1, \cdots ,u'_k$是$U_1+U_2$的基.于是$\dim (U_1+U_2)=m+j+k = \dim U_1 + \dim U_2 - \dim (U_1 \cap U_2)$.
\end{proof}
\begin{remark}
	注意子空间的维数计算公式不一定能推广到更多元的情况,例如本节习题中所示.
\end{remark}

% 线性映射与矩阵

\include{MainMatter/linear_maps_and_matrices.tex}

% 数论, 群论

\include{MainMatter/group_theory.tex}

% 多项式, 环

\chapter{多项式与环}

% 域扩张

% 模论

\part{DLC1-范畴论}


\part{DLC2-线性代数}

% 线性方程组与行列式计算

\include{MainMatter/groups_of_linear_equations_and_deteminant.tex}

% 对角化

\include{MainMatter/diagonalization.tex}

% 多重线性代数, 再论行列式

% 内积空间









% Appendices section.
% \appendix

% Include the "about" appendix from the BackMatter subfolder.
% \include{./BackMatter/about}

% Include the "abbreviation" appendix from the BackMatter subfolder.
% \include{./BackMatter/abbreviation}

% Include the "notation" appendix from the BackMatter subfolder.
% \include{./BackMatter/notation}

% Include the "code" appendix from the BackMatter subfolder.
% \include{./BackMatter/code}

% Include the "glossary" appendix from the BackMatter subfolder.
% \include{./BackMatter/glossary}

% Include the "index" appendix from the BackMatter subfolder.
% \include{./BackMatter/index}

% Include the "references" appendix from the BackMatter subfolder.
% 
\bibliographystyle{plain}
\bibliography{references.bib}

%\nocite{*}

% End the document.
\end{document}
