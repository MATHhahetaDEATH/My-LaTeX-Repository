\chapter{Riemann积分}

\section{Riemann可积的定义}

我们先来看一个例子: 尝试计算(定义)抛物线$y=x^2,0\leq x \leq 1$与$x$轴围成的面积. 一种自然的思路是利用矩形进行逼近, 因为我们可以简单地定义出矩形的面积. 具体来说, 考虑$[0,1]$区间的一个分割$\pi = \{ 0=x_0<x_1<\cdots <x_{n-1}<x_n=1 \}$, 并在每个小区间内部取点$\xi _i \in [x_{i-1},x_i]$, 记$\xi = \{ \xi _1,\cdots ,\xi _n \}$. 我们希望定义如下的极限就是所求面积: $$\lim_{\| \pi \| \to 0} \sum_{i=1}^{n}(x_i-x_{i-1})\xi _i^2. $$
其中, 极限过程$\| \pi \| \to 0$的意思是$\max_{1\leq i \leq n} \{x_i-x_{i-1}\} \to 0$(这的确构成一种极限过程, 后面会证明). 

这种定义有非常明显的问题: 
\begin{itemize}
	\item 分割$\pi$和标记点$\xi$并不确定, 因而无法很方便地计算这个极限. 
	\item 由于不便将该式化简, 我们甚至不知道这样的极限是否存在. 
\end{itemize}
针对这两个问题, 可能会有如下想法: 
\begin{itemize}
	\item 用平均分割和统一的标记点选取标准来代替一般化的方法. 注意这种计算方式一定是充分条件. 
	\item 类似于数列, 既然难以判断极限是否存在, 可以考虑定义上下极限来逼近. 
\end{itemize}
综合这两种想法, 在本例中我们可以如下考察所求的面积: 首先取平均分割$\pi_{n}=\{ x_i=\frac{i}{n} \}$, 接着选择两种不同的标记点集$\displaystyle\xi _L = \big\{ \xi _i :  \xi _i ^2= \inf_{x_{i-1} \leq x \leq x_i} x^2 \big\}$, $\displaystyle\xi _U = \big\{ \xi _i : \xi _i ^2= \sup_{x_{i-1} \leq x \leq x_i} x^2 \big\}$. 考虑极限过程$n\to \infty$, 于是两种极限$S_L$和$S_U$就是: $$S_L = \lim_{n\to \infty} \left( \frac{1}{n} \sum_{i=1}^{n} \frac{(i-1)^2}{n^2} \right) = \lim_{n\to \infty} \frac{2n^2-3n+1}{6n^2} = \frac{1}{3} ,  $$
$$S_U = \lim_{n\to \infty} \left( \frac{1}{n} \sum_{i=1}^{n} \frac{i^2}{n^2} \right) = \lim_{n\to \infty} \frac{2n^2+3n+1}{6n^2} = \frac{1}{3}.$$
注意到$S_L=S_U = \frac{1}{3}$, 于是所求面积(似乎)就是$\frac{1}{3}$. 

上面的例子就是利用Riemann和进行逼近计算(定义)的做法. 我们不难想到另一种方法: 先定义所谓阶梯函数(这种函数的积分易于定义), 再考虑找到特定函数的阶梯函数逼近, 最后计算极限阶梯函数的积分即可. 这两种方法本质上是相同的, 只是操作顺序不一样. 然而利用后者有两个优势: 我们能够自然地理解Riemann积分的某些缺陷(例如只能处理有界函数), 另外实际上Riemann和的定义无法推广到$\R ^n$上(依赖序关系). 


\subsection{阶梯函数及其积分}

首先我们考虑\textit{示性函数}(characteristic function): 设$E \subseteq \R$, 则定义$$\chi _{E}:\R \to \{ 0,1 \},\quad \chi _{E} (x) = \begin{cases}
	1 & x \in E \\ 0 & x \notin E
\end{cases}.$$
我们希望定义$\chi _E$的积分, 而这自然需要定义集合$E$的长度. 我们不希望引入繁杂的测度理论, 于是承认下方的“公理”: 对于某个将集合映射到其对应长度的函数$\mu$, 我们希望有$$\mu ((a,b)) = \mu ([a,b]) = \mu ([a,b)) = \mu ((a,b]) = b-a, $$
$$\mu ((-\infty ,+\infty)) = \mu ((-\infty ,b)) = \mu ((a ,+\infty)) = +\infty .$$
第一行的“公理”也在告诉我们: 像$a,b$这样的点是没有长度的. 

于是, 我们可以定义示性函数的\textit{积分}(integral)为$$\int \chi _E := \mu (E). $$
特别地, $$\int \chi _{(a,b)} = \int \chi _{[a,b]} = \int \chi _{[a,b)} = \int \chi _{(a,b]} = b-a. $$
当然, 实际定义阶梯函数的积分时只需要用到$E$是区间的情况, 因此上方不严谨的思考并不影响结果的正确性. 

接下来看\textit{阶梯函数}(更多时候称为\textit{简单函数}, simple function): 

\begin{definition}{分划, 阶梯函数, 阶梯函数的积分}
	设闭区间$I=[a,b]$. 
	\begin{itemize}
		\item 称形如$$\pi = \{ a=x_0<x_1<\cdots <x_{n-1}<x_n=b \}$$
		的集合为$I$的一个\textit{分划}(partition). 
		\item 定义分划$\pi$的\textit{步长}(step): $$\| \pi \| := \max_{1 \leq i \leq n} \{ x_i-x_{i-1} \}. $$
		\item 对于分划$\pi= \{ a=x_0<\cdots <x_n=b \}$, 称形如$$f = \sum_{i=1}^{n} \lambda _i \chi _{(x_{i-1},x_{i})} $$
		(其中$\lambda _i \in \R$)的函数为$\pi$上的一个\textit{阶梯函数}. 记$I$上全体阶梯函数的集合为$\mathcal{S}(I)$. 
		\item 承上一条, 定义$f$的\textit{积分}(integral)为$$\int f := \sum_{i=1}^{n} \lambda _i \int \chi _{(x_{i-1},x_i)} = \sum_{i=1}^{n} \lambda _i (x_i-x_{i-1}). $$
	\end{itemize}
\end{definition}
\begin{remark}
	刚才说过, 点是没有长度的, 因此阶梯函数在分割点处的取值应当对最终结果没有影响. 
\end{remark}
\begin{remark}
	为了避免与原函数的符号相混淆, 也记$\int f = \int _I f = \int_a^b f$. 不过为了方便, 我们暂时混用这些符号. 
\end{remark}

这里, 我们实际上需要验证阶梯函数积分的良定义性. 也就是说, 如果$\pi, \pi '$均与$f$相容, 则$f$依这两种分划的积分相等. 一般地, 我们称$\pi ' \prec \pi$($\pi '$比$\pi$细), 如果$\pi$的分割点都是$\pi '$的分割点. 于是我们只需要验证, 若$\pi ' \prec \pi$, 则$f$依这两种分划的积分相等(这是因为我们可以取$\overline{\pi} = \pi \cup \pi '$). 而这是显然的. 

接下来研究积分算子的性质: 

\begin{proposition}{阶梯函数积分算子的性质}
	设闭区间$I=[a,b]$. 设$f,g \in \mathcal{S}(I)$. 则$\int : \mathcal{S}(I) \to \R$满足: 
	
	a) 线性: $$\int (\lambda f+ \mu g) = \lambda \int f + \mu \int g.$$
	
	b) 区间可加性: 设$I=[a,c] \cup [c,b]$, 则$$\int_a^b f = \int_a^c f + \int_c^b f. $$
	
	c) 三角不等式: $$\left| \int f \right| \leq \int |f|. $$
	
	d) 保号性/保序性: 若$f \geq 0$, 则$$\int f \geq 0. $$
	\qquad 特别地, 若$f \leq g$, 则$$ \int f \leq  \int g.$$
	
	e) 对积分的估计: 选取与$f$相容的分划$\pi= \{ a=x_0<\cdots <x_n=b \}$后, $$\left| \int_a^b f \right| \leq |b-a| \sup_{1 \leq i \leq n} |\lambda _i|. $$
\end{proposition}
\begin{proof}
	注意到阶梯函数是有限求和, 诸命题的验证就是显而易见的了. 
\end{proof}

\subsection{Riemann可积的定义}

现在来对一般函数定义Riemann积分. 首先是从阶梯函数角度出发: 

\begin{definition}{Riemann可积}
	设闭区间$I=[a,b]$. 称$f$是\textit{Riemann可积的}, 如果对任意$\varepsilon >0$均存在$F,e \in \mathcal{S}(I)$使得$$\forall x \in I,|f(x)-F(x)|<e(x),\qquad \int e < \varepsilon .$$
	全体$I$上Riemann可积的函数构成集合$\mathcal{R}(I)$. 
\end{definition}


我们需要对上述定义中的$F$逐个求积分并取其“极限”. 从上方的定义, 我们不难想到类似于序列逼近函数极限的思路: 考虑某种意义上的一列逼近函数$\{ f_n \}$, 然后定义$$\int f := \lim_{n\to \infty} \int f_n.$$

\begin{proposition}{}
	设闭区间$I=[a,b]$. 则下列说法等价: 
	\begin{itemize}
		\item $f$是Riemann可积的. 
		\item 存在阶梯函数序列$\{ f_n \},\{ e_n \}$使得$$\forall x \in I,|f(x)-f_n(x)|<e_n(x),\qquad \lim_{n\to \infty} \int e_n = 0.$$
		并且令$$\int f := \lim_{n\to \infty} \int f_n .$$
	\end{itemize}
\end{proposition}
\begin{proof}
	1) $\Rightarrow$ 2): 对任意$n>0$, 存在$f_n$使得$|f(x)-f_n(x)|<e_n(x)$使得$\int e_n < 1/n$, 从而$\lim_{n\to \infty} \int e_n = 0$. 

	2) $\Rightarrow$ 1): 显然. 

	另外, 我们还需证明$\int f = \lim_{n\to \infty} \int f_n$是良定义的, 即极限存在且不依赖于$\{ f_n \}$的选取: 

	由于$|f_n(x)-f_m(x)| \leq |f(x)-f_n(x)| + |f(x)-f_m(x)| \leq e_n(x)+e_m(x)$, 在两边积分可得$$\left| \int f_n-\int f_m \right| \leq \int |f_n-f_m| \leq \int e_n + \int e_m \to 0.$$
	这说明$\{ \int f_n \}$是Cauchy列, 进而存在极限. 

	另一方面, 设有另外的$\{ f'_n \}$和$\{ e'_n \}$符合要求, 则$|f_n(x)-f'_n(x)| \leq |f(x)-f_n(x)| + |f(x)-f'_n(x)| \leq e_n(x)+e'_n(x)$, 两侧积分可得$$\left| \int f_n - \int f'_n \right| \leq \int |f_n-f'_n| \leq \int e_n+ \int e'_n \to 0.$$
	于是两个极限相等. 
\end{proof}


类似于阶梯函数积分算子的性质, 对一般的积分算子也有如下性质: 

\begin{proposition}{} \label{pro:jiffxkvi1}
	设闭区间$I=[a,b]$. 设$f,g \in \mathcal{R}(I)$. 则$\int : \mathcal{R}(I) \to \R$满足: 
	
	a) 线性: $$\int (\lambda f+ \mu g) = \lambda \int f + \mu \int g.$$
	
	b) 区间可加性: 设$I=[a,c] \cup [c,b]$, 则$$\int_a^b f = \int_a^c f + \int_c^b f. $$
	
	c) 三角不等式(注意这依赖于$|f|$的可积性, 当然证明是显然的): $$\left| \int f \right| \leq \int |f|. $$
	
	d) 保号性/保序性: 若$f \geq 0$, 则$$\int f \geq 0. $$
	\qquad 特别地, 若$f \leq g$, 则$$ \int f \leq  \int g.$$
	
	e) 对积分的估计: 选取与$f$相容的分划$\pi= \{ a=x_0<\cdots <x_n=b \}$后, $$\left| \int_a^b f \right| \leq |b-a| \sup_{a \leq x \leq b} |f(x)|. $$
\end{proposition}
\begin{remark}
	尽管无法给出$\int fg$关于$\int f,\int g$的表达式, 我们仍然希望证明$fg$是可积的. 然而, 正是由于$\int fg$不好表示, 这里无法证明. 在下一节我们会处理这个问题. (实际上用阶梯函数的确能处理, 只不过太麻烦)
\end{remark}

接着从Riemann和角度出发. 类似于前文的例子, 如下定义: 基于分割点集$\pi$和标记点集$\xi$, 则$f$的\textit{Riemann和}为: $$S(f;\pi ,\xi) := \sum_{i=1}^{n} (x_i-x_{i-1})f(\xi _i) .$$
对于取定的函数$f$, 定义它的基为$\mathcal{B} := \{ B_{\delta}:\delta >0 \}$, 其中$B_{\delta} := \{ (\pi ,\xi) : \| \pi \|< \delta \}$. 不难证明, 这的确是一个基. 

\begin{proposition}{}
	设闭区间$I$, $f$在$I$上Riemann可积, 则
	\begin{center}
		$\displaystyle \lim_{\mathcal{B}} S(f;\pi ,\xi) = \int_I f.$
	\end{center}
\end{proposition}
\begin{proof}
	任取$\varepsilon >0$, 设阶梯函数$F,e$使得$|f(x)-F(x)|<e(x)$且$\int e(x) < \varepsilon$, 且不妨设$F,e$的分割点集均为$\pi _0 =\{ a=x_0<\cdots <x_n=b \}$. 待定分割点集$\pi = \{ a=y_0<\cdots <y_m=b \}$和与之对应的任意标记集$\xi$. 令$\| \pi \| < \min_{1 \leq i \leq n} \{ x_i-x_{i-1} \}$, 于是$$\left| S(f;\pi ,\xi) - \int_I f \right| \leq \sum_{i=1}^{m} \left| f(\xi _i)-\int_{y_{i-1}}^{y_i} f \right|. $$
	对$i$, 分情况讨论: 
	
	若$(y_{i-1},y_i)$完全包含在某个$(x_{j-1},x_j)$中: 易知$|f(x)-f(\xi _i)| \leq |f(x)-F(x)| + |F(\xi _i)-f(\xi _i)| \leq e(x)+e(\xi _i)$, 
	从而$$\left| f(\xi _i)-\int_{y_{i-1}}^{y_i} f \right| \leq 2\int_{y_{i-1}}^{y_i} e.$$
	
	若$(y_{i-1},y_i)$恰包含某个$x_j$: 由$f$可积知$f$有界, 设$|f(x)|<M$. 此时有$$\left| f(\xi _i)-\int_{y_{i-1}}^{y_i} f \right| \leq 2(y_i-y_{i-1})M \leq 2\| \pi \|M.$$
	
	综上可得, $$\left| S(f;\pi ,\xi) - \int_I f \right| \leq 2\sum_{i=1}^{m} \int_{y_{i-1}}^{y_i} e + 2n\| \pi \|M \leq 2\varepsilon + 2nM\| \pi \|.$$
	先令$\| \pi \|\to 0$, 则$\limsup_{\| \pi \|\to 0} LHS \leq 2\varepsilon$. 再令$\varepsilon \to 0$即得. 
\end{proof}


\subsection{非负函数的积分, 连接阶梯函数与Riemann和的途径}

从坐标图上来看, 将$f$拆分为非负部分$f^+$和非正部分$f^-$再分别计算两部分的积分似乎是一个可行的方法, 即令$f=f^+-f^-$, 之后定义非负函数$g$的积分为$$\int g := \sup \left\{ \int G:0\leq G \leq g, G \in \mathcal{S}(I)  \right\}, $$
最后令$\int f = \int f^+ - \int f^-$. 

我们先来研究最简单的情形, 即$f$本身就是非负函数: 

\begin{proposition}{} \label{pro:fwfuhjuubijb}
	设闭区间$I=[a,b]$, $f \geq 0$. 则在下列说法中, 1)可以推出2), 而2)加上条件$$\sum_{i=1}^{\ell} \omega (f,(x_{i-1},x_i)) (x_i-x_{i-1}) \to 0,\quad \| \pi \| \to 0$$可以推出1). 
	\begin{itemize}
		\item $f$是Riemann可积的. 
		\item $f$有界, 则集合$$\mathcal{A} = \left\{ \int F:0\leq F \leq f, F \in \mathcal{S}(I)  \right\}$$
		有界. 进而$\sup \mathcal{A}$存在, 令$$\int f := \sup \mathcal{A}.$$
	\end{itemize}
\end{proposition}
\begin{proof}
	(1) 应用上个命题的等价条件, 由于阶梯函数都有界, 显然$f$有界. 下证$\lim_{n\to \infty} \int f_n = \sup \mathcal{A}$. 假设该式不成立, 即存在$M,\varepsilon$使得$\lim_{n\to \infty} \int f_n < M < M+\varepsilon < \sup \mathcal{A}$, 于是对足够大的$n$均有$\int f_n < M$, 且存在阶梯函数$0\leq F \leq f$使得$\int F > M+\varepsilon$. 这就是说, $$e_n(x) > f(x)-f_n(x) \geq F(x)-f_n(x).$$
	两侧(均为非负的阶梯函数)同时积分, 即得$\int e_n(x) \geq \int F(x) - \int f_n(x) > \varepsilon >0$, 矛盾. 因此$\lim_{n\to \infty} \int f_n = \sup \mathcal{A}$. 
	
	(2) 对于$\mathcal{S}(I) \ni F,F' \geq 0$, 定义$F,F'$等价当且仅当它们的分割点集是一样的. 在每个等价类$[F]$中, 设分割点集为$\pi$, 定义该等价类的“极大项”为$$\varphi ([F]) := \sum_{i=1}^{n} \inf_{(x_{i-1},x_i)} f \cdot \chi _{(x_{i-1},x_i)}.$$
	显然$\tilde{\mathcal{A}}:=\{ \int \varphi ([F]):\mathcal{S}(I) \ni F \geq 0 \}$是$\mathcal{A}$的子集, 且$\sup \tilde{\mathcal{A}} = \sup \mathcal{A}$(为什么?). 于是取$\{ \varphi _n \} \subseteq \{ \varphi ([F]):\mathcal{S}(I) \ni F \geq 0 \}$使得$\int \varphi _n \nearrow \sup \mathcal{A}$. 这样的$\varphi _n$与其分割集$\pi _n$是一一对应的关系. 若$\int \varphi _m > \int \varphi _n$, 考虑$\tilde{\pi} = \pi _n \cup \pi _m$, 则$\tilde{\pi}$所对应的$\varphi$(不一定在该序列中)满足$\int \varphi \geq \int \varphi _m$, 将其添加到序列中. 于是我们不妨设该序列存在一个子列$\{ \varphi _{k_n} \}$, 使得对任意$n>0$, $\varphi _{k_n}$的分割点集都比$\varphi _{k_1}$的分割点集细. 令$$e_{k_n} = \sum_{i=1}^{\ell} \omega (f,(x_{i-1},x_i)) \chi _{(x_{i-1},x_i)}, $$
	其中$e_{k_n}$与$\varphi _{k_n}$的分割点集相同. 那么$f(x)-\varphi _{k_n}(x) \leq e_{k_n}(x)$. 最后, 利用命题中给出的条件, $$\int e_{k_n} = \sum_{i=1}^{\ell} \omega (f,(x_{i-1},x_i)) (x_i-x_{i-1}) \to 0, \quad \| \pi \| \to 0. $$
	这就是说$\int e_{k_n} \to 0,n \to \infty$. 
\end{proof}
\begin{remark}
	从该命题的证明中, 我们发现通过一个特殊的序列$\{ \varphi _{k_n} \}$, 可以将阶梯函数(“$n\to \infty$”)与Riemann和(“$\| \pi \|\to 0$”)联系起来. 这一思路会在下一小节会有体现. 
\end{remark}
\begin{remark}
	上述条件并不是很直白. 实际上我们后面会从连续性的角度给出一个等价条件(Lebesgue-Vitali定理): (闭区间上)有界函数可积当且仅当其不连续点组成的集合是零测集. 
\end{remark}
\begin{remark}
	从(2)的证明可以看出Riemann积分并不够好, 因为我们需要人为添加一个条件才能保证可积. 
\end{remark}


接着看一般情况: 

\begin{corollary}{}
	将上面命题的$f \geq 0$这一条件去掉, 并重新定义$\int f$如下, 原来的两条结论仍成立: 取$f^+ := \max \{ f,0 \}, f^{-} := \max \{ -f,0 \}$, 则$f=f^+-f^-$. 令$\int f = \int f^+ - \int f^-$. 
\end{corollary}
\begin{proof}
	实际上只用证明下面的引理: $f$可积当且仅当$f^+,f^-$均可积且$\int f = \int f^+ - \int f^-$. 充分性显然. 必要性: 应用Lebesgue-Vitali定理, 下证$f^+$可积, 即$D(f^+)$为零测集. 若$D(f^+) \nsubseteq D(f)$, 考虑$x_0 \in D(f^+)$但$x_0 \notin D(f)$, 那么显然$f(x_0)=0$且$f$在$x_0$两侧变号. 结合$f$在$x_0$处连续, 可知存在$\delta$使得(不妨)$f|_{(x_0-\delta,x_0)}<0,f|_{(x_0,x_0+\delta)}>0$, 这就是说$f^+|_{(x_0-\delta,x_0]}=0$, 从而$f^+|_{(x_0-\delta,x_0+\delta) = f|_{(x_0-\delta,x_0+\delta)$, 立得矛盾. 所以$D(f^+) \subseteq D(f)$, 自然是零测集. 
\end{proof}
\begin{remark}
	从上面的证明我们看出, 直接使用阶梯函数进行逼近并不够方便, 这就启发我们要研究Riemann可积和连续性的关系. 
\end{remark}


\subsection{上下积分与Darboux上下和}

现在来实现之前提到过的上下极限的想法. 需要注意, 这样的上下积分依赖于$\R$的序关系, 无法推广至$\R ^n$. 实际上, 这就是将命题\ref{pro:fwfuhjuubijb}中从一侧逼近换成了从上下两侧逼近. 具体地, 对有界函数$f$定义$$\overline{\int}f := \inf \left\{ \int F:F \geq f,F \in \mathcal{S}(I) \right\},\qquad \underline{\int}f := \sup \left\{ \int F:F \leq f,F \in \mathcal{S}(I) \right\}.$$
分别称为$f$的\textit{上积分}和\textit{下积分}. 

回顾命题\ref{pro:fwfuhjuubijb}的证明, 我们希望找到某个序列$\{ \varphi _n \}$, 在$\| \pi \| \to 0$时满足$\int \varphi _n \nearrow \underline{\int}f$. 实际上, 由先前的证明, 存在一个序列$\{ \varphi _n \}$使得$\varphi _n$的分割点集比$\varphi _{n-1}$细, 在$(x_{i-1},x_i)$上的取值恰好为$\inf_{(x_{i-1},x_i)}f$, 且有$\lim_{\| \pi \|\to 0} \int \varphi _n \nearrow \underline{\int}f$. 由此可以证明: 

\begin{proposition}{}
	设有界函数$f$. 则其在$I$上Riemann可积当且仅当$\overline{\int}f=\underline{\int}f$. 
\end{proposition}
\begin{proof}
	充分性: 记$\overline{\int}f=\underline{\int}f=I$. 由上面的论述, 存在$\{ \varphi _n \}$和$\{ \psi _n \}$使得$$\int \varphi _n \leq S(f;\pi ,\xi) \leq \int \psi _n,\qquad \int \varphi _n \nearrow I, \int \psi _n \searrow I,\| \pi \| \to 0.$$
	这就说明$\lim_{\| \pi \| \to 0}S(f;\pi ,\xi) =I$. 
	
	必要性: 类似于命题\ref{pro:fwfuhjuubijb}证明的第一部分. 
\end{proof}


接着从Riemann和的角度: 对于有界函数$f$和一个分划$\pi =\{ a=x_0<\cdots <x_n=b \}$, 定义$$\overline{S}(f;\pi):=\sum_{i=1}^{n}(x_i-x_{i-1})\sup_{x\in[x_{i-1},x_i]}f(x),\qquad \underline{S}(f;\pi):=\sum_{i=1}^{n}(x_i-x_{i-1})\inf_{x\in[x_{i-1},x_i]}f(x).$$
分别称为$f$在分划$\pi$下的\textit{Darboux上和}与\textit{Darboux下和}. 容易证明, $\overline{S}(f;\pi) = \sup_{\xi} S(f;\pi,\xi)$. 进一步, 定义$$\overline{D}(f) := \inf_{\pi} \overline{S}(f;\pi),\qquad \underline{D}(f) := \sup_{\pi} \underline{S}(f;\pi).$$
我们自然有如下的命题: (进而, 可以将上述式子称为上积分和下积分)

\begin{proposition}{}
	设有界函数$f$, 则
	\begin{center}
		$\displaystyle \overline{D}(f) = \overline{\int}f,\qquad \underline{D}(f) = \underline{\int}f.$
	\end{center}
\end{proposition}
\begin{remark}
	大家很可能注意到, 利用Darboux上下和可以轻易证明(考虑分划加细时上下和的大小变化), $f$是Riemann可积的当且仅当$\overline{D}(f)=\underline{D}(f)$, 联系此命题就给出了上个命题的另一种证明. 
\end{remark}
\begin{proof}
	以前者为例. 首先注意到, 在$(x_{i-1},x_i)$上取值为$\sup_{(x_{i-1},x_i)}f$的函数$\varphi$是一个阶梯函数, 从而$\overline{D}(f) \geq \overline{\int}f$. 
	
	另一方面, 任取阶梯函数$\varphi \geq f$, 取与其相容的分划$\pi$, 那么$$\int \varphi = \sum_{i=1}^{n} \lambda _i (x_i-x_{i-1}) \geq \sum_{i=1}^{n} \sup_{x\in[x_{i-1},x_i]}f(x) (x_i-x_{i-1}) = \overline{S}(f;\pi) \geq \overline{D}(f).$$
	 由$\varphi$的任意性, 对左侧取下确界即得$\overline{\int}f \geq \overline{D}(f)$. 这就证明了原命题. 
\end{proof}

\subsection{Riemann积分的缺陷, 反常积分}

尽管Riemann积分对大多数场景都足够了, 我们还是能够找到它的一些不足之处. 大体而言, 有如下两种缺陷: \footnote{这一节内容基本来自于\textit{Measure, Integration \& Real Analysis}, Sheldon Axler.}(实际上, 相对于Lebesgue积分而言, Riemann积分的另一个缺陷是与极限换序非常麻烦, 我们在后面再介绍)

\begin{itemize}
	\item 无法处理具有太多间断点的函数. 在下面的例子中, 我们注意到Dirichlet函数只在有理数点处为$1$, 而有理数远少于无理数, 因此其积分值应当为$0$, 但在Riemann积分这里其积分值无法定义. 
\end{itemize}

\begin{example}
	证明, Dirichlet函数Riemann不可积. 
\end{example}
\begin{proof}
	对任意$(x_{i-1},x_i)$, 其内部总存在有理数和无理数, 因此$\overline{D}(f)=1,\underline{D}(f)=0$, 从而不可积. 
\end{proof}

\begin{itemize}
	\item 无法处理定义域无限长或者无界的函数. 
\end{itemize}

当然, 通过\textit{反常积分}可以部分解决这样的问题. 具体来说, 可以定义$$\int_{a}^{+\infty} f := \lim_{N\to \infty} \int_{a}^{N} f,\qquad \int_{a}^{b} f:= \lim_{N \to b^-} \int_{a}^{N} f.$$
其中, 第二个式子表示$f$在$b$处无定义. 容易验证, 反常积分满足后面将要介绍的Newton-Leibniz公式. 

\begin{example}
	计算
	\begin{center}
		$\displaystyle \int_{-\infty}^{0} e^x \dif x,\qquad \int_{0}^{1} \ln x \dif x.$
	\end{center}
\end{example}
\begin{solution}
	应用Newton-Leibniz公式, $$\int_{-\infty}^{0} e^x \dif x = \lim_{N\to -\infty} \int_{N}^{0} e^x \dif x = \lim_{N\to -\infty} (e^0-e^N)=1,$$
	\begin{center}
		$\displaystyle \int_{0}^{1} \ln x \dif x = \lim_{N\to 0^+} \int_{N}^{1} \ln x \dif x = \lim_{N\to 0^+}((1\ln 1-1)-(N\ln N-N))=-1.$
	\end{center}
\end{solution}

但反常积分仍然无法处理较复杂的情况. 在下面的例子中, 显然每个$f_k$在$[0,1]$上的积分值小于$2$, 因此$f$在$[0,1]$上的积分值应当为小于$2$的某个数, 然而Riemann积分无法为其定义积分值. 

\begin{example}
	将$(0,1)$中的有理数排列为$r_1,\cdots ,r_n,\cdots$. 对正整数$k$, 定义$f_k(x)=\begin{cases}
		(x-r_k)^{-1/2} & x>r_k \\ 0 & x \leq r_k
	\end{cases}$. 最后令定义在$[0,1]$上的$f$满足$\displaystyle f(x)=\sum_{k=1}^{\infty} \frac{f_k(x)}{2^k}$. 由于$f$在$[0,1]$的任意子区间上无界, $f$不可积, 而且连反常积分都无法定义. 
\end{example}


\newpage
\section{Riemann可积的性质与条件}

下面的性质, 实际上在上面几节就出现过了. 

\begin{proposition}{Riemann可积的充要条件} \label{pro:kejiisyc}
	设闭区间上的函数$f$. 则$f$可积当且仅当$f$有界且
	\begin{center}
		$\displaystyle \sum_{i=1}^{n} \omega (f,(x_{i-1},x_i)) (x_i-x_{i-1}) \to 0,\quad \| \pi \| \to 0.$
	\end{center}
\end{proposition}
\begin{proof}
	先前已证明了充分性. 必要性: 注意到原式即为$\overline{S}(f;\pi) - \underline{S}(f;\pi) \to 0,\| \pi \| \to 0$. 
\end{proof}

\begin{corollary}{}
	闭区间上的连续函数Riemann可积. 
\end{corollary}
\begin{proof}
	设闭区间为$[a,b]$, 显然$f$有界. 由一致连续性, 对任意$\varepsilon >0$, 存在$\delta$使得对任意满足$\| \pi \| <\delta$的分割, 有$$\sum_{i=1}^{n} \omega (f,(x_{i-1},x_i)) (x_i-x_{i-1}) \leq \varepsilon \sum_{i=1}^{n} (x_i-x_{i-1}) = \varepsilon (b-a).$$
	令$\varepsilon \to 0$即可. 
\end{proof}

\begin{corollary}{}
	闭区间上的单调函数Riemann可积. 
\end{corollary}
\begin{remark}
	由先前的结论, 单调函数是可以有可数个间断点的, 这引导我们从间断点角度出发研究可积性. 
\end{remark}
\begin{proof}
	设闭区间为$[a,b]$, 显然$f$有界. 不妨$f$在$[a,b]$上单调不减, 则
	\begin{center}
		$\displaystyle \sum_{i=1}^{n} \omega (f,(x_{i-1},x_i)) (x_i-x_{i-1}) \leq \| \pi \| \sum_{i=1}^{n} \omega (f(x_i)-f(x_{i-1})) = \| \pi \| (f(b)-f(a)) \to 0,\quad \| \pi \| \to 0.$
	\end{center}
\end{proof}


利用这个命题, 容易给出下面命题的证明(当然也可以通过更易推广的阶梯函数逼近来证明, 只不过更复杂. 另外, 由于$\lambda f+ \mu g$的积分值易于计算, 我们在前面已经证明了这是可积的. ): 

\begin{proposition}{} \label{pro:jiffxkvi}
	设$f,g \in \mathcal{R}(I), E \subseteq I$, 则
	\begin{center}
		$1)~\forall \lambda ,\mu \in \R ,\lambda f+\mu g \in \mathcal{R}(I),\qquad 2)~|f| \in \mathcal{R}(I),\qquad 3) f|_E \in \mathcal{R}(E),\qquad 4)~fg \in \mathcal{R}(I).$
	\end{center}
\end{proposition}
\begin{remark}
	实际上, 对于一元向量值函数, 只要其值域空间定义了乘法, 4)仍然成立. 
\end{remark}
\begin{proof}
	1)是显然的. 2)只需利用命题\ref{pro:kejiisyc}即可. 
	
	3) 核心的想法是将区间$E$的两端点$a,b$作为固定的分割点处理. 具体地, 对每个分割$\pi$, 在其中增加两点$a,b$得到$\pi '$, 显然$\| \pi ' \| \leq \| \pi \|$, 再利用命题\ref{pro:kejiisyc}即可. 
	
	4) 我们注意到, $(f \cdot g)(x) = \frac{1}{4}((f+g)^2(x)-(f-g)^2(x))$, 因此不妨$f=g$, 即验证$f^2 \in \mathcal{R}(I)$. 实际上, 设$f$的一个界为$M$, 则
	\begin{center}
		$|f^2(x_1)-f^2(x_2)| \leq 2M|f(x_1)-f(x_2)| \quad \Rightarrow \quad \omega (f^2,(x_{i-1},x_i)) \leq 2M\omega (f,(x_{i-1},x_i)).$
	\end{center}
	再利用命题\ref{pro:kejiisyc}即可. 
\end{proof}

在之前我们已引入了零测集的概念, 即称$E$为零测集, 若对任意$\varepsilon >0$都存在至多可数个开区间$\{ I_n \}$构成$E$的开覆盖, 且$\sum_{n=0}^{\infty} |I_n|<\varepsilon$. 需要注意, 这里由于$\sum_{n=0}^{\infty} |I_n|$绝对收敛, 其求和顺序对结果无影响. 

\begin{lemma}{}
	1) 至多可数的集合是零测集. \qquad 2) 至多可数个零测集的并集是零测集. 
	
	\noindent
	3) 设$a<b$, 则闭区间$[a,b]$不是零测集. 
\end{lemma}
\begin{remark}
	实际上可以证明, $[a,b]$的外测度就是$b-a$. 
\end{remark}
\begin{proof}
	1) 只要证明可数集是零测集, 因为显然零测集的子集是零测集. 设该集合$E=\{ x_0,\cdots ,x_n,\cdots \}$, 令$I_k=(x_k-\varepsilon 2^{-k},x_k+\varepsilon 2^{-k})$, 则$\{ I_n \}$覆盖$E$且$\sum_{|I_k|} = 4\varepsilon$, 令$\varepsilon \to 0$即可.
	
	2) 设可数个零测集$E_1,\cdots ,E_n,\cdots$. 任取$\varepsilon >0$, 存在$\{ I_n^i \}_{i \geq 0}$覆盖$E$且$\sum_{i=0}^{\infty} |I_n^i| < \varepsilon 2^{-n}$, 因此$\{ I_n^i \}_{i\geq 0,n\geq 1}$覆盖$\bigcup_{n=1}^{\infty} E_n$且$$\sum_{i\geq 0,n\geq 1}|I_n^i| = \sum_{n=1}^{\infty} \sum_{i=0}^{\infty} |I_n^i| \leq \sum_{n=1}^{\infty} \frac{\varepsilon}{2^n} = \varepsilon .$$
	令$\varepsilon \to 0$即可. 
	
	3) 设$\{ I_n \}$是$[a,b]$的开覆盖, 由有限覆盖定理, 不妨设$I_1,\cdots ,I_n$构成开覆盖, 下面对$n$归纳证明$\sum_{k=1}^{n} |I_k| \geq b-a$. $n=1$的情况是显然的, 假设$n$时成立, 则若$[a,b] \subseteq I_1 \cup \cdots \cup I_{n+1}$, 不妨$b \in I_{n+1} = (c,d)$, 其中$a<c<b<d$. 对前$n$项用归纳假设, 有$\sum_{k=1}^{n+1} |I_k| \geq (c-a)+|I_{n+1}| = c-a+d-c \geq b-a$. 
\end{proof}

在下面的命题中, \textit{几乎处处$\cdots$}(almost everywhere)指的是除了在一个零测集上以为均$\cdots$. 

\begin{theorem}{Lebesgue-Vitali定理}
	设闭区间上的函数$f$. 则$f$可积当且仅当$f$有界且在闭区间上几乎处处连续. 
\end{theorem}
\begin{proof}
	(1)必要性. 先证明一个关键的引理: 若$f$可积, 则对任意$\varepsilon >0$, $\Omega _{\varepsilon} (f) = \{ x\in [a,b]:\omega (f,x)\geq \varepsilon \}$是零测集. 从而, $f$的间断点集$D(f)=\bigcup_{n\geq 1}\Omega _{1/n}(f)$是零测集. 这里使用阶梯函数逼近给出证明: 
	
	取阶梯函数$F,e$使得$|f(x)-F(x)| \leq e(x), \int_a^b e \leq \delta$(其中$\delta$待定), 对应分划$\pi$. 在一个区间$(x_{i-1},x_i)$上, 注意到$|f(x)-f(y)| \leq |f(x)-F(x)| + |f(y)-F(y)| + |F(x)-F(y)| \leq |e(x)|+|e(y)|$, 因此只要$e$在该区间的取值小于$\varepsilon /2$, 该区间就与$\Omega _{\varepsilon}(f)$不交, 或者说其余的区间构成$\Omega _{\varepsilon}(f)$的一个开覆盖. 对这些区间的长度进行估计: 实际上$$\sum_{e_i \geq \varepsilon /2} \frac{\varepsilon}{2} (x_{i}-x_{i-1}) \leq \sum_{e_i \geq \varepsilon /2} e_i (x_{i}-x_{i-1}) \leq \int_a^b e\leq \delta .$$
	因此其长度之和不超过$\frac{2\delta}{\varepsilon}$, 令$\delta \to 0$即可. 
	
	(2) 充分性. 核心想法是对$\varepsilon >0$, 找到$D(f)$的一个开覆盖$\{ I_n \}$, 讨论每个小区间$(x_{i-1},x_i)$和这些$I_n$有无交集的情况. 令$K=[a,b] \setminus \bigcup_{n\geq 1}I_n$, 那么$K$是有界闭集, 从而是紧集. 记$\alpha _1 =\{ i:K \cap (x_{i-1},x_i) \neq \varnothing \}, \alpha _2 =\{ i:K \cap (x_{i-1},x_i) = \varnothing \}$. 
	
	首先考虑$\alpha _1$部分. 由$K$是紧集, 对任意$\varepsilon >0$存在$\delta >0$使得只要$x_i-x_{i-1}<\delta$就有$\omega (f,(x_{i-1},x_i)) \leq \omega (f,(x_{i-1},y)) + \omega (f,(y,x_i)) <2\varepsilon$, 其中$y \in K\cap (x_{i-1},x_i)$. 因此$\sum_{i \in \alpha _1} \omega (f,(x_{i-1},x_i))(x_i-x_{i-1}) < 2\varepsilon (b-a)$. 
	
	接着看$\alpha _2$部分. 将$\omega (f,(x_{i-1},x_i))$放缩为$f$在$[a,b]$上的振幅$\omega$, 我们只需考虑$\omega\sum_{i \in \alpha _2} (x_i-x_{i-1}) \leq \omega \sum_{k \geq 0} |I_k| < \omega \varepsilon$. 
	
	综上所述, $\sum_{i=1}^{n} \omega (f,(x_{i-1},x_i))(x_i-x_{i-1}) < (2(b-a)+\omega)\varepsilon$, 令$\varepsilon \to 0$即可. 
\end{proof}

利用Lebesgue-Vitali定理, 我们可以快速得到先前的一些结果, 例如命题\ref{pro:jiffxkvi}. 

\newpage
\section{Riemann积分的计算}

\subsection{Newton-Leibniz公式}

下面我们来研究这样一类由Riemann积分所定义的函数: 称形如$F(x)=\int_a^x f(t) \dif t$的函数$F:[a,b] \to \R$为$f$的\textit{变上限积分函数}, 类似可以定义变下限积分函数. 

在这个命题中, 我们看到积分让函数的性质变得更好了. 

\begin{proposition}{} \label{pro:jiffvgze}
	设$f \in \mathcal{R}([a,b])$, 则$F(x):=\int_a^x f(t) \dif t$在$[a,b]$上Lipschitz连续. 
\end{proposition}
\begin{proof}
	设$f$有界$M$. 任取$a \leq x_1 < x_2 \leq b$, 则$$|F(x_2)-F(x_1)| = |\int_{x_1}^{x_2} f(t) \dif t| \leq \int_{x_1}^{x_2} |f(t)| \dif t \leq M(x_2-x_1).$$
	这就说明$F$在$[a,b]$上Lipschitz连续. 
\end{proof}

\begin{theorem}{微积分基本定理}
	设$f \in \mathcal{R}([a,b])$, 同上定义$F$. 若$f$在$x_0$处连续, 则$F$在$x_0$处可导, 且$F'(x_0)=f(x_0)$. 
\end{theorem}
\begin{proof}
	在$x_0$附近, 有$f(t)=f(x_0)+\alpha (t)$, 其中$\alpha (t)=o(t-x_0)$, 记$\alpha (t)$的一个界为$M$. 计算$$F(x_0+h)-F(x_0)=\int_{x_0}^{x_0+h} (f(x_0)+\alpha (t)) \dif t = f(x_0)h+\int_{x_0}^{x_0+h} \alpha (t) \dif t.$$
	令$\int_{x_0}^{x_0+h}\alpha (t) \dif t=\beta (h)$, 容易证明$|\beta (h)| \leq \int_{x_0}^{x_0+h} |\alpha (t)| \dif t \leq Mh$, 故$\beta (h)=o(h)$, 这就说明$F'(x_0)=f(x_0)$. 
\end{proof}

下面的定理, 如果把条件改为$f \in C([a,b])$, 则由微积分基本定理是显然的. 这可能会让我们思考Riemann可积函数与连续函数之间的关系. 前文业已介绍, 我们可以用阶梯函数$g$来逼近一个Riemann可积的函数$f$, 于是只需要在分割点处设法使之连续化即可(例如在分割点$x_i$左右侧各延伸一小段$\delta$, 再将$(x_i-\delta ,g_{i-1})$和$(x_i+\delta ,g_{i})$用直线连接). 实际上这种手法允许我们用$C^{\infty}$的函数(在积分的意义下)逼近$f$. 当然, 定理证明本身不需要这样精确的估计, 只需利用阶梯函数即可. 

\begin{theorem}{Newton-Leibniz公式, 弱形式} \label{thm:NLro}
	设$f \in \mathcal{R}([a,b])$, $F$是$f$在$[a,b]$上(除去有限个点以外的)一个原函数, 则$\int_a^b f(x) \dif x = F(b)-F(a)$. 
\end{theorem}
\begin{remark}
	一般记$F(b)-F(a)=F(x)|_a^b$. 
\end{remark}
\begin{proof}
	对任意$\varepsilon >0$, 选取阶梯函数$g,e$使得$|f(x)-g(x)| \leq e(x),\int e <\varepsilon$. 计算可得$$F(b)-F(a)-\int_a^b g = \sum_{i=1}^{n} (F(x_i)-F(x_{i-1})-(x_i-x_{i-1})g_i) = \sum_{i=1}^{n} ((F(x_i)-g_ix_i)-(F(x_{i-1})-g_ix_{i-1})). $$
	令$\varphi (x)=F(x)-g_ix$, 应用Lagrange中值定理, 有$$|\varphi (x_i)-\varphi (x_{i-1})| \leq \sup_{x \in I}|\varphi '(x)|(x_i-x_{i-1}) = \sup_{x \in I}|f(x)-g_i|(x_i-x_{i-1}) \leq e_i(x_i-x_{i-1}).$$
	带入上式, 得$|F(b)-F(a)-\int_a^b g| < \varepsilon$, 从而$|F(b)-F(a)-\int_a^b f| <2\varepsilon$, 由$\varepsilon$的任意性即证. 
\end{proof}

\begin{theorem}{Newton-Leibniz公式, 强形式}
	设$f \in \mathcal{R}([a,b])$, $F$是$f$在$[a,b]$上(除去可数个点以外的)一个原函数, 则$\int_a^b f(x) \dif x = F(b)-F(a)$. 
\end{theorem}
\begin{remark}
	这个形式的证明比较复杂, 此处省略. 
\end{remark}

作为Newton-Leibniz公式的推论, 我们有: 

\begin{proposition}{分部积分公式}
	设$u,v\in C^1([a,b])$, 则
	\begin{center}
		$\displaystyle \int_a^b (u\cdot v')(x) \dif x = (u\cdot v)(x)\mid _a^b - \int_a^b (v \cdot u')(x) \dif x.$
	\end{center}
\end{proposition}

类似于前面的论述, 下面命题同样需要回到定义层面证明. 

\begin{proposition}{换元积分公式} \label{pro:hryrjiff}
	设$\varphi :[a,b] \to [\varphi (a), \varphi (b)]$是$C^1$的, 且严格单调, $f \in \mathcal{R}([a,b])$. 则$f(\varphi (t)) \varphi '(t)$在$[\varphi (a), \varphi (b)]$上可积且
	\begin{center}
		$\displaystyle \int_{\varphi (a)}^{\varphi (b)} f(x) \dif x = \int_a^b f(\varphi (t)) \varphi ' (t) \dif t.$
	\end{center}
\end{proposition}
\begin{remark}
	为了维持形式上的美观, 我们通常会假想积分号内部的$\varphi ' (t) \dif t$就是$\dif \varphi (t)$, 但这一记号会与Stieltjes积分冲突, 因此最好不要混淆. 
\end{remark}
\begin{proof}
	不妨考虑$\varphi$严格单增. 先解决可积的问题. 任取$\varepsilon >0$, 考虑阶梯函数逼近$|f-g| \leq e ,\int e <\varepsilon$对应$\pi$, $|\varphi '-\psi| \leq \delta ,\int \delta < \varepsilon$对应$\pi '$. 因此$$|f(\varphi (t)) \cdot \varphi '(t) - g(\varphi (t)) \cdot \psi (t)| \leq |\varphi '(t)| e(\varphi (t)) + |g(\varphi (t))| \delta (t) \leq M_1e(\varphi (t)) + M_2 \delta (t).$$
	其中$M_1,M_2$分别为$\varphi '(t),g(\varphi (t))$的界. 则阶梯函数$G=g(\varphi (t)) \cdot \psi (t)$和$E=M_1e(\varphi (t)) + M_2 \delta (t)$对应分割$\varphi (\pi) \cup \pi '$, 且$|f-G| \leq E, \int E \leq (M_1+M_2)\varepsilon$, 由$\varepsilon$任意性即证. 
	
	下面用Riemann和证明等式. 记$\varphi (t_i)=x_i$, $\varphi (\tau _i)=\xi _i$. 由Lagrange中值定理, 存在$\tau ' _i \in (t_{i-1},t_i)$使得$x_i-x_{i-1}=\varphi '(\tau ' _i) (t_i-t_{i-1})$. 因此$$\sum_{\varphi (\pi)} f(\xi _i) (x_i-x_{i-1}) = \sum_{\varphi (\pi)} f(\varphi ( \tau _i )) \varphi '(\tau _i) (t_i-t_{i-1}) + R. $$
	对误差进行估计: $$|R | = \left|\sum_{\varphi (\pi)} f(\varphi ( \tau _i )) (\varphi '(\tau ' _i)-\varphi '(\tau _i)) (t_i-t_{i-1}) \right| \leq M \sum_{\varphi (\pi)}  \omega (\varphi ';(t_{i-1},t_i)) (t_i-t_{i-1}) \to 0, \quad \| \varphi (\pi) \| \to 0.$$
	其中$M$是$|f \circ \varphi|$的界. 因此, 令$\| \pi \|\to 0$(等价于$\| \varphi (\pi) \| \to 0$)即得原式成立. 
\end{proof}
\begin{remark}
	 从这个证明可以发现, Riemann和适合处理由离散到连续的问题. 
\end{remark}

\begin{example}
	($\pi$)计算$\displaystyle \int_{-1}^{1} \sqrt{1-x^2} \dif x$. 
\end{example}
\begin{solution}
	令$x=\sin \theta$, 则$$\int_{-1}^{1} \sqrt{1-x^2} \dif x = \int_{-\pi /2}^{\pi /2} \cos ^2 \theta \dif \theta = \int_{-\pi /2}^{\pi /2} \frac{1+\cos 2\theta}{2} \dif \theta = \pi + \frac{1}{4} \sin t\mid _{-\pi}^{\pi} = \pi .$$
	这说明用分析方法定义的$\pi$和用几何方法定义的$\pi$是相同的. 
\end{solution}

下面来看一个经典的例子: 

\begin{example}
	(Wallis积分)计算$\displaystyle I_n=\int_0^{\pi /2} \sin ^n x \dif x$, 并证明Wallis乘积公式: $\displaystyle \prod_{n=1}^{\infty} \frac{4n^2}{(2n-1)(2n+1)}=\frac{\pi}{2}$. 
\end{example}
\begin{solution}
	显然$I_0=\frac{\pi}{2},I_1=1$. 考虑$$I_n = -\int_{0}^{\pi /2} \sin ^{n-1}x \dif \cos x = -\sin ^{n-1}x \cos x \mid_{0}^{\pi /2} + (n-1)\int_{0}^{\pi /2} \sin ^{n-2}x \cos x \dif x = (n-1)(I_{n-2}-I_n). $$
	由此得到递推式$I_{n+2}=\frac{n+1}{n+2} I_n$, 故$$I_{2m}= \frac{(2m-1)!!}{(2m)!!} I_0,\qquad I_{2m+1} = \frac{(2m)!!}{(2m+1)!!} I_1.$$
	注意到$I_{2m}I_{2m+1} = \frac{\pi}{2(2m+1)}$且$I_{2m} \sim I_{2m+1}$, 由此易得$I_n \sim \sqrt{\frac{\pi}{2n}}$, 这恰好等价于Wallis乘积公式. 
\end{solution}

利用Wallis乘积公式, 可以得到用于估计$n!$的Stirling公式. 首先大致估计$n!$的增长率: 利用均值不等式, 可以得到一个简单放缩: $n! < (\frac{1+\cdots +n}{n})^n = (\frac{n+1}{2})^n$, 进一步还有$n!=o((\frac{n}{2})^n)$(提示: 取$b_n=n!(\frac{n}{2})^{-n}$, 研究$\frac{b_{n+1}}{b_n}$的极限). 另一方面, 利用$(1+\frac{1}{n})^n<e$, 有$n! > (\frac{n}{e})^n$. 令$c_n=n!(\frac{n}{e})^{-n}$, 可以给出$c_n$的估计: $$\sqrt{\frac{1}{m\pi}} \sim \frac{(2m)!}{2^{2m}(m!)^2} = \frac{c_{2m} (2m)^{2m} e^{-2m}}{2^{2m} c_m^2 m^{2m} e^{-2m}} = \frac{c_{2m}}{c_m^2}.$$
不难猜到应当在先前的基础上增加$\sqrt{2n}$, 即令$a_n = \frac{c_n}{\sqrt{2n}}$, 从而$a_{2m} \sim a_m^2$. 只要证明$\{a_n \}$存在极限(提示: 证明$\{ a_n \}$是单调的), 即可说明其极限为$1$, 那么就有\textit{Stirling公式}: $$n! \sim \sqrt{2\pi n} \left( \frac{n}{e} \right)^n.$$



\subsection{Riemann积分的中值定理和Taylor公式}

\begin{theorem}{第一积分中值定理}
	设$f,g \in \mathcal{R}([a,b])$, $g$在$[a,b]$上非负(或者非正). 记$m=\inf_{x\in [a,b]}f(x),M=\sup_{x\in [a,b]}f(x)$, 则存在$\mu \in [m,M]$使得$$\int_a^b f(x)g(x)\dif x = \mu \int_a^b g(x) \dif x.$$
	更进一步, 当$f$连续时, 存在$\xi \in [a,b]$使得$$\int_a^b f(x)g(x)\dif x = f(\xi) \int_a^b g(x) \dif x.$$
\end{theorem}
\begin{proof}
	实际上我们有
	\begin{center}
		$\displaystyle m \int_a^b g(x) \dif x \leq \int_a^b f(x)g(x)\dif x \leq M \int_a^b g(x) \dif x.$
	\end{center}
	即存在$\mu \in [m,M]$使得原式成立. 命题的第二部分是显然的. 
\end{proof}

在部分情况下, 第一中值定理可以为性质不那么好的连续函数提供(类似的)微分中值定理. 令连续函数$F(x)=\int_a^x f(t) \dif t$, 注意到此时不一定有$F'=f$. 而由第一中值定理, 存在$\mu \in [m,M]$使得$F(b)-F(a) = \mu (b-a)$. 

\begin{lemma}{第二积分中值定理, 弱形式}
	设$f \in C([a,b]),g \in C^1([a,b])$, $g$是单调不增的非负函数, 则存在$\xi \in [a,b]$使得$$\int_a^b f(x) g(x) \dif x = g(a) \int_a^{\xi} f(x) \dif x.$$
\end{lemma}
\begin{proof}
	考虑$f$的一个原函数$F(x)=\int_a^x f(t) \dif t$, 则$F$是$C^1$的. 对$F,g$用分部积分公式: $$\int_a^b F'(x)g(x) \dif x = F(x)g(x)\mid _a^b - \int_a^b F(x)g'(x) \dif x.$$
	由$g$单调不增知$g' \leq 0$, 对$F,g'$用第一中值定理, 则存在$\xi \in [a,b]$使得$$\int_a^b F(x)g'(x) \dif x = F(\xi) \int_a^b g'(x) \dif x = F(\xi)(g(b)-g(a)).$$
	因此(这里假设$g(a)\neq 0$, 否则$g=0$结论是显然的)$$\frac{1}{g(a)} \int_a^b f(x)g(x) \dif x = \frac{g(b)}{g(a)}F(b)+\left( 1-\frac{g(b)}{g(a)} \right)F(\xi) \in F([\xi ,b]).$$
	即存在$\tau \in [\xi ,b]$使得$$\int_a^b f(x)g(x) \dif x = F(\tau)g(a)= g(a) \int_a^{\tau} f(x) \dif x.$$
\end{proof}

\begin{lemma}{第二积分中值定理}
	设$f,g \in \mathcal{R}([a,b])$, $g$是单调不增的非负函数, 则存在$\xi \in [a,b]$使得$$\int_a^b f(x) g(x) \dif x = g(a) \int_a^{\xi} f(x) \dif x.$$
\end{lemma}
\begin{remark}
	和弱形式相比, 由于$f,g$的性质并不好, 我们无法使用分部积分法. 作为替代, 在Riemann和的层面可以使用Abel求和(离散分部积分)来处理. 实际上(后面会介绍), 用Stieltjes积分就能解决分部积分的问题. 
\end{remark}
\begin{proof}
	令$F(x)=\int_a^x f(t)\dif t$, 这里$F$是连续的. 类似于弱形式, 只要证明(不妨$g(a) \neq 0$)$mg(a) \leq \int_a^b f(x)g(x) \dif x \leq Mg(a)$, 其中$m=\inf_{x \in [a,b]}F(x), M=\sup_{x \in [a,b]}F(x)$. 
	
	我们需要得到类似于分部积分的式子, 因此考虑对$f$用第一中值定理以替代对$F$用微分中值定理, 即存在$\tau _i \in [\inf_{(x_{i-1},x_i)}f,\sup_{(x_{i-1},x_i)}f]$使得$F(x_i)-F(x_{i-1}) = \tau _i (x_i-x_{i-1})$. 由此可以猜测下式成立: $$\left| \sum_{\pi} \tau _i (x_i-x_{i-1}) g(\xi _i) - \int_a^b f(x)g(x) \dif x \right| \to 0,\quad \| \pi \| \to 0.$$
	实际上, 由$f\cdot g$可积, 只要计算
	\begin{align*}
		\left| \sum_{\pi} \tau _i (x_i-x_{i-1}) g(\xi _i) - \sum_{\pi} f(\xi _i)g(\xi _i)(x_i-x_{i-1}) \right| &= \left| \sum_{\pi} (\tau _i-f(\xi _i))g(\xi _i)(x_i-x_{i-1})  \right| \\
		&\leq  g(a) \sum_{\pi} \omega (f;(x_{i-1},x_i))(x_i-x_{i-1})  \to 0, \| \pi \| \to 0.
	\end{align*}
	最后一步是因为$f$可积. 对$\sum_{\pi} \tau _i (x_i-x_{i-1}) g(\xi _i) = \sum_{\pi} (F(x_i)-F(x_{i-1})) g(\xi _i)$作Abel变换, 即$$\sum_{i=1}^n (F(x_i)-F(x_{i-1})) g(\xi _i) = F(b)g(\xi _n) + \sum_{i=1}^{n-1} (g(\xi _i)-g(\xi _{i+1})) F(x_i) \leq Mg(\xi _1).$$
	同理该式$\geq mg(\xi _1)$. 最后令$\xi _1=a$即可(这是因为分割是人为选取的). 
\end{proof}



\begin{theorem}{第二积分中值定理}
	设$f,g \in \mathcal{R}([a,b])$, $g$是单调函数, 则存在$\xi \in [a,b]$使得$$\int_a^b f(x) g(x) \dif x = g(a) \int_a^{\xi} f(x) \dif x + g(b) \int_{\xi}^{b} f(x) \dif x.$$
\end{theorem}
\begin{proof}
	不妨考虑$g$单调不增, 则令$G(x)=g(x)-g(b)$可知$G$是单调不增的非负函数. 利用引理立得上式成立. 
\end{proof}

记$\lim_{h \to 0^+}g(a+h) = g(a_0)$(这个极限显然存在), 则对任意$\varepsilon >0$均存在$\xi \in [a,b]$使得$$\int_a^bf(x)g(x) \dif x = (g(a_0)+\varepsilon) \int_a^{\xi} f(x) \dif x.$$

\newpage
\section*{一些习题 ~~\small 对应原书6.1, 6.2习题} \label{sec:ex7.1}

\newpage
\section*{一些习题 ~~\small 对应原书6.3习题} \label{sec:ex7.2}

\newpage
\section{有向可加函数与Stieltjes积分}

将Riemann积分算子的区间可加性抽象出来, 得到所谓可加函数. 即, 设$I:U \times U \to V$, 若对任意$x,y,z \in U$都有$I(x,z)=I(x,y)+I(y,z)$, 则称$I$为$U$上的\textit{有向可加函数}. 这里用一般的$U,V$是为了强调有向可加函数的性质与$\R$的特殊结构无关. 

实际上, 除了Riemann积分以外, 很多函数都是有向可加函数. 例如, 相对速度的计算公式$v_{13}=v_{12}+v_{23}$, 电势差$\varphi _{ac}=\varphi _{ab}+\varphi _{bc}$等等. 不难发现, 电势差$\varphi _{ab}$的背后是场强的(多元函数)积分$\int_a^b E(r) \dif r$, 但是相对速度背后并没有明显的积分背景. 这一现象提示我们研究, 何种有向可加函数能表示为Riemann积分? 更进一步, 注意到有些离散情况的公式和积分公式非常相似(如Abel变换和分部积分), 那么可否发展一种积分理论将这些公式统一起来? 

在第一小节, 我们会探讨哪些函数能表示为Riemann积分(毕竟这种积分易于计算)及其计算. 而作为推广的Stieltjes积分将在第二小节介绍. 

\subsection{有向可加函数的Riemann积分表示}

下面的命题给出积分产生有向可加函数的一个充分条件. 

\begin{proposition}{}
	设有向可加函数$I:[a,b] \times [a,b] \to \R$. 若存在$f \in \mathcal{R}([a,b])$使得对任意$a \leq \alpha < \beta \leq b$都有$$\inf_{x\in [\alpha ,\beta]} f(x) (\beta -\alpha) \leq I(\alpha ,\beta) \leq \sup_{x\in [\alpha ,\beta]} f(x) (\beta -\alpha),$$
	则$\displaystyle I(a,b)=\int_a^b f(x) \dif x$. 
\end{proposition}

\subsection{Stieltjes积分}

对于有向可加函数$I:[a,b] \times [a,b] \to \R$, 通过定义函数$\mu :[a,b] \to \R ,t \mapsto I(a,t)$, 可将$I(\alpha ,\beta)$视作$\mu (\beta)-\mu (\alpha)$. 从几何的视角, 若将$I(\alpha ,\beta)$视作区间$[\alpha ,\beta ]$的长度(两端点的有向距离), 则$\mu$就给出了某个点的绝对位置(亦称$\mu$为测度函数). Riemann积分即是$\mu (x)=x$的情形. 这启发我们, 将Riemann积分中的$\dif x$替换为$\dif \mu (x)$, 或者说将Riemann和中的$x_i-x_{i-1}$替换为$\mu (x_i)- \mu (x_{i-1})$. 

\begin{definition}{Stieltjes积分}
	设闭区间$I$上的函数$f,\mu$, 定义
	\begin{itemize}
		\item 对于分划$\pi = \{ a=x_0<\cdots <x_n=b \}$, 称形如$$f = \sum_{i=1}^{n} \lambda _i \chi _{(\mu (x_{i-1}),\mu (x_{i}))} $$
		(其中$\lambda _i \in \R$)的函数为$\pi$上的一个\textit{$\mu$-阶梯函数}. 记$I$上全体$\mu$-阶梯函数的集合为$\mathcal{S}(I;\mu )$. 
		\item 承上一条, 定义$f$的\textit{积分}(integral)为$$\int f := \sum_{i=1}^{n} \lambda _i \int \chi _{(\mu (x_{i-1}), \mu (x_i))} = \sum_{i=1}^{n} \lambda _i (\mu (x_i)-\mu (x_{i-1})). $$
		\item 称$f$是\textit{Stieltjes可积的}, 如果对任意$\varepsilon >0$均存在$F,e \in \mathcal{S}(I;\mu )$使得$$\forall x \in I,|f(x)-F(x)|<e(x),\qquad \int e < \varepsilon .$$
	全体$I$上Riemann可积的函数构成集合$\mathcal{R}(I;\mu)$. 
	\end{itemize}
	类似Riemann积分, 我们还能验证Riemann和的定义与上方定义等价: 
	\begin{itemize}
		\item $f$的\textit{Riemann和}为$$S(f;\pi ,\xi) := \sum_{i=1}^{n} (\mu(x_i)-\mu(x_{i-1}))f(\xi _i) .$$
		\item 固定$f$, 它的\textit{基}为$\mathcal{B} := \{ B_{\delta}:\delta >0 \}$, 其中$B_{\delta} := \{ (\pi ,\xi) : \| \pi \|< \delta \}$. 
		\item 称$f$在$I$上\textit{Stieltjes可积}, 若极限$\displaystyle \lim_{\mathcal{B}} S(f;\pi ,\xi)$存在, 且将其定义为$\displaystyle \int_a^b f(x) \dif \mu (x)$. 
	\end{itemize}
\end{definition}

同样地, 对阶梯函数与Riemann和而言, 分别可以定义上下积分及Darboux上下和. 这里省略具体定义. 

可以证明Stieltjes积分拥有和Riemann积分类似的性质, 即命题\ref{pro:jiffxkvi1}, \ref{pro:kejiisyc}(要将$x$换成$\mu (x)$), \ref{pro:jiffxkvi}. 

类似于换元积分公式, 不难证明, 设$\mu (x) = \int_a^x \rho (x)\dif x$, 则对可积的$f$有$\displaystyle \int_a^b f(x) \dif \mu (x) = \int_a^b f(x) \rho (x) \dif x$. 特别地, 当$\mu$连续可微时, 就可以将积分号内的$\dif \mu (x)$看做是$\mu '(x) \dif x$. 

然而, 唯一与Riemann积分不同之处在于, Stieltjes积分没有Newton-Leibniz公式. 这是否意味着其没有作为N-L公式推论的分部积分公式? 











