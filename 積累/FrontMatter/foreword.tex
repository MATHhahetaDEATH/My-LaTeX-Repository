\chapter*{前言}

測試文本

\quo{項籍少時,學書不成,去學劍,又不成。項梁怒之。籍曰:「書足以記名姓而已。劍一人敵,不足學,學萬人敵。」於是項梁乃教籍兵法,籍大喜,略知其意,又不肯竟學。項梁嘗有櫟陽逮,乃請蘄獄掾曹咎書抵櫟陽獄掾司馬欣,以故事得已。項梁殺人,與籍避仇於吳中。吳中賢士大夫皆出項梁下。每吳中有大繇役及喪,項梁常為主辦,陰以兵法部勒賓客及子弟,以是知其能。秦始皇帝游會稽,渡浙江,梁與籍俱觀。\hly{籍曰:「彼可取而代也。」梁掩其口,曰:「毋妄言,族矣!」梁以此奇籍。}籍長八尺餘,力能扛鼎,才氣過人,雖吳中子弟皆已憚籍矣。}{史記-項羽本紀}{司馬遷}
