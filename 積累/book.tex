\documentclass{plainbook}

\usepackage{amsfonts}
\usepackage{amsmath}
\usepackage{amssymb}
\usepackage{hyperref}
\usepackage{svg}
\usepackage{booktabs}
\usepackage{framed}
\usepackage{tikz-cd}
\usepackage{epigraph}
\usepackage{cite}

% quote

\usepackage{adjustbox,libertine,soul,standalone,xcolor,color}

\newcommand*\quotefont{\fontfamily{LinuxLibertineT-LF}}

    \newlength\shadeDepth
    \newcommand{\shadequote}[2]{%
        \begin{center}\begin{adjustbox}{bgcolor=WhiteSmoke,margin=0.5mm,rndcorners=2.5pt}
            \adjustbox{center,minipage=0.9\linewidth,gstore totalheight=\shadeDepth}{#1}%
            \adjustbox{rlap,valign=b,raise=\shadeDepth/2-1.75em,scale={-1}{-1}}{\Huge``}%
            \hspace{-\linewidth}%
            \adjustbox{rlap,valign=B,raise=\shadeDepth/2-1.50em,left=\linewidth}{\Huge``}%
            \hspace{-\linewidth}%   
            \ifblank{#2}{}{\adjustbox{left,raise=-\shadeDepth/2-2em,minipage=\linewidth}{#2}}%
        \end{adjustbox}\end{center}
    }

\newcommand{\hly}[1]{\colorbox{yellow!50}{#1}}
\newcommand{\hlb}[1]{\colorbox{DeepSkyBlue!30}{#1}}
\newcommand{\hll}[1]{\colorbox{lime!30}{#1}}
\newcommand{\hlp}[1]{\colorbox{pink!30}{#1}}

\newcommand{\quo}[3]{\shadequote{#1}{出自#3《#2》}}


\definecolor{quoc}{HTML}{7e0f12}

% font; do not use in overleaf

\usepackage[UTF8,scheme=plain,fontset=none]{ctex}
    \setCJKmainfont[BoldFont={KingHwa_OldSong},ItalicFont={FZSongKeBenXiuKaiT-R-GB}]{Shippori Mincho B1}
    \setCJKsansfont[BoldFont={Source Han Serif TC-SemiBold}]{FZKai-Z03T}
    \setCJKmonofont[BoldFont={Source Han Serif TC-SemiBold}]{Source Han Serif TC-Regular}
    \setCJKfamilyfont{zhsong}{Source Han Serif TC-Regular}
    \setCJKfamilyfont{zhhei}{Source Han Serif TC-SemiBold}
    \setCJKfamilyfont{zhkai}[BoldFont={Source Han Serif TC-SemiBold}]{FZKai-Z03T}
    
\renewcommand\abstractname{摘要}
\renewcommand\refname{參考文獻}
\renewcommand\figurename{圖}
\renewcommand\tablename{表}
\renewcommand\contentsname{目錄}

\title{積累}

% Set the authors of the book (multiple authors separated by \and).
\author{bilibili:晨沐公Kasumi \quad github:MATHhahetaDEATH}

% Set the date to the current date.
\date{\today}

% customised commands
\definecolor{winered}{rgb}{0.5,0,0}



% Begin the document.
\begin{document}

% Front matter section.
\frontmatter

% Include the title page, which is located in the FrontMatter subfolder.
\include{./FrontMatter/titlepage}

% Create the book's title page.
\maketitle\pagebreak

% Include the dedication page from the FrontMatter subfolder.
% \include{./FrontMatter/dedication}

% Include the epigraph page from the FrontMatter subfolder.
% This code snippet creates a quote block attributed to an author.

% Vertically space the content evenly, pushing the quote to the center of the page.
\vspace*{\fill}

% Set the font size to \Large (large) and the text style to italics.
\Large\textit{Young man, in mathematics you don’t understand things. You just get used to them. }

% Add some vertical space after the quote.
\bigskip

% The author's name is right-aligned and set in sans-serif small caps.
\begin{flushright}
    \sffamily\scshape John von Neumann
\end{flushright}

% Set the font back to the default (normal font size and style).
\normalfont\normalsize

% Vertically space the content evenly again, pushing any remaining space to the bottom of the page.
\vspace*{\fill}


% Include the foreword page from the FrontMatter subfolder.
\chapter*{前言}

在数学中, 对于每一个数学对象(例如极限), 我们会例行公事般地考虑它的一些常见的性质. 比如说, 这个对象最基本的例子是什么, 这种对象是否存在, 如果存在的话它是否具有唯一性, 它的子对象和商对象(如果有的话)都具有什么性质(比如说遗传了原来的对象的什么性质), 这个对象的可计算性以及在特定映射下的行为等等.




这份讲义cover的内容: 

- “抽象废话”范畴论, 当然只有一点点

- 作为工具和灵感来源的矩阵理论

- 几乎同构于矩阵理论的线性映射理论, 但我们会研究一些无限维情况

- 经典的群结构$\Z / n\Z$, 以及一般的群论

- 经典的环结构$\F [x]$, 以及初步的环论

- 域理论

- 初步的模论

换句话说, 本讲义大致覆盖高等代数与基础抽象代数的内容. 





抽象的代数需要直觉, 例如“自由度”“置换”等想法. 用具体详实的例子引出抽象理论. 


很多想法都是读者所熟悉的, 例如数论中的中国剩余定理, 多项式理论中的因式分解等. 因此这份讲义将不会侧重解释这些初等理论, 而是试图将其推广. 




% Include the preface page from the FrontMatter subfolder.
% \include{./FrontMatter/preface}

% Include the acknowledgement page from the FrontMatter subfolder.
% \chapter*{致谢}



\undersign

% Table of contents page.
\tableofcontents

% Main matter section.
\mainmatter


\chapter{文言文}

\section{哲理}


\quo{夫所爲求福而辭禍者,以福可喜而禍可悲也。\textbf{人之所欲無窮,而物之可以足吾欲者有盡,美惡之辨戰乎中,而去取之擇交乎前。則可樂者常少,而可悲者常多。是謂求禍而辭福。}夫求禍而辭福,豈人之情也哉?物有以蓋之矣。彼遊於物之內,而不遊於物之外。物非有大小也,自其內而觀之,未有不高且大者也。彼挾其高大以臨我,則我常眩亂反覆,如隙中之觀鬥,又焉知勝負之所在。是以美惡橫生,而憂樂出焉,可不大哀乎!}{超然臺記}{蘇軾}





\chapter{現代文字}


\section{場景植入}

\quo{我正處在生活中和網絡上的邊緣化進程中,對此我沒有什麽意見,也沒有什麽情緒​。包括最後的隱沒,​我知道那也是個時間問題。不單是我,所有的世代,所有今天在網絡上活躍的人,都會經歷這個過程,都會迎接這個結局。當意識到這一切不可避免時,當意識到掙扎毫無作用時,人就能學會平靜接受,​並且開始思考那個眞正的問題:此時此刻我應該做什麽?\\
{\color{quoc} 就像是人類早期的穴居人一樣,當他步出自己的洞穴極目眺望,祇看見茫茫蠻荒,無盡曠野。然後這大地上有營建,人群聚集,市鎮興起,天空中有鐵鳥呼嘯飛過,衛星閃閃發光,火箭穿梭其中。這個世界人聲鼎沸,熱鬧非凡,興衰交替​,晝夜不停。​然而在他眼中看到的是一種荒野替代了另外一種荒野,他手中的木棍替換了另外一根木棍,​賸下的祇是在這荒野之中無盡的漫遊。}}{習慣邊緣化,然後隱沒}{槽邊往事blog}


\quo{關於午睡悲傷綜合癥有許多種解釋,有關於人類睡眠周期的,有關於人體自我保護的,有關於大腦認知原理的,我個人最喜歡其中和​醫學完全無關的一種文學化表述: \\
{\color{quoc} 之所以會有午睡悲傷綜合徵出現,​那是因為遠古的召喚。一個原始人內心最大的恐懼莫過於自己在荒野裏醒來,發現天色昏暗,周遭無人,衹有夜風吹拂。他的第一個想法就是同伴不知道什麽時候都已經離開,衹賸下自己一個人​畱在曠野裏,夜色中有未知的危險​正從四面八方逼近。} \\
我認為這個解釋最為合理,解釋了一個人午睡到晚上七八點鍾醒來時那種不知今夕何夕,不知身在何處的感受。以及為什麽會感覺到孤獨,為什麽在孤獨之後會感覺到緊張、焦慮和煩躁---在漫長的進化過程中,一定有什麽寫入了人類的DNA,區區千年的文明,繁華的都市生活,混凝土墻壁和鋼鐵防盜門阻攩不了這種來自曠野的風息。}{午睡悲傷綜合征}{槽邊往事blog}

\section{比喻}



\chapter{有趣的說法}

\section{人類}

\quo{同樣的,對生活和對生活模式理解的單一而平庸,也導致可以直接去掉​人類。因為去制造那種生活的圖景的時候,人類在其中是最為多餘和最為​低效的一個環節。都在喝紅酒,都在喝咖啡,都在喫提拉米蘇,都在看五星酒店陽臺上的斜陽,都在墊腳翹腳比心比V,​看一個和看十個一百個沒什麽不同。單個人的人都在制造類似的內容,那麽人就是多餘,用機器直接制造​就好了,效率和質量反而高得多。你説這張陽臺夕陽照要是換泳裝就好了,AI打個響指就能​做到,都不需要去衛生間一趟。 \\
那麽,這裏我還可以再向前推導一點:{\color{quoc} 人類這種生物本身就是​一種高度冗餘,高度相互備份的存在。由於欲望類似,理念類似,所以大家身處類似的幻夢中,​做著高度類似的追求行為,同時因為肉體、地理和時間的限制,造成這些行為本身效率​低下。​AI很便當地解決了這個問題,它因為效率極高,學習速度快,它不需要相互備份,​一臺服務器就可以生産制造幾億人夢想的生活,於是就顯得人類​自身很多餘。}在制造幻夢​、展示幻夢這件事上,AI 才是專業的。}{人類退場}{槽邊往事blog}


\section{日常生活}

\quo{通過實驗,我發現拏起手機​已經變成了一個下意識的習慣,​主要用途是為了塡補念頭和念頭之間的空隙。當我看著樹林的時候,想著原來綠色還有那麽多變化的時候,我不會​想著拏起手機。但是當這個念頭結束,下一個念頭還沒有産生的時候,拏手機的衝動就會突然蹦出來,非常突兀,非常草率,蹦出來説:你看一下吧。然後我下意識就想伸手,覺得在兩個念頭之間的空白處突然有了事情要做。
我對自我進行觀察,得出一個結論​:{\color{quoc} 拏起手機看,這是現代人思緒的標點符號。​拏起多少次,證明打了多少個標點符號。次數越多,​思緒越快越亂越跳躍。}}{你能多久不碰手機}{槽邊往事blog}













% Appendices section.
% \appendix

% Include the "about" appendix from the BackMatter subfolder.
% \include{./BackMatter/about}

% Include the "abbreviation" appendix from the BackMatter subfolder.
% \include{./BackMatter/abbreviation}

% Include the "notation" appendix from the BackMatter subfolder.
% \include{./BackMatter/notation}

% Include the "code" appendix from the BackMatter subfolder.
% \include{./BackMatter/code}

% Include the "glossary" appendix from the BackMatter subfolder.
% \include{./BackMatter/glossary}

% Include the "index" appendix from the BackMatter subfolder.
% \include{./BackMatter/index}

% Include the "references" appendix from the BackMatter subfolder.

\bibliographystyle{plain}
\bibliography{references.bib}

%\nocite{*}

% End the document.
\end{document}
