\chapter{群论}

疯狂抄书中……

\section{基本想法}

\begin{definition}{二元运算}
	定义非空集合$S$上的二元运算$\cdot:S \times S \to S$, 考虑$(S,\cdot )$的如下性质:  
	\begin{itemize}
		\item $S$是\textit{交换的}, 如果对任意$x,y \in S$有$xy = yx$. 
		\item $S$是\textit{结合的}, 如果对任意$x,y,z \in S$有$(xy)z=x(yz)$. 
		\item $S$上的一个\textit{幺元}(单位元)$e$, 满足对任意$x \in S$都有$xe=ex=e$. 
		\item $x$的一个\textit{逆元}$x^{-1}$, 满足$xx^{-1} = x^{-1}x=e$. 
	\end{itemize}
\end{definition}
\begin{remark}
	不造成歧义时, $x\cdot y$可简写为$xy$, $(S,\cdot)$可简写为$S$. 括号表示优先的运算. 
\end{remark}
\begin{remark}
	不难(并且应当)验证逆元$x^{-1}$(相对于$x$)和幺元$e$(相对于$S$)的唯一性. 
\end{remark}
\begin{remark}
	后面会看到, 这些性质的常见程度为结合性质$\sim$存在幺元$>$所有元素存在逆元$>$交换性质. 
\end{remark}

我们引入自然的$x^n$定义, 其中$n$是整数. 

\begin{definition}{幺半群, 群}
	设带有二元运算$\cdot$的非空集合$S$, 称$(S,\cdot)$为\textit{幺半群}, 若运算$\cdot$满足结合律且$S$中存在幺元. 进一步, 称幺半群$G$为\textit{群}, 如果其所有元素均可逆. 
\end{definition}
\begin{remark}
	关于交换性质的扩展定义: 称交换的幺半群为\textit{交换幺半群}, 交换的群为\textit{Abel群}. 
\end{remark}

\begin{definition}{子幺半群, 子群}
	对于幺半群$S$和非空子集$T \subset S$, 称$T$是一个\textit{子幺半群}, 如果
	\begin{itemize}
		\item $e \in T$. 
		\item $T$对于$\cdot$封闭, 即$\cdot$在$T$上的限制映射之值域包含于$T$. 
	\end{itemize}
	对于群$G$和非空子集$H \subset G$, 称$H$是一个\textit{子群}(记作$H<G$), 如果
	\begin{itemize}
		\item $e \in H$. 
		\item $H$对于$\cdot$封闭. 
		\item $H$对于取逆元映射$\bigcdot ^{-1}$封闭. 
	\end{itemize}
\end{definition}
\begin{remark}
	子群的后两个条件可以合并为一个: 对任意$x,y \in H$有$xy^{-1} \in H$. 
\end{remark}

\begin{example}
	幺半群中所有可逆元素构成的子幺半群是一个群. 
\end{example}

\begin{example}
	记$n$阶实矩阵构成集合$\mathrm{M}(n,\R)$, 该集合对矩阵乘法构成幺半群. 考虑其子集$\mathrm{GL}(n,\R):= \{ A \in \mathrm{M}(n,\R):\det A \neq 0 \}$和$\mathrm{SL}(n,\R):= \{ A \in \mathrm{M}(n,\R):\det A =1 \}$, 它们对矩阵乘法构成群, 分别称作\textit{一般线性群}和\textit{特殊线性群}. 
\end{example}

\begin{example}
	对于集合$X$, 考虑所有$X \to X$的双射所成集合$\sym X$, 这个集合对映射复合构成群, 称为\textit{对称群}(置换群). 参考后文. 
\end{example}

记群$G$的\textit{阶}$\ord G:=|G|$, 当$G$为无限集合时$|G|$表示其基数. 

在$G$中任取集合$E$, 考虑$E$所\textit{生成}的“闭包”, 即包含$E$的最小子群$\displaystyle \ang{E}:= \bigcap_{H<G,E\subset H} H$. 容易证明, 对于有限集合$E=\{ a_1,\cdots ,a_n \}$, $\ang{E}=\{ a_1^{\alpha _1} \cdots a_n^{\alpha _n}:\alpha _j \in \Z ,j=1,\cdots ,n \}$. 特别地, 当$E$是单点集$\{ x \}$时, 简记$\ang{x} = \ang{\{ x \}}$. 此时定义$\ord x:=\ord \ang{x}$为$x$的\textit{阶}. 亦可证明, $\ord x$有限时即为最小的使得$x^n=e$的正整数$n$. 

\begin{example}
	对于群$G$, 若存在$x \in G$使得$G=\ang{x}$, 则称$G$为一个\textit{循环群}. 下一节会研究循环群的结构. 
\end{example}

我们采用惯常的记号$AB$表示$\{ ab:a \in A,b \in B \}$, 并简记$\{ x \}A$为$xA$, $Bx$同理. 

类似于利用等价类和商集对集合进行划分的方法, 这里可以考虑一个群$G$的划分. 选取其任一子群$H$, 我们希望利用$H$的某一特征划分$G$. 自然的想法是作出$GH$或者$HG$并设法去除其中重复的集合. 这就是定义陪集的动机: 

\begin{definition}{陪集}
	设$H,K$为群$G$的子群. 
	\begin{itemize}
		\item 定义\textit{左陪集}为$G$中形如$xH$的子集, 并记全体左陪集构成集合$G/H$. 类似可得右陪集和$G\setjianfa H$的定义. 
		\item 定义\textit{双陪集}为$G$中形如$HxK$的子集, 并记全体双陪集构成集合$H\setjianfa G /K$. 
		\item 定义$H$在$G$中的\textit{指数}$(G:H):=\ord G/H$. 
	\end{itemize}
\end{definition}
\begin{remark}
	在三种陪集中, 我们也许更偏爱左陪集, 因为左乘一个元素可以自然地视作映射$H \to G, h \mapsto xh$. 
\end{remark}
\begin{remark}
	左右陪集是双陪集的特例, 即$H$或$K$为$\{ e \}$的情况. 
\end{remark}
\begin{remark}
	令$x \sim y$当且仅当$HxK=HyK$, 那么这是一个等价关系. 
\end{remark}
\begin{remark}
	可以证明$\ord G/H = \ord G \setjianfa H$. 因此讨论指数时无需指定左或右. 
\end{remark}

自然想到, 取映射$\tau _K :K \to K, k \mapsto y^{-1}xk$, 则$xK=yK$等价于$\tau _K$是双射. 同理可得, 令$\tau _H:H \to H, h \mapsto hyx^{-1}$, 则$Hx=Hy$等价于$\tau _H$是双射. 最后, $HxK=HyK$等价于$\tau _H \circ \tau _K$是双射. 

更进一步, 以$\tau _K$为例: 若想要$\tau _K$是双射, 由于其本身就是单射, 故只需要求满射, 一个充分条件是$y^{-1}x \in K$. 反过来, 当$xK=yK$时自然有$xe=yk$即$y^{-1}x=k \in K$. 这就证明了下面的命题. 

另一种证明该命题的方法是考虑$x$驱使$H$“平移”的作用. 如果将$H$想象成向量空间, $x$想象成向量并取加法为$\cdot$, 这一点会非常直观. 

\begin{proposition}{}
	设$H$为群$G$的子群, 则对于$x \in G$有$xH=H \Leftrightarrow x \in H$. 进而对于$x,y \in G$有$xH=yH \Leftrightarrow y^{-1}x \in H$. 右陪集和双陪集的情况同理. 
\end{proposition}
\begin{proof}
	必要性: 取$e \in H$即得$x=xe \in xH=H$. 充分性: 由乘法封闭性可知$xH \subset H$. 同理有$x^{-1}H \subset H$, 即$H \subset xH$. 这说明$xH=H$. 
\end{proof}

现在考虑用陪集划分群$G$: 

\begin{proposition}{}
	设$H,K$为群$G$的子群, 则
	\begin{itemize}
		\item $G=\bigsqcup_x HxK$, 其中$x$遍历每个双陪集的代表元. 
		\item (Lagrange定理)$\ord G = (G:H) \ord H$. 
	\end{itemize}
\end{proposition}
\begin{proof}
	(1) 先证明若$HxK,HyK$有交则相等: 记$hxk=h'yk'$, 则$x=h^{-1}h'yk'k^{-1} \in HyK$, 由等价关系的传递性可知$HxK=HyK$. 这就证明了无交部分. 并部分则是显然的. 
	
	(2) 将$G$进行无交并分解即$G=\bigsqcup_x xH$, 记$E=\{ e \}$, 则$G=\bigsqcup_x x \bigsqcup_{y} yE = \bigsqcup_{x,y} xy E$. 注意到对于$G,H$, $(G:H)$就是可能的$x$的个数, 而$(G:1)=\ord G,(H:1)=\ord H$, 结合$x,y$个数是$x$个数与$y$个数之积, 立得原式成立. 
\end{proof}
\begin{remark}
	在(2)中实际上证明了: 若$K<H<G$, 则$(G:K)=(G:H)(H:K)$. 当然这直接由Lagrange定理可得. 
\end{remark}
























