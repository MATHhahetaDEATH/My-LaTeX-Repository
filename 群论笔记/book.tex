\documentclass{plainbook}

\usepackage{amsfonts}
\usepackage{amsmath}
\usepackage{amssymb}
\usepackage{hyperref}
\usepackage{svg}
\usepackage{booktabs}
\usepackage{framed}


% font; do not use in overleaf

\usepackage[UTF8,scheme=plain,fontset=none]{ctex}
    \setCJKmainfont[BoldFont={Source Han Serif SC-SemiBold},ItalicFont={FZKai-Z03}]{FZShuSong-Z01}
    \setCJKsansfont[BoldFont={Source Han Serif SC-SemiBold}]{FZKai-Z03}
    \setCJKmonofont[BoldFont={Source Han Serif SC-SemiBold}]{FZFangSong-Z02}
    \setCJKfamilyfont{zhsong}{FZShuSong-Z01}
    \setCJKfamilyfont{zhhei}{Source Han Serif SC-SemiBold}
    \setCJKfamilyfont{zhkai}[BoldFont={Source Han Serif SC-SemiBold}]{FZKai-Z03}
    \setCJKfamilyfont{zhfs}[BoldFont={Source Han Serif SC-SemiBold}]{FZFangSong-Z02}


\title{临时笔记}

% Set the authors of the book (multiple authors separated by \and).
\author{bilibili:晨沐公Johnny \quad github:MATHhahetaDEATH}

% Set the date to the current date.
\date{\today}

% customised commands
\definecolor{winered}{rgb}{0.5,0,0}
\newcommand{\exref}[1]{\ref}
\newcommand{\xl}[1]{\overrightarrow{#1}}
\newcommand{\flr}[1]{\lfloor #1 \rfloor}
\newcommand{\ang}[1]{\langle #1 \rangle}
\newcommand{\R}{\mathbb{R}}
\newcommand{\C}{\mathbb{C}}
\newcommand{\Z}{\mathbb{Z}}
\newcommand{\F}{\mathbb{F}}
\newcommand{\lmap}{\mathcal{L}}
\newcommand{\mmatrix}{\mathcal{M}}
\newcommand{\sw}[1]{\boxed{\text{解法 #1}} \ }
\newcommand{\buzhou}[1]{$#1^{\circ} \ $}
\usepackage{ulem}
	\newcommand{\tk}{\uline{\hspace{4em}}}
\newcommand{\pspace}{\vspace{0.5em}}
\usepackage{amsmath,amsfonts}
	\DeclareMathOperator{\spn}{span}
	\DeclareMathOperator{\card}{card}
	\DeclareMathOperator{\ic}{i}
	\DeclareMathOperator{\arccot}{arccot}
	\DeclareMathOperator{\setjianfa}{\textbackslash}
	\DeclareMathOperator{\nul}{null}
	\DeclareMathOperator{\rank}{rank}
	\DeclareMathOperator{\rge}{range}
	\DeclareMathOperator{\sgn}{sgn}
	\DeclareMathOperator{\T}{T}
	\DeclareMathOperator{\ord}{ord}
	\DeclareMathOperator{\sym}{Sym}

\makeatletter
\newcommand*\bigcdot{\mathpalette\bigcdot@{.5}}
\newcommand*\bigcdot@[2]{\mathbin{\vcenter{\hbox{\scalebox{#2}{$\m@th#1\bullet$}}}}}
\makeatother

% Begin the document.
\begin{document}

% Front matter section.
\frontmatter

% Include the title page, which is located in the FrontMatter subfolder.
% This code snippet creates a title page for a book.

% The 'titlepage' environment starts the title page.
\begin{titlepage}
    % The 'colorbox' is used to create a colored background for the book title and subtitle.
    % 'black!5' sets the color to 5% black (a light gray shade).
    \colorbox{black!5}{
        % The first 'parbox' is used to center the title and subtitle within the colored background.
        \parbox[t]{0.975\textwidth}{%
            % The second 'parbox' is used to center the title and subtitle text.
            \parbox[t]{0.95\textwidth}{%
                % Right-align the title and subtitle text, and set it in uppercase and huge font size.
                \raggedleft\vspace{0.75cm}\Huge\scshape
                代数笔记 \\[7.5pt]
                \large\bf Algebra
                \vspace{0.75cm}
            }
        }
    }

    % Vertically space the content evenly, pushing the text to the center of the page.
    \vfill

    % The first 'parbox' is used to display horizontal rules on both sides of the authors' information.
    \parbox[t]{0.95\textwidth}{%
        % Right-align the horizontal rule and add some vertical space above and below it.
        \hfill\rule{0.15\linewidth}{0.5pt}\\[7.5pt]
        % Right-align the authors' names and affiliations.
        \raggedleft
        \textcopyright\:{晨沐公\textsuperscript{\textdagger}}\\[4pt]
        
        % Display the superscript \textdagger symbol and authors' affiliations.
        \normalsize\textsuperscript{\textdagger} 成都市锦江区嘉祥外国语高级中学\\

        % Right-align the second horizontal rule.
        \hfill\rule{0.15\linewidth}{0.5pt}
    }
\end{titlepage}


% Create the book's title page.
\maketitle\pagebreak

% Include the dedication page from the FrontMatter subfolder.
% % This code snippet creates a centered dedication page with two authors' names.

\begin{center}
    % The dedication page has no page number (empty page style).
    \thispagestyle{empty}
    
    % Vertically space the content evenly, pushing the text to the center of the page.
    \vspace*{\fill}
    
    % First author's dedication text in italics.
    \textit{To my someone and someone}
    
    % The first author's name is right-aligned and set in sans-serif small caps.
    \begin{flushright}
        {\sffamily\scshape First Author}
    \end{flushright}
    
    % Add some vertical space between the first and second author.
    \bigskip
    
    % Second author's dedication text in italics.
    \textit{To my someone and someone}
    
    % The second author's name is right-aligned and set in sans-serif small caps.
    \begin{flushright}
        {\sffamily\scshape Second Author}
    \end{flushright}
    
    % Vertically space the content evenly again, pushing any remaining space to the bottom of the page.
    \vspace*{\fill}
\end{center}


% Include the epigraph page from the FrontMatter subfolder.
% This code snippet creates a quote block attributed to an author.

% Vertically space the content evenly, pushing the quote to the center of the page.
\vspace*{\fill}

% Set the font size to \Large (large) and the text style to italics.
\Large\textit{It is not from the benevolence of the butcher, the brewer, or the baker, that we expect our dinner, but from their regard to their own interest. }

% Add some vertical space after the quote.
\bigskip

% The author's name is right-aligned and set in sans-serif small caps.
\begin{flushright}
    \sffamily\scshape Adam Smith
\end{flushright}

% Set the font back to the default (normal font size and style).
\normalfont\normalsize

% Vertically space the content evenly again, pushing any remaining space to the bottom of the page.
\vspace*{\fill}


% Include the foreword page from the FrontMatter subfolder.
\chapter*{前言}

本讲义的大致结构基于Zorich的教材, 作者本着易于理解的原则做了一些调整. 

参考书目如下: 

\begin{enumerate}

\item
B.A.卓里奇.
\newblock {\em 数学分析(第一卷)}.
\newblock 高等教育出版社, 2019.

\item
B.A.卓里奇.
\newblock {\em 数学分析(第二卷)}.
\newblock 高等教育出版社, 2019.

\item
清华大学数学系及丘成桐数学科学中心.
\newblock {\em
  数学分析之课程讲义(丘成桐数学英才班试用)}.
\newblock 2020.

\item
Ayumu.
\newblock {\em 数学分析I}.
\newblock 复旦大学出版社, 2024.

\item
Ayumu.
\newblock {\em 数学分析II}.
\newblock 2024.

\item
Ayumu.
\newblock {\em 数学分析III}.
\newblock 2024.

\item
陈天权.
\newblock {\em 数学分析讲义(第一册)}.
\newblock 北京大学出版社, 2009.

\item
陈天权.
\newblock {\em 数学分析讲义(第二册)}.
\newblock 北京大学出版社, 2010.

\item
陈天权.
\newblock {\em 数学分析讲义(第三册)}.
\newblock 北京大学出版社, 2010.

\item
汪林.
\newblock {\em 数学分析中的问题和反例}.
\newblock 高等教育出版社, 2015.

\end{enumerate}


% Include the preface page from the FrontMatter subfolder.
% \chapter*{序}



\undersign

% Include the acknowledgement page from the FrontMatter subfolder.
% \chapter*{致谢}



\undersign

% Table of contents page.
\tableofcontents

% Main matter section.
\mainmatter


\chapter{线性方程组与行列式}

\section{线性方程组解的情况}

\subsection*{结论}

\begin{theorem}{通过阶梯型矩阵判断解的情况}
	设$n$元线性方程组$Ax=b$. 将其增广矩阵化为阶梯型矩阵: 若出现“$0=d$”型方程则原方程组无解; 否则有解, 此时若阶梯型矩阵的非零行数目$r$等于$n$则有唯一解, 若$r<n$则有无穷多解. 
\end{theorem}

\begin{theorem}{通过系数矩阵与增广矩阵的秩判断解的情况}
	设$n$元线性方程组$Ax=b$, 对应增广矩阵$\tilde{A}$. 则其有解当且仅当$\rank A = \rank \tilde{A}$. 有解时, 若$\rank A = n$则有唯一解, 若$\rank A < n$则有无穷多解. 
\end{theorem}

\begin{corollary}{}
	设$n$元齐次线性方程组$Ax=0$, 若方程个数小于$n$则一定有非零解. 
\end{corollary}

\begin{theorem}{用系数矩阵的行列式判断解的情况}
	设$n$元线性方程组$Ax=b$. 当方程个数为$n$时, 其有唯一解当且仅当$|A| \neq 0$. 
\end{theorem}

\begin{corollary}{}
	设$n$元齐次线性方程组$Ax=0$. 当方程个数为$n$时, 其只有零解当且仅当$|A| \neq 0$. 
\end{corollary}

\subsection*{应用}

\begin{example}{p122,例3}
	证明: 线性方程组的增广矩阵$\tilde{A}$的秩或者等于它的系数矩阵$A$的秩, 或者等于$\rank A +1$. 
\end{example}

\begin{example}{p123,例5}
	证明: 线性方程组$$\begin{cases}
		a_{11}x_1+\cdots +a_{1n}x_n=b_1, \\
		\cdots \\
		a_{m1}x_1+\cdots +a_{mn}x_n=b_m.
	\end{cases}$$
	有解的充分必要条件是下列线性方程组无解: $$\begin{cases}
		a_{11}x_1+\cdots +a_{m1}x_m=0, \\
		\cdots \\
		a_{1n}x_1+\cdots +a_{mn}x_m=0, \\
		b_1x_1 + \cdots + b_mx_m=1. 
	\end{cases}$$
\end{example}

\section{线性方程组解集的结构}

\subsection*{结论}

计算齐次线性方程组的解集: 求出关于自由变量$x_{i_1},\cdots ,x_{i_m}$的解, 再令自由变量取遍$m$个线性无关的向量, 如$(1,\cdots ,0),\cdots ,(0,\cdots ,1)$. 由此得到线性无关的一组解$\eta _1,\cdots ,\eta _m$. 通过这种操作过程, 同时可以证明: 

\begin{theorem}{秩—零化度定理}
	设$n$元其次线性方程组$Ax=0$, 其解空间记作$W$, 则$\dim W = n-\rank A$. 
\end{theorem}

\begin{theorem}{非齐次线性方程组解集的结构}
	设非齐次线性方程组$Ax=b$及其导出组$Ax=0$, 两者的解集记为$U,W$. $\gamma _0$是$Ax=b$的一个特解, 则$$U=\{ \gamma _0 + \eta : \eta \in W \}. $$
\end{theorem}

\begin{corollary}{}
	若$Ax=b$有解, 则该解唯一当且仅当$Ax=0$只有零解. 
\end{corollary}
\begin{remark}
	从线性映射的角度看, 就是说$Ax=b$的解唯一当且仅当$A$是单射. 
\end{remark}

利用行列式, 还可以得到某种意义上的通式: 

\begin{theorem}{Cramer法则}
	设$n$个方程的$n$元线性方程组$Ax=b$, 并记将$A$的第$j$列换成$b$所得矩阵为$B_j$. 当$|A| \neq 0$时, 方程组的唯一解是$$\begin{pmatrix}
		\dfrac{|B_1|}{|A|} & \cdots & \dfrac{|B_n|}{|A|}
	\end{pmatrix}^{\T}.$$
\end{theorem}

\subsection*{应用}

\begin{example}{p128,例3}
	设$n$个方程的$n$元齐次线性方程组的系数矩阵$A$的行列式等于$0$, 并且$A$的$(k,\ell)$元的代数余子式$A_{k\ell} \neq 0$. 证明: $\begin{pmatrix}
		A_{k1} & \cdots & A_{kn}
	\end{pmatrix}^{\T}$是该方程组的一个基础解系. 
\end{example}

\begin{example}{p129,例4}
	设$n-1$个方程的$n$元齐次线性方程组$Bx=0$. 将$B$划去第$j$列得到的$n-1$阶子式记作$D_j$, 令$\eta = \begin{pmatrix}
		D_1 & -D_2 & \cdots & (-1)^{n-1}D_n
	\end{pmatrix}^{\T}$. 若$\eta \neq 0$, 则$\eta$是该方程组的一个基础解系. 
\end{example}

\begin{example}{p135,例2}
	设$\gamma _0$是$n$元线性方程组$Ax=b$的特解, $\eta _1,\cdots ,\eta _t$是$Ax=0$的基础解系, 令$\gamma _i=\gamma _0+\eta _i,i=1,\cdots ,t$. 证明, $Ax=b$的解集为$\{ c_0\gamma _0+c_1\gamma _1+\cdots + c_t\gamma _t:c_0+c_1+\cdots +c_t=1 \}$. 
\end{example}

\section{行列式的计算}

\subsection*{结论}

行列式函数可以有不同的定义方式, 这里直接用完全展开式定义之: 设矩阵$A=(a_{ij})_{n\times n}$, 则$$\det A = \sum_{j_1,\cdots ,j_n} (-1)^{\tau (j_1\cdots j_n)} a_{1j_1} \cdots a_{nj_n}. $$

\begin{proposition}{行列式的性质}
	设$A$是$n$阶方阵, 记$A=\begin{pmatrix}
		u_1 & \cdots & u_n
	\end{pmatrix}$. 
	\begin{enumerate}
		\item $|A^{\T}| = |A|$. 
		\item 倍乘变换: $\det \begin{pmatrix}
		u_1 & \cdots & ku_i & \cdots & u_n
	\end{pmatrix} = k \det \begin{pmatrix}
		u_1 & \cdots & u_i & \cdots & u_n
	\end{pmatrix}$. 
		\item 倍加变换: $\det \begin{pmatrix}
		u_1 & \cdots & u_i & \cdots ku_i+u_j & \cdots & u_n
	\end{pmatrix} = \det \begin{pmatrix}
		u_1 & \cdots & u_i & \cdots u_j & \cdots & u_n
	\end{pmatrix}$. 
		\item 对换变换: $\det \begin{pmatrix}
		u_1 & \cdots & u_j & \cdots & \cdots u_i & \cdots & u_n
	\end{pmatrix} = -\det \begin{pmatrix}
		u_1 & \cdots & u_i & \cdots & \cdots u_j & \cdots & u_n
	\end{pmatrix}$. 
		\item 多线性: $\det \begin{pmatrix}
			u_1 & \cdots & u_i+u'_i & \cdots & u_n
		\end{pmatrix} = \det \begin{pmatrix}
			u_1 & \cdots & u_i & \cdots & u_n
		\end{pmatrix} + \det \begin{pmatrix}
			u_1 & \cdots & u'_i & \cdots & u_n
		\end{pmatrix}$. 
		\item 两行成比例, 行列式的值为$0$. 
	\end{enumerate}
\end{proposition}
\begin{remark}
	作为推论, 初等行变换保持行列式的非零性. 
\end{remark}

在$n$阶方阵$A$中, 划去第$i$行和第$j$列所成的矩阵的行列式称为$(i,j)$元的余子式, 记作$M_{ij}$. 令$A_{ij}=(-1)^{i+j}M_{ij}$为$(i,j)$元的代数余子式. 

\begin{theorem}{行列式按一行展开}
	设$n$阶方阵$A=(a_{ij})$, 则$$|A| = a_{i1}A_{i1} + \cdots + a_{in}A_{in}. $$
\end{theorem}
\begin{remark}
	当$i \neq j$时还有$$a_{i1}A_{j1} + \cdots + a_{in}A_{jn}=0. $$
\end{remark}

进一步地, 考虑在$n$阶方阵$A$中选取第$i_1,\cdots ,i_k$行和第$j_1,\cdots ,j_k$列所成的$k$阶方阵的行列式, 称其为一个$k$阶子式, 记作$\displaystyle A \begin{pmatrix}
	i_1,\cdots ,i_k \\ j_1,\cdots ,j_k
\end{pmatrix}$. 令$\{ i'_1,\cdots ,i'_{n-k} \}=\{ 1,\cdots ,n \} \setminus \{ i_1,\cdots ,i_n \}$, $\{ j'_1,\cdots ,j'_{n-k} \}=\{ 1,\cdots ,n \} \setminus \{ j_1,\cdots ,j_n \}$, 则$\displaystyle A \begin{pmatrix}
	i'_1,\cdots ,i'_{n-k} \\ j'_1,\cdots ,j'_{n-k}
\end{pmatrix}$称为上述子式的余子式. 同样地, 令$$(-1)^{i_1+\cdots +i_k+j_1+\cdots +j_k} A \begin{pmatrix}
	i'_1,\cdots ,i'_{n-k} \\ j'_1,\cdots ,j'_{n-k}
\end{pmatrix}$$为对应的代数余子式. 

\begin{theorem}{行列式按$k$行展开, Laplace}
	设$n$阶方阵$A$, 则$$|A| = \sum_{1<j_1<\cdots <j_k \leq n} A\begin{pmatrix}
	i_1,\cdots ,i_k \\ j_1,\cdots ,j_k
\end{pmatrix} (-1)^{i_1+\cdots +i_k+j_1+\cdots +j_k} A \begin{pmatrix}
	i'_1,\cdots ,i'_{n-k} \\ j'_1,\cdots ,j'_{n-k}
\end{pmatrix}.$$
\end{theorem}

\begin{proposition}{Vandermonde行列式}
	$$\begin{vmatrix}
 1 & 1 & 1 & \cdots & 1\\
 a_1 & a_2 & a_3 & \cdots & a_n\\
 a_1^2 & a_2^2 & a_3^2 & \cdots & a_n^2\\
 \vdots & \vdots & \vdots &  & \vdots \\
 a_1^{n-1} & a_2^{n-1} & a_3^{n-1} & \cdots & a_n^{n-1}
\end{vmatrix} = \prod_{1 \leq j < i \leq n} (a_i-a_j).$$
\end{proposition}

\subsection*{应用}

计算行列式的常见方法: 

\begin{enumerate}
	\item 用完全展开式计算, 适用于行列式内$0$较多的情况. 
	\item 做初等行变换. 
	\item 利用多线性拆成几个行列式的和. 
	\item 按行/列展开, 常用于归纳(递推)法. 
	\item 加边法. 
\end{enumerate}

\begin{example}{p48,例9}
	设$n$阶方阵$A=(a_{ij})$, $J$表示元素均为$1$的$n$阶方阵. 则$$|A+tJ| = |A|+t\sum_{i=1}^{n}\sum_{j=1}^{n} A_{ij}.$$
\end{example}

\begin{example}{p52,例13}
	计算下列$n (\geq 2)$阶行列式: $$D_n = \begin{vmatrix}
 1 & 1 & 1 & \cdots & 1\\
 x_1 & x_2 & x_3 & \cdots & x_n\\
 x_1^2 & x_2^2 & x_3^2 & \cdots & x_n^2\\
 \vdots & \vdots & \vdots &  & \vdots \\
 x_1^{n-2} & x_2^{n-2} & x_3^{n-2} & \cdots & x_n^{n-2} \\
 x_1^{n} & x_2^{n} & x_3^{n} & \cdots & x_n^{n}
\end{vmatrix}.$$
\end{example}

\begin{example}{p54,习题2.4,7}
	计算下列$n (\geq 2)$阶行列式: $$D_n = \begin{vmatrix}
 x & y & \cdots & y & y\\
 z & x & \cdots & y &y \\
 \vdots & \vdots &  &\vdots  &\vdots \\
  z& z & \cdots & x & y\\
 z &z  & \cdots & z &x
\end{vmatrix}.$$
\end{example}

\begin{example}{p55,习题2.4,10}
	计算下列$n (\geq 2)$阶行列式: $$\begin{vmatrix}
 1 & 1 & 1 & \cdots & 1\\
 1 & a_1 & 0 & \cdots &0 \\
 1 & 0 & a_2  & \cdots & 0 \\
 \vdots & \vdots & \vdots &  & \vdots\\
 1 & 0  & 0 & \cdots & a_{n-1}
\end{vmatrix}.$$
\end{example}

\begin{example}{p67,例2}
	设$|A|$是关于$1,\cdots ,n$的Vandermonde行列式, 求: $$A \begin{pmatrix}
		1,\cdots ,n-1 \\ 1,\cdots ,j-1,j+1,\cdots ,n
	\end{pmatrix}.$$
\end{example}







\chapter{$n$维向量空间$\F ^n$}







\chapter{矩阵的运算}

\section{特殊矩阵}




\section{特殊矩阵的秩与行列式}





\chapter{矩阵的相抵与相似}




\chapter{二次型与矩阵的合同}





\chapter{多项式环}




\chapter{线性空间与线性映射}





\chapter{带有度量的线性空间}




\chapter{多重线性代数}


% Appendices section.
% \appendix

% Include the "about" appendix from the BackMatter subfolder.
% \chapter{About the Authors}

\section{First Author}

\blindtext

\section{Second Author}

\blindtext

% Include the "abbreviation" appendix from the BackMatter subfolder.
% \chapter{Abbreviations}

\blindtext

% Include the "notation" appendix from the BackMatter subfolder.
% \chapter{Notation}

\blindtext

% Include the "code" appendix from the BackMatter subfolder.
% \chapter{Supplementary Scripts}

\blindtext

% Include the "glossary" appendix from the BackMatter subfolder.
% \chapter{Glossary}

\blindtext

% Include the "index" appendix from the BackMatter subfolder.
% \chapter{Index}

\blindtext

% Include the "references" appendix from the BackMatter subfolder.
% \bibliography{references.bib}
\bibliographystyle{ieeetr}
\nocite{*}

% End the document.
\end{document}
